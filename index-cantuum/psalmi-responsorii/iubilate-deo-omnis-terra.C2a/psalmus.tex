\SetVerseAnnotation{Sl 66(65),1b.2ab--7.20}

\SetLinkTextL{\Responsorium{Date}}
\SetLinkTextR{\Responsorium{Glorificai-o}}

\def\VersePairs{
  % 3
  {Dícite Deo: ``Quam terribília sunt ópera \MediatioC{tu}[a.] Præ multitúdine virtútis tuæ blandiéntur tibi inimíci \TerminatioCII{tu}i.}%
    {Dizei a Deus: ``Quão assombrosas são vossas \MediatioC{o}[bras!] Pela grandeza da vossa força, vossos inimigos vos lison\TerminatioCII{jei}am.},
  % 4
  {``Omnis terra adóret te et psallat \MediatioC{ti}[bi,] psalmum dicat nómini \TerminatioCII{tu}o''.}%
    {``Toda a terra vos adore e vos cante \MediatioC{sal}[mos,] entoe salmos ao vosso \TerminatioCII{no}me''!},
  % 5
  {Veníte, et vidéte ópera \MediatioC{De}[i,] terríbilis in adinventiónibus super fílios \TerminatioCII{hómi}num.}%
    {Vinde e vede as obras de \MediatioC{Deus}[,] assombroso no que faz pelos filhos dos \TerminatioCII{ho}mens.},
  % 6
  {Convértit mare in \Flexa*{á}[ridam,] et in flúmine pertransíbunt \MediatioC{pe}[de;] ibi lætábimur in \TerminatioCII{ip}so.}%
    {Ele mudou o mar em terra \Flexa*{fir}[me,] e os fez atravessar a pé o \MediatioC{ri}[o;] lá nos alegraremos \TerminatioCII{ne}le.},
  % 7
  {Qui dominátur in virtúte sua in æ\Flexa*{tér}[num,] óculi eius super gentes re\MediatioC{spí}[ciunt;] rebélles non exalténtur in semet\TerminatioCII{íp}sis.}%
    {Ele domina com sua força para \Flexa*{sem}[pre,] seus olhos observam as na\MediatioC{ções}[:] ele preservou nossa alma para a \TerminatioCII{vi}da.},
  % 20
  {Benedíctus Deus, qui non amóvit oratiónem \MediatioC{me}[am] et misericórdiam suam \TerminatioCII{a} me.}%
    {Bendito seja Deus, que não rejeitou minha ora\MediatioC{ção} e não afastou de mim a sua miseri\TerminatioCII{cór}dia.}
}