% chktex-file 1
\subsection{Antífona de Entrada}\label{subsection:tempus-nativitatis/in-nativitatis-domini/introitus}
\MakeChantAntiphon{dominus-dixit-ad-me.2D}
\nobreaksubsection{Salmo 2}
\MakeChantPsalm{psalmi-ad-introitum}{dominus-dixit-ad-me.2D}

%\AllowPageFlush

\subsection[Salmo Responsorial]{Salmo Responsorial \textmd{C 3 g}}\label{subsection:tempus-nativitatis/in-nativitatis-domini/psalmus-responsorius}
\subsubsection{Salmo 109(110)}
\MakeChantPsalmMulti{psalmi-responsorii}{tecum-principatus.C3g}{
  {psalmus-v1}{psalmus-v1-pt},
  {psalmus-v2}{psalmus-v2-pt}
}

\AllowPageFlush

\subsection{Aleluia}\label{subsection:tempus-nativitatis/in-nativitatis-domini/alleluia}
\MakeChantAlleluia{3g}
\nobreaksubsection{Salmo 97(98)}
\MakeChantPsalm{psalmi-ad-alleluia}{cantate-domino-canticum-novum.3g}

\AllowPageFlush

\subsection[Salmo Aleluiático]{Salmo Aleluiático \textmd{C 4}}\label{subsection:tempus-nativitatis/in-nativitatis-domini/psalmus-alleluiaticus}
\subsubsection{Salmo 97(98)}
\begin{rubrica}
  O primeiro {\normalfont\Rbar} pode ser cantado apenas pelo grupo de cantores ou por todos. O segundo {\normalfont\Rbar} é cantado por todos.
\end{rubrica}
\MakeChantPsalm{psalmi-alleluiatici}{cantate-domino-canticum-novum.C4}

\AllowPageFlush

\subsection{Antífona de Ofertório}\label{subsection:tempus-nativitatis/in-nativitatis-domini/offertorium}
\MakeChantAntiphon{laetentur-caeli.4A}
\nobreaksubsection{Salmo 95(96)}
\MakeChantPsalm{psalmi-ad-offertorium}{laetentur-caeli.4A}

\AllowPageFlush

\subsection{Comunhão}\label{subsection:tempus-nativitatis/in-nativitatis-domini/communio}
\MakeChantAntiphon{viderunt-omnes.7a}
\nobreaksubsection{Salmo 97(98)}
\MakeChantPsalm{psalmi-ad-communionem}{viderunt-omnes.7a}

\AllowPageFlush