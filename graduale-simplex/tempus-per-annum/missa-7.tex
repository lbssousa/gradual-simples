% !TeX root = ../fasciculo-3/a4.tex
% chktex-file 1

\subsection{Entrada}\label{subsection:tempus-per-annum/missa-7/introitus}
\MakeChantAntiphonPsalm{qui-habitas.8G}{introitus}

\nobreaksubsection{Salmo Responsorial}
\begin{rubrica}
    Ver Missa IV, página~\pageref{subsection:tempus-per-annum/missa-4/psalmus-responsorius}, ou Missa V, página~\pageref{subsection:tempus-per-annum/missa-5/psalmus-responsorius}.
\end{rubrica}

\nobreaksubsection{Aleluia}
\begin{rubrica}
    Ver Missa IV, página~\pageref{subsection:tempus-per-annum/missa-4/alleluia}, ou Missa V, página~\pageref{subsection:tempus-per-annum/missa-5/alleluia}.
\end{rubrica}

\nobreaksubsection{Salmo Aleluiático}
\begin{rubrica}
    Ver Missa IV, página~\pageref{subsection:tempus-per-annum/missa-4/psalmus-alleluiaticus}, ou Missa V, página~\pageref{subsection:tempus-per-annum/missa-5/psalmus-alleluiaticus}.
\end{rubrica}

\subsection{Ofertório}\label{subsection:tempus-per-annum/missa-7/offertorium}
\MakeChantAntiphonPsalm{beati-omnes.2D}{offertorium}

\AllowPageFlush

\subsection{Comunhão}\label{subsection:tempus-per-annum/missa-7/communio}
\MakeChantAntiphonPsalm{tu-mandasti-domine.7a}{communio}