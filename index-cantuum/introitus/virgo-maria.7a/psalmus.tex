\def\PsalmAnnotation{Sl 46(45),2--11}

\def\VersePairs{
  % 3
  {\Inchoatio{Prop}{té}rea non timébimus, dum tur\MediatioVII{bá}{tur }{ter}[ra,] et transferéntur montes \TerminatioII{in}{ cor }{ma}ris. \Antiphona{Virgo María}[\LinkLA]}%
    {\Inchoatio{Por}[ ]{is}so, não temeremos, se a \MediatioVII{ter}{ra tre}{mer}[,] e se as montanhas afun\TerminatioVII{da}{rem no }{mar}. \Antiphona{Virgem Maria}[\LinkPT]},
  % 4
  {\Inchoatio{Fre}{mant} et intuméscant \MediatioVII{a}{quæ }{ei}[us,] conturbéntur montes in elati\TerminatioVII{ó}{ne }{e}ius. \Antiphona{Virgo María}[\LinkLA]}%
    {\Inchoatio{Mes}{mo} que rujam e se ergam \MediatioVII{su}{as }{á}[guas,] e os montes estremeçam ao \TerminatioVII{seu}{ em}{ba}te. \Antiphona{Virgem Maria}[\LinkPT]},
  % 5
  {\Inchoatio{Flú}{mi}nis rivi lætíficant civi\MediatioVII{tá}{tem }{De}[i,] sancta tabernácu\TerminatioVII{\-la}{ Al}{tís}simi. \Antiphona{Virgo María}[\LinkLA]}%
    {\Inchoatio{Os}[ ]{ca}nais de um rio alegram a ci\MediatioVII{da}{de de }{Deus}[,] as tendas santas \TerminatioVII{do}{ Al}{tís}simo. \Antiphona{Virgem Maria}[\LinkPT]},
  % 6
  {\Inchoatio{De}{us} in médio eius, non \MediatioVII{com}{mo}{vé}[bitur;] adiavábit eam Deus \TerminatioVII{ma}{ne di}{lú}culo. \Antiphona{Virgo María}[\LinkLA]}%
    {\Inchoatio{Deus}[ ]{es}tá em seu meio: ela não se a\MediatioVII{ba}{la}{rá}[;] Desde o amanhercer Deus a so\TerminatioVII{cor}{re}{rá}. \Antiphona{Virgem Maria}[\LinkPT]},
  % 7
  {\Inchoatio{Fre}{mu}érunt gentes, com\MediatioVII{mó}{ta sunt }{re}[gna;] dedit vocem suam, lique\TerminatioVII{fác}{ta est }{ter}ra. \Antiphona{Virgo María}[\LinkLA]}%
    {\Inchoatio{As}[ ]{na}ções se amotinaram, os reinos se mo\MediatioVII{vi}{men}{ta}[\-ram;] Ele fez ouvir a sua voz, e a terra se \TerminatioVII{dis}{sol}{veu}. \Antiphona{Virgem Maria}[\LinkPT]},
  % 8
  {\Inchoatio{Dó}{mi}nus vir\MediatioVII{tú}{tum no}{bís}[cum,] refúgium nobis \TerminatioVII{De}{\-us }{Ia}cob. \Antiphona{Virgo María}[\LinkLA]}%
    {\Inchoatio{O}[ ]{Se}nhor dos exércitos es\MediatioVII{tá}{ co}{nos}[co;] é nosso refúgio o \MediatioVII{Deus}{ de Ja}{có}. \Antiphona{Virgem Maria}[\LinkPT]},
  % 9
  {\Inchoatio{Ve}{ní}te, et vidéte \MediatioVII{ó}{pera }{Dó}[mini] quæ pósuit prodígia \TerminatioVII{su}{per }{ter}ram. \Antiphona{Virgo María}[\LinkLA]}%
    {\Inchoatio{Vin}{de} e contemplai as obras \MediatioVII{do}{ Se}{nhor}[;] os grandes feitos que ele realizou \TerminatioVII{so}{bre a }{ter}ra. \Antiphona{Virgem Maria}[\LinkPT]},
  % 10
  {\Inchoatio{Áu}{fe}ret bella usque ad finem \Flexa{ter}[ræ,] arcum cónteret et con\MediatioVII{frín}{get }{ar}[ma,] et scuta com\TerminatioVII{bú}{ret }{i}gni. \Antiphona{Virgo María}[\LinkLA]}%
    {\Inchoatio{A}{ca}bará com as guerras até os confins do \Flexa{mun}[do,] quebrará o arco e esmaga\MediatioVII{rá}{ as }{ar}[mas;] e queimará os es\TerminatioVII{cu}{dos ao }{fo}go. \Antiphona{Virgem Maria}[\LinkPT]},
  % 11
  {\Inchoatio{Va}{cá}te, et vidéte quóniam \MediatioVII{e}{go sum }{De}[us;] exaltábor in géntibus, exal\TerminatioVII{tá}{bor in }{ter}ra. \Antiphona{Virgo María}[\LinkLA]}%
    {\Inchoatio{Pa}{rai}, e reconhecei que \MediatioVII{eu}{ sou }{Deus}[:] exaltado entre as nações, exal\TerminatioVII{ta}{do na }{ter}ra. \Antiphona{Virgem Maria}[\LinkLA]}
}