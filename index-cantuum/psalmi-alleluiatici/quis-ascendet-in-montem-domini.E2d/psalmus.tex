% !TeX root = ../../../a4.tex
\DeclareDocumentCommand{\Annotation}{}{
  \SetVerseAnnotation{Sl 24(23),3.7--8a.9--10}
}

\SetLinkTextL{\Responsorium{Allelúia}}
\SetLinkTextR{\Responsorium{Aleluia}}

\SetVersePairs{
  % 7
  {Attóllite, portæ, cápita \Flexa*{ve}[stra,] et elevámini, portæ æter\MediatioE{ná}[les,] et introíbit rex \TerminatioEII{gló}riæ.}%
    {Levantai, ó portas, vossos fron\Flexa*{tões}[,] levantai-vos, portas e\MediatioE{ter}[nas,] para que entre o rei da \TerminatioEII{gló}ria!},
  % 8a
  {Quis est iste rex \MediatioE{gló}[riæ?] Dóminus fortis et \TerminatioEII{po}tens.}%
    {Quem é este rei da \MediatioE{gló}[ria?] É o Senhor, o forte e pode\TerminatioEII{ro}so.},
  % 9
  {Attóllite, portæ, cápita \Flexa*{ve}[stra,] et elevámini, portæ æter\MediatioE{ná}[les,] et introíbit rex \TerminatioEII{gló}riæ.}%
    {Levantai, ó portas, vossos fron\Flexa*{tões}[,] levantai-vos, portas e\MediatioE{ter}[nas,] para que entre o rei da \TerminatioEII{gló}ria!},
  % 10
  {Quis est iste rex \MediatioE{gló}[riæ?] Dóminus virtútum ipse est rex \TerminatioEII{gló}riæ.}%
    {Quem é este rei da \MediatioE{gló}[ria?] O Senhor do universo, é ele o rei da \TerminatioEII{gló}ria. \Responsorium{Aleluia}}
}