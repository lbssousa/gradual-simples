% !TeX root = ../../../main.tex
\DeclareDocumentCommand{\Annotation}{}{
  \MakeAnnotation{\CantusID{003262}[Sl 19(18),5]}{Sl 19(18),1--4.6}
}

\def\LinkLA{communia/commune-apostolorum/introitus/antiphona}
\def\LinkPT{\LinkLA-pt}

\def\VersePairs{
  % 2
  {\Inchoatio{Di}{es} diéi erúctat \MediatioII{ver}[bum,] et nox nocti índicat \TerminatioII{sci}{\-én}\-tiam. \Antiphona{In omnem terram}[\LinkLA]}%
    {\Inchoatio{O}[ ]{di}a transmite ao dia a men\MediatioII{sa}[gem] e a noite dá conhecimento a ou\TerminatioII{tra}[ ]{noi}te. \Antiphona{Por tod\Elisio{a} a terra}[\LinkPT]},
  % 3
  {\Inchoatio{Non}[ ]{sunt} loquélæ neque ser\MediatioII{mó}[nes,] quorum non intellegán\TerminatioII{tur}[ ]{vo}ces. \Antiphona{In omnem terram}[\LinkLA]}%
    {\Inchoatio{Não}[ ]{são} falas, nem dis\MediatioII{cur}[sos,] nem se ouve a su\TerminatioII{a}[ ]{voz}. \Antiphona{Por tod\Elisio{a} a terra}[\LinkPT]},
  % 4
  {\Inchoatio{So}{li} pósuit tabernáculum in \Flexa{e}[is,] et ipse tamquam sponsus procédens de thálamo \MediatioII{su}[o,] exsultávit ut gigas ad currén\TerminatioII{dam}[ ]{vi}am. \Antiphona{In omnem terram}[\LinkLA]}%
    {\Inchoatio{A}{li} armou uma tenda para o \Flexa{sol} e sai como um noivo do quarto nupci\MediatioII{al}[,] e exulta como um gigante a percorrer o seu \TerminatioII{ca}{mi}nho. \Antiphona{Por tod\Elisio{a} a terra}[\LinkPT]},
  % 6
  {\Inchoatio{A}[ ]{fí}nibus cælórum egréssio \Flexa{e}[ius,] et occúrsus eius usque ad fines e\MediatioII{ó}[rum,] nec est quod abscondátur a caló\TerminatioII{re}[ ]{e}ius. \Antiphona{In omnem terram}[\LinkLA]}%
    {\Inchoatio{A}[ ]{su}a saída é desde os confins dos \Flexa{céus} e o seu percurso vai até o outro ex\MediatioII{tre}[mo,] e nada pode subtrair-se ao seu \TerminatioII{ca}{lor}. \Antiphona{Por tod\Elisio{a} a terra}[\LinkPT]}
}