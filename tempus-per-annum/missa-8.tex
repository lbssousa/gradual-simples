% !TeX root = ../fasciculo-3/a4.tex
% chktex-file 1

\subsection{Entrada}\label{subsection:tempus-per-annum/missa-8/introitus}
\MakeChantAntiphonPsalm{../index-cantuum/antiphonae-ad-introitum/}{sit-nomen-domini.7c/}

\subsection[Salmo Responsorial]{Salmo Responsorial \textmd{E 4}}\label{subsection:tempus-per-annum/missa-8/psalmus-responsorius}
\MakeChantPsalmOneVerse{../index-cantuum/psalmi-responsorii/}{confitemini-domino.E4/}

\AllowPageFlush

\subsection{Aleluia}\label{subsection:tempus-per-annum/missa-8/alleluia}
\MakeChantAntiphonPsalm{../index-cantuum/alleluia/}{exaltabo-te-deus.8c/}

\AllowPageFlush

\subsection[Salmo Aleluiático]{Salmo Aleluiático \textmd{C 1}}\label{subsection:tempus-per-annum/missa-8/psalmus-alleluiaticus}
\MakeChantPsalmOneVerse{../index-cantuum/psalmi-alleluiatici/}{exaltabo-te-deus.C1/}

\AllowPageFlush

\subsection{Ofertório}\label{subsection:tempus-per-annum/missa-8/offertorium}
\MakeChantAntiphonPsalm{../index-cantuum/antiphonae-ad-offertorium/}{a-solis-ortu.4E/}

\nobreaksubsection{Comunhão}
\begin{rubrica}
    Ver cantos de comunhão à escolha, na página~\pageref{appendix:communio-ad-libitum} e seguintes.
\end{rubrica}