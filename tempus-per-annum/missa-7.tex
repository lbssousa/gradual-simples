% !TeX root = ../fasciculo-3/a4.tex
% chktex-file 1

\subsection{Entrada}\label{subsection:tempus-per-annum/missa-7/introitus}
\def\AntiphonAnnotation{\CantusID{004473}[Sl 123(122),1]}
\def\AntiphonScore{qui-habitas.8G/}
\MakeChantAntiphonPsalm{introitus/}{qui-habitas.8G/}

\nobreaksubsection{Salmo Responsorial}
\begin{rubrica}
  Ver Missa IV, página~\pageref{subsection:tempus-per-annum/missa-4/psalmus-responsorius}, ou Missa V, página~\pageref{subsection:tempus-per-annum/missa-5/psalmus-responsorius}.
\end{rubrica}

\nobreaksubsection{Aleluia}
\begin{rubrica}
  Ver Missa IV, página~\pageref{subsection:tempus-per-annum/missa-4/alleluia}, ou Missa V, página~\pageref{subsection:tempus-per-annum/missa-5/alleluia}.
\end{rubrica}

\nobreaksubsection{Salmo Aleluiático}
\begin{rubrica}
  Ver Missa IV, página~\pageref{subsection:tempus-per-annum/missa-4/psalmus-alleluiaticus}, ou Missa V, página~\pageref{subsection:tempus-per-annum/missa-5/psalmus-alleluiaticus}.
\end{rubrica}

\subsection{Ofertório}\label{subsection:tempus-per-annum/missa-7/offertorium}
\def\AntiphonAnnotation{\CantusID{001587}[Sl 128(127),1]}
\def\AntiphonScore{beati-omnes.2D/}
\MakeChantAntiphonPsalm{offertorium/}{beati-omnes.2D/}

\AllowPageFlush

\subsection{Comunhão}\label{subsection:tempus-per-annum/missa-7/communio}
\def\AntiphonAnnotation{\CantusID{005214}[Sl 119(118),4]}
\def\AntiphonScore{tu-mandasti-domine.7a/}
\MakeChantAntiphonPsalm{communio/}{tu-mandasti-domine.7a/}