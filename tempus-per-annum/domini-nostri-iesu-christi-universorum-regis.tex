% !TeX root = ../fasciculo-3/a4.tex
% chktex-file 1

\subsection{Entrada}\label{subsection:tempus-per-annum/domini-nostri-iesu-christi-universorum-regis/introitus}
\MakeChantAntiphonPsalm{index-cantuum/antiphonae-ad-introitum/}{regnum-eius.8c/}

\AllowPageFlush

\subsection[Salmo Responsorial]{Salmo Responsorial \textmd{E 5}}\label{subsection:tempus-per-annum/domini-nostri-iesu-christi-universorum-regis/psalmus-responsorius}
\MakeChantPsalmTwoVerses{index-cantuum/psalmi-responsorii/}{adorabunt-eum.E5/}

\AllowPageFlush

\subsection{Aleluia}
\MakeChantAntiphonPsalm{index-cantuum/alleluia/}{afferte-domino.3g/}

\AllowPageFlush

\subsection[Salmo Aleluiático]{Salmo Aleluiático \textmd{C 4}}\label{subsection:tempus-per-annum/domini-nostri-iesu-christi-universorum-regis/psalmus-alleluiaticus}
\begin{center}
  \begin{rubrica}
    O primeiro {\normalfont\Rbar} pode ser cantado apenas pelo grupo de cantores ou por todos. O segundo {\normalfont\Rbar} é cantado por todos.
  \end{rubrica}
\end{center}
\MakeChantPsalmOneVerse{index-cantuum/psalmi-alleluiatici/}{afferte-domino.C4/}

\AllowPageFlush

\subsection{Ofertório}\label{subsection:tempus-per-annum/domini-nostri-iesu-christi-universorum-regis/offertorium}
\MakeChantAntiphonPsalm{index-cantuum/antiphonae-ad-offertorium/}{ecce-dedi-te.8G/}

\AllowPageFlush

\subsection{Comunhão}\label{subsection:tempus-per-annum/domini-nostri-iesu-christi-universorum-regis/communio}
\MakeChantAntiphonPsalm{index-cantuum/antiphonae-ad-communionem/}{magnificabitur.7a/}