\SetVerseAnnotation{Sl 46(45),2--6.8--12}

\def\VersePairs{
  % 3
  {\Inchoatio{Prop}{té}rea non timébimus, dum turbátur \MediatioII{ter}[ra,] et transferéntur montes in \TerminatioII{cor}[ ]{ma}ris. \Antiphona{Beáta Mater}[\LinkLA]}%
    {\Inchoatio{Por}[ ]{is}so, não temeremos, se a terra tre\MediatioII{mer}[,] e se as montanhas afundarem \TerminatioII{no}[ ]{mar}. \Antiphona{Bendita Mãe}[\LinkPT]},
  % 4
  {\Inchoatio{Fre}{mant} et intuméscant aquæ \MediatioII{ei}[us,] conturbéntur montes in elatió\TerminatioII{ne}[ ]{e}ius. \Antiphona{Beáta Mater}[\LinkLA]}%
    {\Inchoatio{Mes}{mo} que rujam e se ergam suas \MediatioII{á}[guas,] e os montes estremeçam ao seu \TerminatioII{em}{ba}te. \Antiphona{Bendita Mãe}[\LinkPT]},
  % 5
  {\Inchoatio{Flú}{mi}nis rivi lætíficant civitátem \MediatioII{De}[i,] sancta tabernácula \TerminatioII{Al}{tís}simi. \Antiphona{Beáta Mater}[\LinkLA]}%
    {\Inchoatio{Os}[ ]{ca}nais de um rio alegram a cidade de \MediatioII{Deus}[,] as tendas santas do \TerminatioII{Al}{tís}simo. \Antiphona{Bendita Mãe}[\LinkPT]},
  % 6
  {\Inchoatio{De}{us} in médio eius, non commo\MediatioII{vé}[bitur;] adiavábit eam Deus mane \TerminatioII{di}{lú}culo. \Antiphona{Beáta Mater}[\LinkLA]}%
    {\Inchoatio{Deus}[ ]{es}tá em seu meio: ela não se abala\MediatioII{rá}[;] Desde o amanhercer Deus a socor\TerminatioII{re}{rá}. \Antiphona{Bendita Mãe}[\LinkPT]},
  % 8
  {\Inchoatio{Dó}{mi}nus virtútum no\MediatioII{bís}[cum,] refúgium nobis De\-\TerminatioII{us}[ ]{Ia}cob. \Antiphona{Beáta Mater}[\LinkLA]}%
    {\Inchoatio{O}[ ]{Se}nhor dos exércitos está co\MediatioII{nos}[co;] é nosso refúgio o Deus de \TerminatioII{Ja}{có}. \Antiphona{Bendita Mãe}[\LinkPT]},
  % 9
  {\Inchoatio{Ve}{ní}te, et vidéte ópera \MediatioII{Dó}[mini] quæ pósuit prodígia su\TerminatioII{per}[ ]{ter}ram. \Antiphona{Beáta Mater}[\LinkLA]}%
    {\Inchoatio{Vin}{de} e contemplai as obras do Se\MediatioII{nhor}[;] os grandes feitos que ele realizou sobre \TerminatioII{a}[ ]{ter}ra. \Antiphona{Bendita Mãe}[\LinkPT]},
  % 10
  {\Inchoatio{Áu}{fe}ret bella usque ad finem \Flexa{ter}[ræ,] arcum cónteret et confrínget \MediatioII{ar}[ma,] et scuta combú\TerminatioII{ret}[ ]{i}gni. \Antiphona{Beáta Mater}[\LinkLA]}%
    {\Inchoatio{A}{ca}bará com as guerras até os confins do \Flexa{mun}[do,] quebrará o arco e esmagará as \MediatioII{ar}[mas;] e queimará os escudos \TerminatioII{ao}[ ]{fo}go. \Antiphona{Bendita Mãe}[\LinkPT]},
  % 11
  {\Inchoatio{Va}{cá}te, et vidéte quóniam ego sum \MediatioII{De}[us;] exaltábor in géntibus, exaltábor \TerminatioII{in}[ ]{ter}ra. \Antiphona{Beáta Mater}[\LinkLA]}%
    {\Inchoatio{Pa}{rai}, e reconhecei que eu sou \MediatioII{Deus}[:] exaltado entre as nações, exaltado \TerminatioII{na}[ ]{ter}ra. \Antiphona{Bendita Mãe}[\LinkLA]},
  % 12
  {\Inchoatio{Dó}{mi}nus virtútum no\MediatioII{bis}[cum,] refúgium nobis De\-\TerminatioII{us}[ ]{Ia}cob. \Antiphona{Beáta Mater}[\LinkLA]}%
    {\Inchoatio{O}[ ]{Se}nhor dos exércitos está co\MediatioII{nos}[co;] é nosso refúgio o Deus de \TerminatioII{Ja}{có}. \Antiphona{Bendita Mãe}[\LinkPT]}
}