% !TeX root = ../../../main.tex
\DeclareDocumentCommand{\Annotation}{}{
  \MakeAnnotation{\CantusID{004404}[Mt 25,6--7]}{Sl 119(118),1--4.7.12--15}
}

\def\LinkLA{communia/commune-sanctarum/communio-2/antiphona}
\def\LinkPT{\LinkLA-pt}

\def\VersePairs{
  % 2
  {\Inchoatio{Be}{á}ti, qui servant testimó\MediatioIV{ni}{a}[ ]{e}[ius,] in toto corde \TerminatioIV{ex}{quí}{runt}[ ]{e}um. \Antiphona{Prudentes vírgines}[\LinkLA]}%
    {\Inchoatio{Fe}{li}zes os que guardam seus en\MediatioIV{si}{na}{men}[tos] e de todo o co\TerminatioIV{ra}{ção}[ ]{o}[ ]{bus}cam. \Antiphona{Virgens prudentes}[\LinkPT]},
  % 3
  {\Inchoatio{Non}[ ]{e}nim operáti sunt i\MediatioIV{ni}{qui}{tá}[tem,] in viis eius \TerminatioIV{am}{bu}{la}{vé}runt. \Antiphona{Prudentes vírgines}[\LinkLA]}%
    {\Inchoatio{Não}[ ]{pra}ticam a i\MediatioIV{ni}{qui}{da}[de,] mas andam \TerminatioIV{em}[ ]{seus}[ ]{ca}{mi}\-nhos. \Antiphona{Virgens prudentes}[\LinkPT]},
  % 4
  {\Inchoatio{Tu}[ ]{man}dásti man\MediatioIV{dá}{ta}[ ]{tu}[a] cus\TerminatioIV{to}{dí}{ri}[ ]{ni}mis. \Antiphona{Prudentes vírgines}[\LinkLA]}%
    {\Inchoatio{Vós}[ ]{or}denastes vos\MediatioIV{sos}[ ]{pre}{cei}[tos] para que sejam exata\-\TerminatioIV{men}{\-te}[ ]{cum}{\-pri}\-dos. \Antiphona{Virgens prudentes}[\LinkPT]},
  % 7
  {\Inchoatio{Con}{fi}tébor tibi in directi\MediatioIV{ó}{ne}[ ]{cor}[dis,] in eo quod dídici iúdici ius\TerminatioIV{tí}{ti}{æ}[ ]{tu}æ. \Antiphona{Prudentes vírgines}[\LinkLA]}%
    {\Inchoatio{Eu}[ ]{vos} darei graças com a retidão do \MediatioIV{co}{ra}{ção}[,] quan\-do tiver aprendido os julgamentos da \TerminatioIV{vos}{sa}[ ]{jus}{ti}ça. \Antiphona{Virgens prudentes}[\LinkPT]},
  % 12
  {\Inchoatio{Be}{ne}díc\MediatioIV{tus}[ ]{es}[, ]{Dó}[mine] dace me iustifica\TerminatioIV{ti}{ó}{nes}[ ]{tu}as. \Antiphona{Prudentes vírgines}[\LinkLA]}%
    {\Inchoatio{Ben}{di}to sois \MediatioIV{vós}[, ]{Se}{nhor}[!] Ensinai-me os vossos \TerminatioIV{jus}{tos}[ ]{de}{cre}tos. \Antiphona{Virgens prudentes}[\LinkPT]},
  % 13
  {\Inchoatio{In}[ ]{lá}biis meis \MediatioIV{nu}{me}{rá}[vi] ómnia iudíci\TerminatioIV{a}[ ]{o}{ris}[ ]{tu}i. \Antiphona{Prudentes vírgines}[\LinkLA]}%
    {\Inchoatio{Com}[ ]{os} meus lábios \MediatioIV{e}{nun}{ci}[o] todos os julgamentos \TerminatioIV{de}[ ]{vos}{sa}[ ]{bo}ca. \Antiphona{Virgens prudentes}[\LinkPT]},
  % 14
  {\Inchoatio{In}[ ]{vi}a testimoniórum tuórum \MediatioIV{de}{lec}{tá}[tus sum,] sicut in óm\TerminatioIV{ni}{bus}[ ]{di}{ví}tiis. \Antiphona{Prudentes vírgines}[\LinkLA]}%
    {\Inchoatio{A}{le}gro-me no caminho dos vossos en\MediatioIV{si}{na}{men}[tos,] tanto como em to\TerminatioIV{das}[ ]{as}[ ]{ri}{que}zas. \Antiphona{Virgens prudentes}[\LinkPT]},
  % 15
  {\Inchoatio{In}[ ]{man}dátis tuis \MediatioIV{e}{xer}{cé}[bor,] et considerá\TerminatioIV{bo}[ ]{vi}{as}[ ]{tu}as. \Antiphona{Prudentes vírgines}[\LinkLA]}%
    {\Inchoatio{Que}{ro} meditar nos vos\MediatioIV{sos}[ ]{pre}{cei}[tos,] considerando os \TerminatioIV{vos}{sos}[ ]{ca}{\-mi}nhos. \Antiphona{Virgens prudentes}[\LinkPT]}
}