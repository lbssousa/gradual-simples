% !TeX root = ../../../a4.tex
\DeclareDocumentCommand{\Annotation}{}{
  \SetVerseAnnotation{\CantusID[?]{001744}[Sl 92(91),2]}{Sl 92(91),3.5--6.13}
}

\def\LinkLA{tempus-quadragesimae/dominica-4/offertorium/antiphona}
\def\LinkPT{\LinkLA-pt}

\def\VersePairs{
  % 5
  {\Inchoatio{Qui}{a} delectásti me, Dómine, in fac\MediatioIV{tú}{ra}[ ]{tu}[a,] et in opéribus mánuum tuárum \TerminatioIV{ex}{sul}{tá}bo. \Antiphona{Bonum est}[\LinkLA]}%
    {\Inchoatio{Pois}[ ]{vós} me alegrais, Senhor, com os \MediatioIV{vos}{sos}[ ]{fei}[tos,] e eu exulto com as obras \TerminatioIV{de}[ ]{vos}{sas}[ ]{mãos}. \Antiphona{Bom é}[\LinkPT]},
  % 6
  {\Inchoatio{Quam}[ ]{ma}gnificáta sunt ópera \MediatioIV{tu}{a}[, ]{Dó}[mine:] nimis profúndæ factæ sunt cogita\TerminatioIV{ti}{ó}{nes}[ ]{tu}æ. \Antiphona{Bonum est}[\LinkLA]}%
    {\Inchoatio{Quão}[ ]{gran}diosas as vossas o\MediatioIV{bras}[, ]{Se}{nhor}[,] como são profundos os vos\TerminatioIV{sos}[ ]{pen}{sa}{men}tos. \Antiphona{Bom é}[\LinkPT]},
  % 13
  {\Inchoatio{Ju}{stus} ut pal\MediatioIV{ma}[ ]{flo}{ré}[bit,] sicut cedrus Lí\TerminatioIV{ba}{ni}[ ]{suc}{\-cré}scet. \Antiphona{Bonum est}[\LinkLA]}%
    {\Inchoatio{O}[ ]{jus}to, porém, florescerá como \MediatioIV{a}[ ]{pal}{mei}[ra,] e como o cedro do Líba\TerminatioIV{no}[ ]{cres}{ce}{rá}. \Antiphona{Bom é}[\LinkPT]}
}