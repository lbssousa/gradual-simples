% !TeX root = ../../a4.tex
% chktex-file 1

\subsection{Entrada}\label{subsection:hebdomada-sancta/missa-in-cena-domini/psalmi-ad-introitum}
\MakeChantAntiphonPsalm{nos-autem-gloriari.7c}{psalmi-ad-introitum}

\AllowPageFlush

\subsection{Entrada (opcional)}\label{subsection:hebdomada-sancta/missa-in-cena-domini/psalmi-ad-introitum-ad-libitum}
\MakeChantAntiphonPsalm{sacerdos-in-aeternum.1f}[2]{psalmi-ad-introitum}

\AllowPageFlush

\subsection[Salmo Responsorial I]{Salmo Responsorial I \textmd{D \protect\GreStar}}\label{subsection:hebdomada-sancta/missa-in-cena-domini/psalmus-responsorius-1}
\MakeChantPsalmTwoVerses{psalmi-responsorii}{dominus-pascit-me.DS}

\AllowPageFlush

\subsection[Salmo Responsorial II]{Salmo Responsorial II \textmd{E 5}}\label{subsection:hebdomada-sancta/missa-in-cena-domini/psalmus-responsorius-2}
\MakeChantPsalmTwoVerses{psalmi-responsorii}{exaltabo-te.E5}

\AllowPageFlush

\subsection{Lava-Pés}\label{subsection:hebdomada-sancta/missa-in-cena-domini/psalmi-ad-lotionem-pedum}
\MakeChantAntiphonPsalm{mandatum-novum-do-vobis.3a}{psalmi-ad-lotionem-pedum}

\subsection{Ofertório}\label{subsection:hebdomada-sancta/missa-in-cena-domini/psalmi-ad-offertorium}
\begin{annotation}
  Missal Romano, 2ª Edição Típica (CNBB)
\end{annotation}
\MakeChantLongPsalm{varia}{ubi-caritas}{
  {psalmus-v1}{psalmus-v1-pt},
  {psalmus-v2}{psalmus-v2-pt},
  {psalmus-v3}{psalmus-v3-pt}
}

\AllowPageFlush

\subsection{Comunhão}\label{subsection:hebdomada-sancta/missa-in-cena-domini/psalmi-ad-communionem}
\MakeChantAntiphonPsalm{calicem.2D}{psalmi-ad-communionem}

\AllowPageFlush

\subsection{Transladação do Santíssimo Sacramento}\label{subsection:hebdomada-sancta/missa-in-cena-domini/hymnus-ad-translatione-ssmi-sacramenti}
\begin{annotation}
  Missal Romano, 2ª Edição Típica (CNBB)
\end{annotation}
\MakeChantLongPsalm{hymni}{pange-lingua}{
  {psalmus-v1}{psalmus-v1-pt},
  {psalmus-v2}{psalmus-v2-pt},
  {psalmus-v3}{psalmus-v3-pt},
  {psalmus-v4}{psalmus-v4-pt}
}
\begin{rubrica}
  Repetem-se os versos acima, ou cantam-se outros hinos eucarísticos, durante a procissão, se necessário. Após a chegada ao altar da reposição, canta-se os dois últimos versos do hino, a seguir:
\end{rubrica}
\MakeChantLongPsalm*{hymni}{pange-lingua}{
  {psalmus-v5}{psalmus-v5-pt},
  {psalmus-v6}{psalmus-v6-pt}
}