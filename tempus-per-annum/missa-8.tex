% !TeX root = ../fasciculo-3/a4.tex
% chktex-file 1

\subsection{Entrada}\label{subsection:tempus-per-annum/missa-8/introitus}
\def\AntiphonAnnotation{{\CantusID{004971}[Sl 113(112),2]}}
\def\AntiphonScore{sit-nomen-domini.7c/}
\MakeChantAntiphonPsalm{introitus/}{sit-nomen-domini.7c/}

\subsection[Salmo Responsorial]{Salmo Responsorial \textmd{E 4}}\label{subsection:tempus-per-annum/missa-8/psalmus-responsorius}
\MakeChantPsalmOneVerse{psalmi-responsorii/}{confitemini-domino.E4/}

\AllowPageFlush

\subsection{Aleluia}\label{subsection:tempus-per-annum/missa-8/alleluia}
\def\AntiphonScore{alleluia.8c.2/}
\MakeChantAntiphonPsalm{alleluia/}{exaltabo-te-deus.8c/}

\AllowPageFlush

\subsection[Salmo Aleluiático]{Salmo Aleluiático \textmd{C 1}}\label{subsection:tempus-per-annum/missa-8/psalmus-alleluiaticus}
\MakeChantPsalmOneVerse{psalmi-alleluiatici/}{exaltabo-te-deus.C1/}

\AllowPageFlush

\subsection{Ofertório}\label{subsection:tempus-per-annum/missa-8/offertorium}
\def\AntiphonAnnotation{\CantusID{020003}[Sl 113(112),3]}
\def\AntiphonScore{a-solis-ortu.4E/}
\MakeChantAntiphonPsalm{offertorium/}{a-solis-ortu.4E/}

\nobreaksubsection{Comunhão}
\begin{rubrica}
  Ver cantos de comunhão à escolha, na página~\pageref{appendix:communio-ad-libitum} e seguintes.
\end{rubrica}