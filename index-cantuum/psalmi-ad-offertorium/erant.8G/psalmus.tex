% chktex-file 38
\SetLinkTextL{\Antiphona{E\-rant}}
\SetLinkTextR{\Antiphona{Estavam}}

\SetFirstVersePair{
  {\VSup{2}\Inchoatio{Ex}{spéc}tans exspectávi \MediatioVIII{Dó}[minum,] et in\TerminatioVIII{tén}{dit}[ ]{mi}hi.}%
    {\Inchoatio{Es}{pe}rei firmemente no Se\MediatioVIII{nhor}[,] e ele inclinou-se \TerminatioVIII{pa}{ra}[ ]{mim}.}
}

\SetVersePairs{
  {\VSup{7b}\Inchoatio{Ho}{lo}cáustum et pro peccáto non postu\MediatioVIII{lá}[sti,] \VSup{8}tunc dixi: ``\TerminatioVIII{Ec}{ce}[ ]{vé}nio.}%
    {\Inchoatio{Não}[ ]{pe}distes holocausto nem vítima pelo pe\MediatioVIII{ca}[do,] e então eu disse: ``\TerminatioVIII{Eis}[ ]{que}[ ]{ve}nho.},
  {``\Inchoatio{In}[ ]{vo}lúmine libri scriptum est \MediatioVIII{de}[ me] \VSup{9}fácere volun\TerminatioVIII{tá}{tem}[ ]{tu}am.}%
    {\Inchoatio{No}[ ]{li}vro está escrito a meu res\MediatioVIII{pei}[to:] Fazer a vos\TerminatioVIII{sa}[ ]{von}{\-ta}\-de.},
  {``\Inchoatio{De}{us} meus, \MediatioVIII{vó}[lui,] et lex tua in præcór\TerminatioVIII{di}{is}[ ]{me}is''.}%
    {``\Inchoatio{Meu}[ ]{Deus}, eu \MediatioVIII{que}[ro;] a vossa lei está no fundo do meu \TerminatioVIII{co}{ra}{ção}''.},
  {\VSup{17}\Inchoatio{Ex}{súl}tent et læténtur in te omnes quæ\MediatioVIII{rén}[tes te,] et dicant semper: ``Magnificétur Dóminus'' qui díligunt salu\TerminatioVIII{tá}{re}[ ]{tu}um.}%
    {\Inchoatio{E}{xul}tem e em vós se alegrem todos os que vos \MediatioVIII{bus}[\-cam,] e digam sempre: ``O Senhor seja engrandecido!'', os que amam a vossa \TerminatioVIII{sal}{va}{ção}.}
}