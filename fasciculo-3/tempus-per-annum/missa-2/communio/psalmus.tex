% !TeX root = ../../../main.tex
\DeclareDocumentCommand{\Annotation}{}{
  \MakeAnnotation{\CantusID[?]{020072}[Sl 13(12),6]}{Sl 13(12)}
}

\def\LinkLA{tempus-per-annum/missa-2/communio/antiphona}
\def\LinkPT{\LinkLA-pt}

\def\VersePairs{
  % 3
  {\Inchoatio{Us}{qué}quo ponam consília in ánima \Flexa{me}[a,] dolórem in corde meo, per \MediatioII{di}[em?] Usquéquo exaltaábitur inimícus me\TerminatioII{us}[ ]{su}per me? \Antiphona{Cantábo Dómino}[\LinkLA]}%
    {\Inchoatio{A}{té} quando revolverei planos em minha \Flexa{al}[ma,] e a dor no meu coração cada \MediatioII{di}[a?] Até quando meu prevalecerá meu inimigo con\TerminatioII{tra}[ ]{mim}? \Antiphona{Cantarei ao Senhor}[\LinkPT]},
  % 4
  {\Inchoatio{Ré}{spi}ce et exáudi me, Dómine Deus \MediatioII{me}[us.] Illúmina óculos meos, nequándo obdórmiam \TerminatioII{in}[ ]{mor}te. \Antiphona{Cantábo Dómino}[\LinkLA]}%
    {\Inchoatio{O}{lhai}-me e escutai-me, Senhor, meu \MediatioII{Deus}[!] Iluminais meus olhos, para que eu não adormeça \TerminatioII{na}[ ]{mor}te. \Antiphona{Cantarei ao Senhor}[\LinkPT]},
  % 5
  {\Inchoatio{Ne}{quán}do dicat inimícus \Flexa{me}[us:] ``Præválui advérsus \MediatioII{e}[um'';] neque exsúltent qui tríbulant me, si mo\TerminatioII{tus}[ ]{fú}ero. \Antiphona{Cantábo Dómino}[\LinkLA]}%
    {\Inchoatio{Que}[ ]{meu} inimigo não venha a di\Flexa{zer}[:] ``Prevaleci contra \MediatioII{e}[le'',] e os que me atribulam não exultem, se eu \TerminatioII{ca}{ir}. \Antiphona{Cantarei ao Senhor}[\LinkPT]},
  % 6ab
  {\Inchoatio{E}{go} autem in misericórdia tua spe\MediatioII{rá}[vi,] exsultábit cor meum in salutá\TerminatioII{ri}[ ]{tu}o. \Antiphona{Cantábo Dómino}[\LinkLA]}%
    {\Inchoatio{Mas}[ ]{eu} confio na vossa miseri\MediatioII{cór}[dia:] meu coração se alegrará com a vossa sal\TerminatioII{va}{ção}. \Antiphona{Cantarei ao Senhor}[\LinkPT]}
}