% !TeX root = ../../../a4.tex
\DeclareDocumentCommand{\Annotation}{}{
  \SetVerseAnnotation{\CantusID{003515}[Sl 119(118),154]}{Sl 22(21),2--3.21--22}
}

\def\LinkLA{tempus-quadragesimae/dominica-5/offertorium/antiphona}
\def\LinkPT{\LinkLA-pt}

\def\VersePairs{
  % 3
  {\Inchoatio{De}{us} meus, clamo per diem, et non ex\MediatioVIII{áu}[dis,]
      et nocte, et non est ré\TerminatioVIII{qui}{es}[ ]{mi}hi. \Antiphona{Iudíca}[\LinkLA]}%
    {\Inchoatio{Meu}[ ]{Deus}, clamo de dia, e não me aten\MediatioVIII{deis}[;] de noite, e não encon\TerminatioVIII{tro}[ ]{sos}{se}go. \Antiphona{Julgai}[\LinkPT]},
  % 21
  {\Inchoatio{E}{ru}e a frámea ánimam \MediatioVIII{me}[am] et de manu canis únicam meam. \Antiphona{Iudíca}[\LinkLA]}%
    {\Inchoatio{Li}{vrai} da espada a minha \MediatioVIII{al}[ma] e, das unhas dos cães, a \TerminatioVIII{mi}{nha}[ ]{vi}da. \Antiphona{Julgai}[\LinkPT]},
  % 22
  {\Inchoatio{Sal}{va} me ex ore le\MediatioVIII{ó}[nis] et a córnibus unicórnium humili\TerminatioVIII{tá}{tem}[ ]{me}am. \Antiphona{Iudíca}[\LinkLA]}%
    {\Inchoatio{Sal}{vai}-me da boca do le\MediatioVIII{ão} e dos chifres dos búfalos, na minha hu\TerminatioVIII{mi}{lha}{ção}. \Antiphona{Julgai}[\LinkPT]}
}