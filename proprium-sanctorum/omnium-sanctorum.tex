% !TeX root = ../../a4.tex
% chktex-file 1

\subsection{Entrada}\label{subsection:proprium-sanctorum/omnium-sanctorum/introitus}
\def\AntiphonAnnotation{\CantusID{002235}[Sl 31(30),24]}
\def\AntiphonScore{diligite-dominum.2D/}
\MakeChantAntiphonPsalm{introitus/}{\AntiphonScore}

\subsection[Salmo Responsorial]{Salmo Responsorial \textmd{C 3 g}}\label{subsection:proprium-sanctorum/omnium-sanctorum/psalmus-responsorius}
\MakeChantPsalmTwoVerses{psalmi-responsorii/}{laudate-dominum-in-sanctis-eius.C3g/}

\AllowPageFlush

\subsection{Aleluia}\label{subsection:proprium-sanctorum/omnium-sanctorum/alleluia}
\def\AntiphonScore{alleluia.8c.1/}
\MakeChantAntiphonPsalm{alleluia/}{exsultate-iusti-in-domino.8c/}

\subsection[Salmo Aleluiático]{Salmo Aleluiático \textmd{C 4}}\label{subsection:proprium-sanctorum/omnium-sanctorum/psalmus-alleluiaticus}
\begin{rubrica}
  O primeiro {\normalfont\Rbar} pode ser cantado apenas pelo grupo de cantores ou por todos. O segundo {\normalfont\Rbar} é cantado por todos.
\end{rubrica}
\MakeChantPsalmOneVerse{psalmi-alleluiatici/}{exsultate-iusti-in-domino.C4/}

\AllowPageFlush

\subsection{Ofertório}\label{subsection:proprium-sanctorum/omnium-sanctorum/offertorium}
\def\AntiphonAnnotation{\CantusID{003900}[Is 56,5; Is 35,10]}
\def\AntiphoScore{nomen-sempiternum.7a/}
\MakeChantAntiphonPsalm{offertorium/}{\AntiphonScore}

\AllowPageFlush

\subsection{Comunhão}\label{subsection:proprium-sanctorum/omnium-sanctorum/communio}
\def\AntiphonAnnotation{\CantusID{001588}[Mt 5,9.8]}
\def\AntiphonScore{beati-pacifici.1f/}
\MakeChantAntiphonPsalm{communio/}{\AntiphonScore}