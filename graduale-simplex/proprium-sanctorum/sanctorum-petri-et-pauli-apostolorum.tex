% !TeX root = ../../a4.tex
% chktex-file 1

\subsection{Entrada}\label{subsection:proprium-sanctorum/sanctorum-petri-et-pauli-apostolorum/psalmi-ad-introitum}
\MakeChantAntiphonPsalm{misit-dominus.7c2}{psalmi-ad-introitum}

\subsection[Salmo Responsorial]{Salmo Responsorial \textmd{C 3 g}}\label{subsection:proprium-sanctorum/sanctorum-petri-et-pauli-apostolorum/psalmus-responsorius}
\MakeChantPsalmThreeVerses{psalmi-responsorii}{constitues-eos-principes.C3g}

\AllowPageFlush

\subsection{Aleluia}\label{subsection:proprium-sanctorum/sanctorum-petri-et-pauli-apostolorum/alleluia}
\MakeChantAlleluiaPsalm{3g}{in-convertendo-dominus.3g}

\subsection[Salmo Aleluiático]{Salmo Aleluiático \textmd{C 4}}\label{subsection:proprium-sanctorum/sanctorum-petri-et-pauli-apostolorum/psalmus-alleluiaticus}
\begin{rubrica}
    O primeiro {\normalfont\Rbar} pode ser cantado apenas pelo grupo de cantores ou por todos. O segundo {\normalfont\Rbar} é cantado por todos.
\end{rubrica}
\MakeChantPsalmOneVerse{psalmi-alleluiatici}{in-convertendo-dominus.C4}

\AllowPageFlush

\subsection{Ofertório}\label{subsection:proprium-sanctorum/sanctorum-petri-et-pauli-apostolorum/psalmi-ad-offertorium}
\MakeChantAntiphonPsalm{constitues-eos-principes.7a}{psalmi-ad-offertorium}

\AllowPageBreak

\subsection{Comunhão}\label{subsection:proprium-sanctorum/sanctorum-petri-et-pauli-apostolorum/psalmi-ad-communionem}
\MakeChantAntiphonPsalm{tu-es-petrus.7c}{psalmi-ad-communionem}