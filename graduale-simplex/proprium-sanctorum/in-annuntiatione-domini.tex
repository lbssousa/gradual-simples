% !TeX root = ../../a4.tex
% chktex-file 1

\begin{rubrica}
  Quando o dia 25 de março está situado dentro da Semana Santa ou da Oitava da Páscoa, esta solenidade é transferida para o primeiro dia livre após o II Domingo do Tempo Pascal.
\end{rubrica}

\subsection{Entrada}\label{subsection:proprium-sanctorum/in-annuntiatione-domini/introitus}
\MakeChantAntiphonPsalm{angelus-domini.1f}{psalmi-ad-introitum}

\nobreaksubsection{Salmo Responsorial}
\begin{rubrica}
  Ver Solenidade da Imaculada Conceição de Nossa Senhora, página~\pageref{subsection:proprium-sanctorum/in-conceptione-immaculata-bmv/psalmus-responsorius}
\end{rubrica}

\nobreaksubsection{Aleluia}
\begin{rubrica}
  Ver Solenidade da Imaculada Conceição de Nossa Senhora, página~\pageref{subsection:proprium-sanctorum/in-conceptione-immaculata-bmv/alleluia}
\end{rubrica}

\nobreaksubsection{Salmo Aleluiático}
\begin{rubrica}
  Ver Solenidade da Imaculada Conceição de Nossa Senhora, página~\pageref{subsection:proprium-sanctorum/in-conceptione-immaculata-bmv/psalmus-alleluiaticus}
\end{rubrica}

\AllowPageFlush

\subsection{Ofertório}\label{subsection:proprium-sanctorum/in-annuntiatione-domini/offertorium}
\MakeChantAntiphonPsalm{ave-maria.1g}[2]{psalmi-ad-offertorium}

\AllowPageBreak

\subsection{Comunhão}\label{subsection:proprium-sanctorum/in-annuntiatione-domini/communio}
\MakeChantAntiphonPsalm{ecce-ancilla-domini.8c}{psalmi-ad-communionem}

\AllowPageFlush