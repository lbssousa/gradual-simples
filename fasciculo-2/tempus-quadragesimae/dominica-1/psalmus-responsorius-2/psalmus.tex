% !TeX root = ../../../main.tex
\DeclareDocumentCommand{\Annotation}{}{
  \MakeAnnotation{}{Sl 41(40),5--10}
}

\def\LinkLA{tempus-quadragesimae/dominica-1/psalmus-responsorius-2/psalmus-v1}
\def\LinkPT{\LinkLA-pt}

\def\VersePairs{
  % 6
  {Inimíci mei dixérunt mala \MediatioDI{mi}[hi:] ``Quando moriétur, et períbit nomen, \TerminatioDI{e}ius?'' \Responsorium{Quia}[\LinkLA]}%
    {Meus inimigos dizem maldades a meu res\MediatioDI{pei}[to:] ``Quan\-do é que vai morrer, e se extinguirá o seu \TerminatioDI{no}me?'' \Responsorium{Pois}[\LinkPT]},
  % 7
  {Et si ingrediebátur, ut visitáret, vana loque\Respirare{bá}[tur;] cor eius congregábat iniquitátem \MediatioDI{si}[bi,] egrediebátur foras et detra\TerminatioDI{hé}bat. \Responsorium{Quia}[\LinkLA]}%
    {Se alguém entrava, para me visitar, falava \Respirare{fal}[so:] seu coração se enchia de mal\MediatioDI{da}[de,] e, ao sair, ainda caluni\TerminatioDI{a}va. \Responsorium{Pois}[\LinkPT]},
  % 8
  {Simul advérsum me susurrábant omnes inimíci \MediatioDI{me}[i;] advérsum me cogitábant mala \TerminatioDI{mi}{hi:} \Responsorium{Quia}[\LinkLA]}%
    {Juntos murmuravam contra mim meus ini\MediatioDI{mi}[gos,] contra mim planejavam o \TerminatioDI{mal}: \Responsorium{Pois}[\LinkPT]},
  % 9
  % Nota: substituído o termo hebraico original
  % da Bíblia oficial da CNBB "algo de Belial" por "um mal mortal",
  % tal como se encontra na Bíblia Ave-Maria.
  {``Malefícium effúsum est in \MediatioDI{e}[o;] et qui decúmbit non adíciet ut re\TerminatioDI{súr}gat''. \Responsorium{Quia}[\LinkLA]}%
    {``Um mal mortal o atin\MediatioDI{giu}[;] agora que deitou, não vai mais levan\TerminatioDI{tar}-se.'' \Responsorium{Pois}[\LinkPT]},
  % 10
  {Sed et homo pacis meæ, in quo spe\MediatioDI{rá}[vi,] qui edébat panem meum, levávit contra me cal\TerminatioDI{cá}neum. \Responsorium{Quia}[\LinkLA]}%
    {Até meu amigo pessoal, no qual eu confi\MediatioDI{a}[va,] que comia à minha mesa, levantou o calcanhar contra \TerminatioDI{mim}. \Responsorium{Pois}[\LinkPT]}
}