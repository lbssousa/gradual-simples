% !TeX root = ../../a4.tex
% chktex-file 1

\subsection{Entrada}\label{subsection:missae-pro-varii-necessitatibus/pro-unitate-christianorum/introitus}
\MakeChantAntiphonPsalm{ego-sum-pastor-bonus.3a}[1]{psalmi-ad-introitum}

\subsection[Salmo Responsorial]{Salmo Responsorial \textmd{D 1 e}}\label{subsection:missae-pro-varii-necessitatibus/pro-unitate-christianorum/psalmus-responsorius}
\MakeChantPsalmTwoVerses{psalmi-responsorii}{laetatus-sum-in-eo.D1e}

\AllowPageFlush

\subsection[Salmo Aleluiático]{Salmo Aleluiático \textmd{C 4}}\label{subsection:missae-pro-varii-necessitatibus/pro-unitate-christianorum/psalmus-alleluiaticus}
\begin{rubrica}
    O primeiro {\normalfont\Rbar} pode ser cantado apenas pelo grupo de cantores ou por todos. O segundo {\normalfont\Rbar} é cantado por todos.
  \end{rubrica}
\MakeChantPsalmOneVerse{psalmi-alleluiatici}{in-convertendo-dominus.C4}

\AllowPageFlush

\subsection{Ofertório}\label{subsection:missae-pro-varii-necessitatibus/pro-unitate-christianorum/offertorium}
\MakeChantAntiphonPsalm{qui-te-exspectant.1f}[1]{psalmi-ad-offertorium}

\AllowPageFlush

\subsection{Comunhão}\label{subsection:missae-pro-varii-necessitatibus/pro-unitate-christianorum/communio-1}
\MakeChantAntiphonPsalm{illuminatio-mea.8c}[1]{psalmi-ad-communionem}