% !TeX root = ../../a4.tex
% chktex-file 1

\subsection{Entrada}\label{subsection:proprium-sanctorum/in-nativitate-sancti-ioannes-baptistae/introitus}
\def\AntiphonAnnotation{\CantusID{002400}[Is 49,1]}
\def\AntiphonScore{dominus-ab-utero.8G/}
\MakeChantAntiphonPsalm{introitus/}{\AntiphonScore}

\AllowPageFlush

\subsection[Salmo Responsorial]{Salmo Responsorial \textmd{E 5}}\label{subsection:proprium-sanctorum/in-nativitate-sancti-ioannes-baptistae/psalmus-responsorius}
\MakeChantPsalmThreeVerses{psalmi-responsorii/}{in-te-domine-speravi.E5/}

\subsection{Aleluia}\label{subsection:proprium-sanctorum/in-nativitate-sancti-ioannes-baptistae/alleluia}
\def\AntiphonScore{alleluia.1g2/}
\MakeChantAntiphonPsalm{alleluia/}{ad-te-domine.1g2/}

\AllowPageFlush

\subsection[Salmo Aleluiático]{Salmo Aleluiático \textmd{C 4}}\label{subsection:proprium-sanctorum/in-nativitate-sancti-ioannes-baptistae/psalmus-alleluiaticus}
\begin{rubrica}
  O primeiro {\normalfont\Rbar} pode ser cantado apenas pelo grupo de cantores ou por todos. O segundo {\normalfont\Rbar} é cantado por todos.
\end{rubrica}
\MakeChantPsalmOneVerse{psalmi-alleluiatici/}{ad-te-domine.C4/}

\AllowPageFlush

\subsection{Ofertório}\label{subsection:proprium-sanctorum/in-nativitate-sancti-ioannes-baptistae/offertorium}
\def\AntiphonAnnotation{\CantusID{004412}[Lc 7,26.28]}
\def\AntiphonScore{puer.7d/}
\MakeChantAntiphonPsalm{offertorium/}{\AntiphonScore}

\AllowPageFlush

\subsection{Comunhão}\label{subsection:proprium-sanctorum/in-nativitate-sancti-ioannes-baptistae/communio}
\def\AntiphonAnnotation{\CantusID{005218}[Lc 1,76]}
\def\AntiphonScore{tu-puer.3a/}
\MakeChantAntiphonPsalm{communio/}{\AntiphonScore}