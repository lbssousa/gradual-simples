\SetVerseAnnotation{Sl 45(46),2.3.4.5.6.8.9ab.9c--10.11.12}

\SetLinkTextL{\Antiphona{Beáta mater}}
\SetLinkTextR{\Antiphona{Bendita Mãe}}

\SetFirstVersePair{
  % 2
  {\Inchoatio{De}{us} est nobis refúgium et \MediatioII{vir}[tus] adiutórium in tribulatiónibus invéntus \TerminatioII{est}[ ]{ni}mis.}%
    {\Inchoatio{O}[ ]{Se}nhor para nós é refúgio e vi\MediatioII{gor}[,] sempre pronto, mostrou-se um socorro na \TerminatioII{an}{gús}tia.}
}

\SetVersePairs{
  % 3
  {\Inchoatio{Prop}{té}rea non timébimus, dum turbátur \MediatioII{ter}[ra,] et transferéntur montes in \TerminatioII{cor}[ ]{ma}ris.}%
{\Inchoatio{As}{sim} não tememos, se a terra estre\MediatioII{me}[ce,] se os montes desabam, caindo \TerminatioII{nos}[ ]{ma}res.},
  % 4
  {\Inchoatio{Fre}{mant} et intuméscant aquæ \MediatioII{ei}[us,] conturbéntur montes in elatió\TerminatioII{ne}[ ]{e}ius.}%
{\Inchoatio{Se}[ ]{as} águas trovejam e as ondas se a\MediatioII{gi}[tam,] se, em feroz tempestade, as montanhas se \TerminatioII{a}{ba}lam.},
  % 5
  {\Inchoatio{Flú}{mi}nis rivi lætíficant civitátem \MediatioII{De}[i,] sancta tabernácula \TerminatioII{Al}{tís}simi.}%
{\Inchoatio{Os}[ ]{bra}ços de um rio vêm trazer ale\MediatioII{gri}[a] à Cidade de Deus, à morada do \TerminatioII{Al}{tís}simo.},
  % 6
  {\Inchoatio{De}{us} in médio eius, non commo\MediatioII{vé}[bitur;] adiavábit eam Deus mane \TerminatioII{di}{lú}culo.}%
{\Inchoatio{Quem}[ ]{a} pode abalar? Deus está no seu \MediatioII{mei}[o!] Já bem antes da aurora, ele vem a\TerminatioII{ju}{dá}-la.},
  % 8
  {\Inchoatio{Dó}{mi}nus virtútum no\MediatioII{bís}[cum,] refúgium nobis De\-\TerminatioII{us}[ ]{Ia}cob.}%
{\Inchoatio{Co}{nos}co está o Senhor do uni\MediatioII{ver}[so!] O nosso refúgio é o Deus de \TerminatioII{Ja}có!},
% 9ab
  {\Inchoatio{Ve}{ní}te, et vidéte ópera \MediatioII{Dó}[mini] quæ pósuit prodígia su\TerminatioII{per}[ ]{ter}ram.}%
{\Inchoatio{Vin}{de} ver, contemplai os prodígios de \MediatioII{Deus} e a obra estupenda que fez no u\TerminatioII{ni}{ver}so.},
% 9c-10
  {\Inchoatio{Áu}{fe}ret bella usque ad finem \Flexa{ter}[ræ,] arcum cónteret et confrínget \MediatioII{ar}[ma,] et scuta combú\TerminatioII{ret}[ ]{i}gni.}%
{\Inchoatio{Re}{pri}me as guerras na face da \Flexa{ter}[ra,] ele quebra os arcos, as lanças des\MediatioII{trói}[,] e queima no fogo os escudos e \TerminatioII{as}[ ]{ar}mas.},
  % 11
  {\Inchoatio{Va}{cá}te, et vidéte quóniam ego sum \MediatioII{De}[us;] exaltábor in géntibus, exaltábor \TerminatioII{in}[ ]{ter}ra.}%
{\Inchoatio{Pa}{rai} e sabei, conhecei que eu sou \MediatioII{Deus}[,] que domino as nações, que domino \TerminatioII{a}[ ]{ter}ra!},
  % 12
  {\Inchoatio{Dó}{mi}nus virtútum no\MediatioII{bis}[cum,] refúgium nobis De\-\TerminatioII{us}[ ]{Ia}cob.}%
{\Inchoatio{Co}{nos}co está o Senhor do uni\MediatioII{ver}[so!] O nosso refúgio é o Deus de \TerminatioII{Ja}có!}
}