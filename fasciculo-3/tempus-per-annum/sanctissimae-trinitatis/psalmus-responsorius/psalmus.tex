% !TeX root = ../../../main.tex
\DeclareDocumentCommand{\Annotation}{}{
  \MakeAnnotation{}{Sl 148,1--3.5--6.12.3cd}
}

\def\LinkLA{tempus-per-annum/sanctissimae-trinitatis/psalmus-responsorius/psalmus-v1}
\def\LinkPT{\LinkLA-pt}

\def\VersePairs{
  %
  {Laudáte eum, sol et \MediatioC{lu}[na,] laudáte eum, omnes stellæ et \TerminatioCII{lu}men. \Responsorium{In excélsis}[\LinkLA]}%
    {Louvai-o, sol e \MediatioC{lu}[a,] Louvai-o, todas as estrelas relu\TerminatioCII{zen}tes. \Responsorium{Louvai a Deus}[\LinkPT]},
  %
  {Láudent nomen \MediatioC{Dó}[mini,] quia ipse mandávit, et cre\TerminatioCII{áta} sunt. \Responsorium{In excélsis}[\LinkLA]}%
    {Louvem o nome do Se\MediatioC{nhor}[,] pois ele mandou, e foram cri\TerminatioCII{a}dos. \Responsorium{Louvai a Deus}[\LinkPT]},
  %
  {Státuit ea in ætérnum, et in sǽculum \MediatioC{sǽ}[culi;] præcéptum pósuit et non præter\TerminatioCII{i}bit. \Responsorium{In excélsis}[\LinkLA]}%
    {Estabeleceu-os para sempre e pelos séculos dos \MediatioC{sé}[\-cu\-los;] fixou-lhes um decreto, que não passa\TerminatioCII{rá}. \Responsorium{Louvai a Deus}[\LinkPT]},
  %
  {Reges terræ et omnes \MediatioC{pó}[puli,] príncipes et omnes iúdices \TerminatioCII{ter}ræ. \Responsorium{In excélsis}[\LinkLA]}%
    {Reis da terra e todos os \MediatioC{po}[vos,] príncipes e todos os juízes da \TerminatioCII{ter}ra. \Responsorium{Louvai a Deus}[\LinkPT]},
  %
  {Iuvenes et \Flexa*{vír}[gines,] senes cum iunióribus laudent nomen \MediatioC{Dó}[mini,] quia exaltátum est nomen eius so\TerminatioCII{lí}us. \Responsorium{In excélsis}[\LinkLA]}%
    {Moços e \Flexa*{mo}[ças,] anciãos e cri\MediatioC{an}[ças,] porque seu nome é o único su\TerminatioCII{bli}me. \Responsorium{Louvai a Deus}[\LinkPT]},
  %
  {Conféssio eius super cælum et \MediatioC{ter}[ram,] et exaltávit cornu pópuli \TerminatioCII{su}i. \Responsorium{In excélsis}[\LinkLA]}%
    {A sua majestade está acima do céu e da \MediatioC{ter}[ra.] Ele reergueu a força do seu \TerminatioCII{po}vo. \Responsorium{Louvai a Deus}[\LinkPT]},
  %
  {Hymnus ómnibus sanctis \MediatioC{e}[ius,] fíliis Israel, pópulo qui propínquus est \TerminatioCII{e}i. \Responsorium{In excélsis}[\LinkLA]}%
    {É canto de louvor para todos os seus fi\MediatioC{éis} e para os filhos de Israel, povo que lhe está \TerminatioCII{próxi}mo. \Responsorium{Louvai a Deus}[\LinkPT]}
}