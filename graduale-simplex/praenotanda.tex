% chktex-file 1
% chktex-file 19
\newcommand{\GR}{\emph{Graduale Romanum}}
\newcommand{\GS}{\emph{Graduale Simplex}}
\newcommand{\KS}{\emph{Kyriale Simplex}}
\newcommand{\Al}{\emph{Allelúia}}
\newcommand{\Schola}{\emph{schola}}
\newcommand{\ScholaC}{\emph{schola cantorum}}
\newcommand{\FirstPara}[1]{\noindent\textcolor{gregoriocolor}{#1.}}
\newcommand{\Para}[1]{\vspace{\baselineskip}\noindent\textcolor{gregoriocolor}{#1.}}

\tocchapter{Introdução}

\tocsection{Sobre o projeto}

O {\GS} é o que temos hoje de mais acessível em matéria de canto gregoriano para uso no âmbito paroquial. A edição mais recente, publicada em 1975, atendendo a um pedido da Constituição \emph{Sacrosanctum Concilium}, é uma das mais fortes evidências de que o canto gregoriano não foi abolido pelo Concílio Vaticano II, mas, ao contrário, ratificado por ele como canto próprio para a sagrada liturgia.

Apesar disso, quase cinquenta anos após sua publicação, o {\GS} ainda é pouco acessível para a maioria de nossas paróquias. A fim de contornar os obstáculos que ainda dificultam a utilização deste repertório, trazemos esta proposta bilíngue, com a adaptação das melodias do {\GS} para língua portuguesa.

\tocsection{Sobre a organização e construção do projeto}

Este projeto dividiu o conteúdo do {\GS} em seis fascículos, a saber:
\begin{enumerate}[I:]
  \item Tempo do Advento e Tempo do Natal
  \item Tempo da Quaresma, Tríduo Pascal e Tempo paschal
  \item Tempo Comum e Liturgia dos Defuntos
  \item Próprio dos Santos e Comuns
  \item Missas Rituais, por Diversas Necessidades e Votivas
  \item Cantos para o Ordinário da Missa e outros cantos diversos
\end{enumerate}

As adaptações das antífonas utilizam uma tradução livre, com a escolha de textos que mais bem se encaixem à melodia original sem comprometer demais a fidelidade da tradução. Também procurou-se reproduzir na melodia adaptada, na medida do possível, os mesmos elementos retóricos presentes nos neumas adiastemáticos para a melodia original em latim. Para tanto, foram aplicadas, quando necessário, diéreses\footnote{Quebra dos neumas originais em neumas menores para acomodá-los em um número maior de sílabas no texto adaptado.} e sinéreses\footnote{Aglutinação dos neumas originais em neumas maiores, para acomodá-los em um número menor de sílabas no texto adaptado.} nos neumas originais, observadas as seguintes premissas:
\begin{itemize}
  \item Neumas de duas ou mais notas que apresentam algum tipo de alargamento não sofrerão diérese na melodia adaptada.
  \item A sinérese de dois ou mais neumas na melodia adaptada não resultará em um neuma inexistente na notação adiastemática (como aqueles em que a penúltima nota é alargada, mas a última nota não o é.). A única exceção a esta regra é a sinérese dos neumas da cadência final do tom salmódico D \GreStar, resultando em um \emph{climacus} isolado de quatro notas\footnote{Semiologicamente, o limite para o \emph{climacus} isolado, isto é, que não está em composição com outros neumas em um melisma, é de três notas.} sobre a última sílaba do verso salmódico.
\end{itemize}

Para favorecer a prosódia da língua portuguesa, foram empregadas elisões\footnote{Aglutinação de duas ou mais vogais consecutivas de palavras distintas, como se pertencecessem a uma única sílaba.} nos textos fixos, como antífonas e hinos. Por outro lado, não foram empregadas elisões nos versos salmódicos\footnote{Há uma versão alternativa deste fascículo, em notação musical moderna, em que as elisões são aplicadas também aos versos salmódicos.}, delegando-as à livre interpretação do cantor.

Por uma questão de completeza, foram incluídas as adaptações não só das antífonas de entrada, ofertório e comunhão, mas também dos salmos responsoriais e aleluiáticos, bem como do Aleluia com seus versículos, tratos, hinos e sequências. Para os versos salmódicos, hinos e sequências em vernáculo, foram utilizados os textos litúrgicos oficiais presentes na Liturgia das Horas, no Lecionário e no Missal Romano (CNBB), exceto quando indicada explicitamente outra fonte.

Devido à estrutura peculiar dos versos salmódicos na Liturgia das Horas, por vezes um único versículo está dividido em dois ou mais versos salmódicos, resultando em uma quebra de paridade com os versos salmódicos originais em latim. É facultado, portanto, para a salmodia em vernáculo, cantar dois ou mais versos salmódicos em sequência, conforme a conveniência, antes de retornar à antífona. Neste caso, não se canta a entonação inicial nos versos salmódicos subsequentes, que devem ser iniciados já na corda de recitação.

\subsection{Lista de acrônimos presentes nas adaptações para o vernáculo}

As adaptações para o vernáculo presentes nesta edição podem trazer um ou mais dos acrônimos abaixo, que indicam quem fez a adaptação ou a fonte do texto em língua portuguesa utilizado (no caso de hinos ou sequências):

\begin{center}
  \begin{tabular}{rl}
    AG      & André Gaby                                                    \\
    LHH     & Lincoln Haas Hein                                             \\
    LS      & Laércio de Sousa                                              \\
    RV      & Reginaldo Veloso                                              \\
    WL      & Willian Lovato                                                \\
    HL-CNBB & Hinário Litúrgico (CNBB)                                      \\
    SL-LC   & Saltério Litúrgico (Editora Lumen Christi)                    \\
    BS-CNBB & Bíblia Sagrada --- Tradução Oficial da CNBB (6ª Edição, 2022) \\
    LH-CNBB & Liturgia das Horas (CNBB, 2ª Edição Típica, 2004)             \\
    MR-CNBB & Missal Romano (CNBB, 3ª Edição Típica, 2023)                  \\
  \end{tabular}
\end{center}

\AllowPageFlush

\tocsection{Instrução Geral}\label{section:praenotanda}

\begin{center}
  \begin{rubrica}
    Texto adaptado de duas traduções para língua portuguesa, gentilmente cedidas pelo Prof.\ Dr.\ Clayton Dias e por Luís Augusto Rodrigues Domingues.
  \end{rubrica}
\end{center}

\subsection{I.\@ Da natureza da presente edição}\label{subsection:praenotanda-1}

\FirstPara{1} Desejando desenvolver o canto sagrado e a ativa participação dos fiéis nas sagradas celebrações cantadas, o Sacrossanto Concílio Ecumênico Vaticano II, na Constituição sobre a Sagrada Liturgia, dispôs que, além de se dever terminar a edição típica das melodias gregorianas, fosse preparada ``uma edição contendo melodias mais simples, para uso das igrejas menores'' (n. 117). Cumprindo com o desejo dos Padres do Concílio, alguns especialistas prepararam esta edição para os cantos estabelecidos no  Ordinário e no Próprio da Missa, destinando-a àquelas igrejas que dificilmente podem executar de modo digno as melodias mais ornadas do {\GR}.

\Para{2} O {\GR} seja mantido em grandíssima honra pela Igreja, pelos admiráveis aspectos de piedade e de arte, dos quais é abundantemente dotado, e seja conservado integralmente o seu valor. Por esta razão é desejável que, segundo a nova estruturação dele mesmo a partir do \emph{Ordo Cantus Missæ} (\emph{Typis Polyglottis Vaticanis} 1972), seja oportunamente utilizado naquelas igrejas que são dotadas de uma {\ScholaC} preparada com a necessária formação técnica para poder executar dignamente as melodias mais ornadas.

Na verdade, recomenda-se que também nas igrejas menores, que usam o {\GS}, sejam  adotadas algumas partes tomadas do {\GR}, particularmente aquelas mais fáceis ou aquelas que são mais tradicionalmente usadas entre as pessoas.

\Para{3} Por este motivo, não é obrigatório que o tesouro destes dois livros seja usado separadamente; na verdade, uma certa fusão das melodias extraídas de uma e da outra fonte pode oferecer uma maior e feliz variedade.

\Para{4} Portanto, se for feita uma escolha sábia, o {\GS} não empobrece o tesouro musical das melodias gregorianas, mas, ao contrário, o enriquece. Em primeiro lugar, no que diz respeito à escolha dos textos, porque ele introduz outros textos no uso litúrgico, que até agora não estão contidos no Missal Romano; em segundo lugar, também no que se refere às melodias, no momento em que no {\GS} foram acolhidas e propostas numerosas outras, extraídas do tesouro autêntico das fontes gregorianas; e, em terceiro lugar, também sob o aspecto pastoral, porque oferece a possibilidade de desenvolver celebrações com canto também nos grupos menores.

\subsection{II.\@ Dos critérios adotados}\label{subsection:praenotanda-2}

\FirstPara{5} Para que a celebração eucarística possa ser realizada em todos os lugares de uma forma mais nobre, isto é, com canto, e nela se possa obter a participação dos fiéis, era absolutamente necessário que fossem encontradas melodias simples.

\Para{6} Mas tais melodias simples não poderiam ser extraídas das melodias mais ornadas, contidas no {\GR}, visto que, de modo algum, era lícito retirar notas ou melismas destas últimas; e muito menos parecia apropriado para tal trabalho criar integralmente melodias neogregorianas para os textos do Missal Romano.

\Para{7} Assim, foram extraídas do tesouro do canto gregoriano melodias autênticas, correspondentes ao objetivo, seja das edições típicas já em circulação, seja das fontes manuscritas, tanto do rito romano como dos outros ritos latinos.

\Para{8} Mas desta nova seleção de melodias nasceu também uma nova série de textos: de fato, rarissimamente foi encontrada uma melodia simples unida às mesmíssimas palavras reportadas no Missal. Por isso, toda vez que não foi possível ter uma concordância, foram escolhidas partes que oferecem palavras semelhantes àquelas do Missal Romano, ou pelo menos próximas ao seu significado. Muitas vezes, todavia, o texto da antífona, não podendo ser proposto como antífona, aparece como versículo do próprio salmo a ser cantado após a antífona.

\Para{9} Consequentemente, visto que estes novos textos foram escolhidos somente com base em motivações musicais, de modo algum será lícito usá-los sem notas musicais.

\subsection{III.\@ Dos cantos do {\KS}}\label{subsection:praenotanda-3}

\FirstPara{10} As partes são dispostas de forma a constituir cinco esquemas simples do Ordinário da Missa, mas que não possuem por si mesmas nenhuma relação com o grau de festividade dos dias litúrgicos. O primeiro destes esquemas responde de modo particular ao desejo do Concílio: ``Tomem-se providências para que os fiéis possam rezar ou cantar juntos, mesmo em latim, as partes do Ordinário que lhes competem.'' (SC n.\ 54). Todavia, as partes individuais são designadas com números progressivos, com o objetivo de facilitar, se desejável, a composição do Ordinário da Missa com partes extraídas dos diversos esquemas.

\Para{11} Ao final do hino \textcolor{gregoriocolor}{\emph{Glória in excélsis}}, retomado do rito ambrosiano, foi acrescentado um \textcolor{gregoriocolor}{\emph{Amen}} simples, mas este não está em desacordo com a genuína tradição mais antiga.

\Para{12} Nada impede que o povo não cante todo o \textcolor{gregoriocolor}{\emph{Agnus Dei}}, mas somente responda com as palavras \textcolor{gregoriocolor}{\emph{miserére nobis}} e \textcolor{gregoriocolor}{\emph{dona nobis pacem}}.

\subsection{IV.\@ Da forma dos cantos da Missa}

\FirstPara{13} Para os cantos de entrada, de ofertório e de comunhão foi utilizada a forma apropriada que consiste em uma antífona a ser repetida após o versículo de um salmo.

\Para{14} Para os cantos que ocorrem entre as leituras, segundos os diversos tempos do ano, encontram-se:
\begin{enumerate}[a)]
  \item o salmo responsorial, com o responsório salmódico ou aleluiático;

  \item o salmo sem responsório, que tradicionalmente é chamado de \emph{tractus};

  \item o {\Al} com alguns versículos do salmo no tempo em que se canta o \textcolor{gregoriocolor}{\Al}, ou uma outra aclamação não aleluiática ao Evangelho no Tempo da Quaresma.
\end{enumerate}

\subsection{V.\@ Da estrutura dos esquemas das Missas}\label{subsection:praenotanda-5}

\FirstPara{15} No Próprio do Tempo, com exceção da Quaresma, não há cantos próprios para cada domingo, mas para cada tempo litúrgico são oferecidos um ou vários esquemas com a faculdade de adotá-los nos domingos de qualquer tempo.

Todavia, cada celebração do Senhor possui cantos próprios.

\Para{16} No Próprio dos Santos, encontram-se os cantos da Missa próprios para as celebrações que tem precedência sobre os domingos.

\Para{17} Os Comuns dos Santos são organizados no mesmo modo dos Comuns do Missal Romano; todavia, normalmente para cada uma das ordens dos santos é oferecido somente um esquema, mas com vários cantos para as diversas partes da Missa de modo que se possa escolher um ou outro que melhor se adapte ao Santo.

\subsection{VI.\@ Das pessoas requeridas para a execução dos cantos do {\GS}}

\FirstPara{18} Tendo presente o princípio proposto na Constituição sobre a Liturgia, que ``nas celebrações litúrgicas, seja quem for, ministro ou fiel, exercendo o seu ofício, faça tudo e só aquilo que pela natureza da coisa ou pelas normas litúrgicas lhe compete'' (SC n.\ 28), pela estrutura do {\GS}:
\begin{enumerate}[a)]
  \item o cantor entoa as antífonas e propõe os versículos dos salmos, ao que o povo responde. O salmo pode ser cantado também pela {\Schola}.

  \item a assembleia de fiéis deve cantar as antífonas e os responsórios dos salmos entre as leituras. A parte dos fiéis às vezes pode ser confiada também à {\Schola}; contudo, convém que ao menos os responsórios dos salmos entre as leituras, dado o seu caráter e a facilidade com a qual podem ser cantados, sejam executados por toda a assembleia.
\end{enumerate}

\subsection{VII.\@ Do uso do {\GS}}
\FirstPara{19} Para a entrada, o ofertório e a comunhão, canta-se a antífona com um ou vários versículos do salmo, conforme a ocasião.

A antífona é repetida depois do versículo do salmo; porém, os versículos podem ser escolhidos livremente dentre aqueles que são propostos, não se omitindo aqueles que são necessários para que a frase tenha um sentido completo. Para a entrada e a comunhão, como conclusão, pode ser cantado o \textcolor{gregoriocolor}{\emph{Gloria Patri}} e o \textcolor{gregoriocolor}{\emph{Sicut erat}}, combinados num único versículo, como é indicado nos tons comuns\footnote{Em nosso projeto, os tons comuns para o \textcolor{gregoriocolor}{\emph{Gloria Patri}} estão disponíveis no fascículo 6.}.

Ao cantar o salmo, sejam respeitados dois elementos particulares do tom salmódico: o início, onde o final da antífona se conecta com a corda de récita do salmo; e a terminação, onde o final do tom salmódico se conecta com o início da antífona.

\Para{20} Os cantos entre as leituras sejam ordenados deste modo:

Quando são realizadas duas leituras antes do Evangelho, os cantos são ordenados assim:
\begin{enumerate}[I)]
  \item Fora do Tempo da Quaresma e do Tempo Pascal, depois da primeira leitura canta-se o salmo responsorial; depois da segunda leitura: ou o salmo aleluiático ou a antífona \textcolor{gregoriocolor}{\Al} com os seus versículos.

  \item Na Quaresma, depois da primeira leitura: o primeiro salmo responsorial; depois da segunda: ou o segundo salmo responsorial, ou a antífona de aclamação ou o \emph{tractus}.

  \item No Tempo Pascal, depois da primeira leitura: o primeiro ou o segundo salmo aleluiático; depois da segunda leitura: ou o segundo salmo aleluiático, ou a antífona \textcolor{gregoriocolor}{\Al} com os seus versículos.
\end{enumerate}

Toda vez que é feita somente uma leitura antes do Evangelho, realiza-se somente um canto à escolha entre os cantos apropriados.

Do salmo, cantam-se ao menos cinco versículos, tomados à escolha, quando são propostos vários.

\Para{21} Quando para o mesmíssimo tempo são oferecidos vários esquemas, pode-se escolher à vontade um ou outro, segundo aquilo que parece mais oportuno. Na verdade, pode-se também tomar partes de um e partes de outros.

Para a comunhão pode-se cantar sempre o Salmo 33 \textcolor{gregoriocolor}{\emph{Benedícam Dóminum}}, com o \Rbar. \textcolor{gregoriocolor}{\Al} ou \textcolor{gregoriocolor}{\emph{Gustáte}}. Na verdade, pode-se também cantar à vontade outros cantos adequados, como vem indicado no final do livro, p. \textcolor{gregoriocolor}{462}\footnote{Em nosso projeto, os cânticos de comunhão à escolha estão presentes no fascículo 6.}.