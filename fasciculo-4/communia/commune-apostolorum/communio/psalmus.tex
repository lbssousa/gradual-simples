% !TeX root = ../../../main.tex
\DeclareDocumentCommand{\Annotation}{}{
  \MakeAnnotation{\CantusID{005502}[Mt 19,28]}{Sl 139(138),1--6.17--18.23--24}
}

\def\LinkLA{communia/commune-apostolorum/communio/antiphona}
\def\LinkPT{\LinkLA-pt}

\def\VersePairs{
  % 3
  {\Inchoatio{In}{tel}lexísti cogitatiónes \MediatioI{me}{as de }{lon}[ge,] sémitam me\-am et accúbitum meum in\TerminatioI{ve}{sti}{\-gá}\-sti. \Antiphona{Vos qui secúti estis me}[\LinkLA]}%
    {\Inchoatio{De}[ ]{lon}ge penetrais meus \MediatioI{pen}{\-sa}{\-men}[\-tos,] investigais meu caminho e \TerminatioI{meu}[ ]{re}{pou}so. \Antiphona{Vós, que seguistes a mim}[\LinkPT]},
  % 4
  {\Inchoatio{Et}[ ]{om}nes vias meas perspe\Flexa{xís}[ti,] quia nondum est sermo in \MediatioI{lin}{gua }{me}[a,] et ecce, Dómine, tu no\TerminatioI{ví}{sti}[ ]{óm}nia. \Antiphona{Vos qui secúti estis me}[\LinkLA]}%
    {\Inchoatio{To}{das} as minhas veredas vos são conhe\Flexa{ci}[das.] A palavra não está ainda na \MediatioI{mi}{nha }{lín}[gua,] e eis, Senhor, já \TerminatioI{sa}{beis}[ ]{tu}do. \Antiphona{Vós, que seguistes a mim}[\LinkPT]},
  % 5
  {\Inchoatio{A}[ ]{ter}go et a fronte \MediatioI{co}{ar}{tás}[ti me,] et posuísti super me \TerminatioI{ma}{num}[ ]{tu}am. \Antiphona{Vos qui secúti estis me}[\LinkLA]}%
    {\Inchoatio{Por}[ ]{trás} e pela frente me \MediatioI{en}{vol}{veis}[,] e pondes sobre mim a \TerminatioI{Vos qui secúti estis me}{sa}[ ]{mão}. \Antiphona{Vós, que seguistes a mim}[\LinkPT]},
  % 6
  {\Inchoatio{Mi}{rá}bilis nimis facta est sciéntia \MediatioI{tu}{a }{su}[per me;] sublímis, et non at\TerminatioI{tín}{gam}[ ]{e}am. \Antiphona{Vos qui secúti estis me}[\LinkLA]}%
    {\Inchoatio{Mui}{to} admirável é vosso conhecimento, que me \MediatioI{ul}{\-tra}{\-pas}[\-sa,] sublime, e não posso \TerminatioI{a}{tin}{gi}-lo. \Antiphona{Vós, que seguistes a mim}[\LinkPT]},
  % 17
  {\Inchoatio{Mi}{hi} autem nimis pretiosre cogitatiónes \MediatioI{tu}{æ, }{De}[us;] nimis gravis sum\TerminatioI{ma}[ ]{e}{á}rum. \Antiphona{Vos qui secúti estis me}[\LinkLA]}%
    {\Inchoatio{Quão}[ ]{pre}ciosos, para mim, ó Deus, vossos \MediatioI{pen}{sa}{\-men}[\-tos;] imenso é \TerminatioI{o}[ ]{seu}[ ]{nú}mero! \Antiphona{Vós, que seguistes a mim}[\LinkPT]},
  % 18
  {\Inchoatio{Si}[ ]{di}numerábo eas, super arénam multi\-\MediatioI{pli}{ca}{\-bún}[\-tur;] si ad finem pervéniam, ad\TerminatioI{huc}[ ]{sum}[ ]{te}cum. \Antiphona{Vos qui secúti estis me}[\LinkLA]}%
    {\Inchoatio{Se}[ ]{me} ponho a contá-los, são mais numerosos que a a\MediatioI{rei}{a do }{mar}[;] se chegar ao fim, ainda es\TerminatioI{tou}[ ]{con}{\-vos}\-co. \Antiphona{Vós, que seguistes a mim}[\LinkPT]},
  % 23
  {\Inchoatio{Scru}{tá}re me Deus, et \MediatioI{sci}{to cor }{me}[um;] proba me, et cognósce sé\TerminatioI{mi}{tas}[ ]{me}as. \Antiphona{Vos qui secúti estis me}[\LinkLA]}%
    {\Inchoatio{Son}{dai}-me, ó Deus, e conhecei o meu \MediatioI{co}{ra}{ção}[;] pon\-de-me à prova, e conhecei as mi\TerminatioI{nhas}[ ]{ve}{re}das. \Antiphona{Vós, que seguistes a mim}[\LinkPT]},
  % 24
  {\Inchoatio{Et}[ ]{vi}de, si via vani\MediatioI{tá}{tis in }{me}[ est,] et deduc me in vi\TerminatioI{a}[ ]{æ}{tér}na. \Antiphona{Vos qui secúti estis me}[\LinkLA]}%
    {\Inchoatio{Ve}{de} se meu ca\MediatioI{mi}{nho é }{vão}[,] e conduzi-me pelo cami\TerminatioI{nho}[ ]{e}{ter}no. \Antiphona{Vós, que seguistes a mim}[\LinkPT]}
}