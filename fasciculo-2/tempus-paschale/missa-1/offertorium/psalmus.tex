% !TeX root = ../../../a4.tex
\DeclareDocumentCommand{\Annotation}{}{
  \SetVerseAnnotation{\CantusID{001254}[Sl 63(62),2]}{Sl 63(62),2--5}
}

\def\LinkLA{tempus-paschale/missa-1/offertorium/antiphona}
\def\LinkPT{\LinkLA-pt}

\def\VersePairs{
  % 2d--3
  {\Inchoatio{In}[ ]{ter}ra desérta et árida et ina\Flexa{quó}[sa,] sic in sancto ap\MediatioVII{pá}{rui }{ti}[bi,] ut vidérem virtútem tuam et \TerminatioVII{gló}{riam }{tu}am. \Antiphona{Ad te de luce}[\LinkLA]}%
    {\Inchoatio{Nu}{ma} terra deserta, seca, sem \Flexa{á}[gua.] Assim me apresentei no \MediatioVII{san}{tu}{á}[rio] para contemplar o vosso poder e a \TerminatioVII{vos}{sa }{gló}ria. \Antiphona{Desd\Elisio{e} a aurora}[\LinkPT]},
  % 4
  {\Inchoatio{Quó}{ni}am mélior est misericórdia tua \MediatioVII{su}{per }{vi}[tas,] lábia \TerminatioVII{me}{a lau}{dá}bunt te. \Antiphona{Ad te de luce}[\LinkLA]}%
    {\Inchoatio{Pois}[ ]{vos}sa misericórdia vale mais do \MediatioVII{que}{ a }{vi}[da] e vos louva\TerminatioVII{rão}{ meus }{lá}bios. \Antiphona{Desd\Elisio{e} a aurora}[\LinkPT]},
  % 5
  {\Inchoatio{Sic}[ ]{be}nedícam te in \MediatioVII{vi}{ta }{me}[a] et in nómine tuo levábo \TerminatioVII{ma}{nus }{me}as. \Antiphona{Ad te de luce}[\LinkLA]}%
    {\Inchoatio{As}{sim} vos bendirei en\MediatioVII{quan}{to vi}{ver} e erguerei minhas mãos no \TerminatioVII{vos}{so }{no}me. \Antiphona{Desd\Elisio{e} a aurora}[\LinkPT]}
}





