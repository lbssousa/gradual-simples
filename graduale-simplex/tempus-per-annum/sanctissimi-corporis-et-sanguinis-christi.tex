% !TeX root = ../fasciculo-3/a4.tex
% chktex-file 1

\subsection{Entrada}\label{subsection:tempus-per-annum/sanctissimi-corporis-et-sanguinis-christi/psalmi-ad-introitum}
\MakeChantAntiphonPsalm{sacerdos-in-aeternum.1f}{psalmi-ad-introitum}

\AllowPageFlush

\subsection[Salmo Responsorial]{Salmo Responsorial \textmd{E 5}}\label{subsection:tempus-per-annum/sanctissimi-corporis-et-sanguinis-christi/psalmus-responsorius}
\MakeChantPsalmTwoVerses{psalmi-responsorii}{exaltabo-te.E5}

\subsection{Sequência}\label{subsection:tempus-per-annum/sanctissimi-corporis-et-sanguinis-christi/sequentia}
\begin{rubrica}
  A forma breve está entre colchetes.
\end{rubrica}
\MakeChantLongPsalm{sequentiae}{lauda-sion}{
  {psalmus-v1}{psalmus-v1-pt}
}

\subsection{Aleluia}\label{subsection:tempus-per-annum/sanctissimi-corporis-et-sanguinis-christi/alleluia}
\MakeChantAlleluiaPsalm{3g}{attendite-popule-meus.3g}

\AllowPageFlush

\subsection[Salmo Aleluiático]{Salmo Aleluiático \textmd{C 4}}\label{subsection:tempus-per-annum/sanctissimi-corporis-et-sanguinis-christi/psalmus-alleluiaticus}
\begin{rubrica}
  O primeiro {\normalfont\Rbar} pode ser cantado apenas pelo grupo de cantores ou por todos. O segundo {\normalfont\Rbar} é cantado por todos.
\end{rubrica}
\MakeChantPsalmOneVerse{psalmi-alleluiatici}{attendite-popule-meus.C4}

\AllowPageFlush

\subsection{Ofertório}\label{subsection:tempus-per-annum/sanctissimi-corporis-et-sanguinis-christi/psalmi-ad-offertorium}
\MakeChantAntiphonPsalm{angelorum-esca.2D}{psalmi-ad-offertorium}

\AllowPageFlush

\subsection{Comunhão}\label{subsection:tempus-per-annum/sanctissimi-corporis-et-sanguinis-christi/psalmi-ad-communionem}
\MakeChantAntiphonPsalm{ego-sum-panis-vivus.1f}{psalmi-ad-communionem}

\subsection{Procissão}\label{subsection:tempus-per-annum/sanctissimi-corporis-et-sanguinis-christi/ad-processionem}
\begin{rubrica}
  Cantam-se os versos 1 a 4 abaixo, repetidamente ou seguidos de outros hinos eucarísticos, durante a procissão, se necessário. Após a chegada ao altar, cantam-se os versos 5 e 6 do hino.
\end{rubrica}
\MakeChantLongPsalm{hymni}{pange-lingua}{
  {psalmus-v1}{psalmus-v1-pt},
  {psalmus-v2}{psalmus-v2-pt},
  {psalmus-v3}{psalmus-v3-pt},
  {psalmus-v4}{psalmus-v4-pt},
  {psalmus-v5}{psalmus-v5-pt},
  {psalmus-v6}{psalmus-v6-pt}
}