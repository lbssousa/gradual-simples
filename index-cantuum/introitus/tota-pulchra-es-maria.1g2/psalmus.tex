\SetVerseAnnotation{Is 61,10--62,3}

\def\VersePairs{
  % 3
  {\Inchoatio{Qui}{a} índuit me vesti\MediatioI{mén}{tis sa}{lú}[tis,] et induménto iustítiæ cir\TerminatioI{cúm}{de}{dit} me. \Antiphona{Tota pulchra es María}[\LinkLA]}%
    {\Inchoatio{Por}{que} me revestiu com vestes de \MediatioI{sal}{va}{ção}[,] e me cobriu com o manto \TerminatioI{da}[ ]{jus}{ti}ça. \Antiphona{Toda bela sois, Maria}[\LinkPT]},
  % 4
  {\Inchoatio{Qua}{si} sponsum deco\MediatioI{rá}{tum co}{ró}[na,] et quasi sponsam ornátam moní\TerminatioI{li}{bus}[ ]{su}is. \Antiphona{Tota pulchra es María}[\LinkLA]}%
    {\Inchoatio{Co}{mo} o noivo que se adorna com \MediatioI{a}{ co}{ro}[a,] como a noiva que se enfeita com \TerminatioI{su}{as}[ ]{joi}as. \Antiphona{Toda bela sois, Maria}[\LinkPT]},
  % 5
  {\Inchoatio{Sic}{ut} enim terra profert germen \Flexa{su}[um,] et sicut hortus semen \MediatioI{su}{um }{gér}[minat,] sic Dóminus Deus germinábit iustítiam et laudem coram uni\TerminatioI{vér}{sis}[ ]{gén}ti\-bus. \Antiphona{Tota pulchra es María}[\LinkLA]}%
    {\Inchoatio{Tal}[ ]{co}mo a terra faz surgir os seus re\Flexa{no}[vos]  e o jardim faz germinar \MediatioI{su}{as se}{men}[tes,] assim o Senhor Deus fará brotar a justiça e o seu louvor, por todas \TerminatioI{as}[ ]{na}{ções}. \Antiphona{Toda bela sois, Maria}[\LinkPT]},
  % 6
  {\Inchoatio{Prop}{ter} Sion \MediatioI{non}{ ta}{cé}[bo] et propter Ierúsalem \TerminatioI{non}[ ]{qui}{és}\-cam. \Antiphona{Tota pulchra es María}[\LinkLA]}%
    {\Inchoatio{Por}[ ]{a}mor de Sião não me \MediatioI{ca}{la}{rei} e por amor de Jerusalém não des\TerminatioI{can}{sa}{rei}. \Antiphona{Toda bela sois, Maria}[\LinkPT]},
  % 8
  {\Inchoatio{Do}{nec} egrediátur ut splendor \MediatioI{iu}{stus }{e}[ius,] et salvátor cius ut lampas \Inchoatio{ac}{cen}{dá}tur. \Antiphona{Tota pulchra es María}[\LinkLA]}%
    {\Inchoatio{En}{quan}to não raiar como um clarão \MediatioI{su}{a jus}{ti}[ça,] e a sua salvação não brilhar como \TerminatioI{u}{ma}[ ]{to}cha. \Antiphona{Toda bela sois, Maria}[\LinkPT]},
  % 9
  {\Inchoatio{Et}[ ]{vi}débunt gentes \MediatioI{iu}{stum }{tu}[um,] et cuncti ín\TerminatioI{cli}{tum}[ ]{tu}um. \Antiphona{Tota pulchra es María}[\LinkLA]}%
    {\Inchoatio{As}[ ]{na}ções hão de ver \MediatioI{tu}{a jus}{ti}[ça,] e todos os reis, a \TerminatioI{tu}{a}[ ]{gló}ria. \Antiphona{Toda bela sois, Maria}[\LinkPT]},
  % 10
  {\Inchoatio{Et}[ ]{vo}cábitur tibi \MediatioI{no}{men }{no}[vum,] quod os Dómini reges \TerminatioI{no}{mi}{ná}bit. \Antiphona{Tota pulchra es María}[\LinkLA]}%
    {\Inchoatio{E}[ ]{te}rás um \MediatioI{no}{me }{no}[vo,] que a boca do Senhor pronun\TerminatioI{ci}{\-a}{\-rá}. \Antiphona{Toda bela sois, Maria}[\LinkPT]},
  % 11
  {\Inchoatio{Et}[ ]{e}ris coróna glóriæ in \MediatioI{ma}{nu }{Dó}[mini] et diadéma regni in manu \TerminatioI{De}{i}[ ]{tu}i. \Antiphona{Tota pulchra es María}[\LinkLA]}%
    {\Inchoatio{A}{ca}bará com as guerras até os confins do \Flexa{mun}[do;] quebrará o arco e esmaga\MediatioI{rá}{ as }{ar}[mas,] e queimará os escu\TerminatioI{dos}[ ]{ao}[ ]{fo}go. \Antiphona{Toda bela sois, Maria}[\LinkPT]},
  % 11
  {\Inchoatio{Va}{cá}te et vidéte quóniam \MediatioI{e}{go sum }{De}[us:] exaltábor in géntibus et exaltá\TerminatioI{bor}[ ]{in}[ ]{ter}ra. \Antiphona{Tota pulchra es María}[\LinkLA]}%
    {\Inchoatio{Se}{rás} uma coroa gloriosa nas mãos \MediatioI{do}{ Se}{nhor}[,] um diadema de rei na palma \TerminatioI{do}[ ]{teu}[ ]{Deus}. \Antiphona{Toda bela sois, Maria}[\LinkPT]}
}