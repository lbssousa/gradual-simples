% !TeX root = ../../../main.tex
\DeclareDocumentCommand{\Annotation}{}{
  \MakeAnnotation{\CantusID{005003}[Jo 15,26]}{Sl 78(77),1.4b.5b--7a.20.23--25.29}
}

\def\LinkLA{tempus-paschale/dominica-pentecostes/communio/antiphona}
\def\LinkPT{\LinkLA-pt}

\def\VersePairs{
  % 4b
  {\Inchoatio{Ge}{ne}ratióni álteri narrántes laudes Dómini et virtútes \MediatioVIII{e}[ius,] et mirabília e\TerminatioVIII{ius}[ ]{quæ}[ ]{fe}cit. \Antiphona{Spíritus}[\LinkLA]}%
    {\Inchoatio{Va}{mos} contar à geração futura os louvores do Senhor e seus pro\MediatioVIII{dí}[gios,] as maravilhas que ele re\TerminatioVIII{a}{li}{zou}. \Antiphona{\Elisio{O} Espírito}[\LinkPT]},
  % 5b
  {\Inchoatio{Quan}{ta} mandáverat pátribus \Flexa{no}[stris] nota fácere ea fíliis \MediatioVIII{su}[is,] ut cognśscat generátio áltera, fílii \TerminatioVIII{qui}[ ]{na}{scén}tur. \Antiphona{Spíritus}[\LinkLA]}%
    {\Inchoatio{Man}{dou} a nossos \Flexa{pais}[,] que transmitissem a lei a seus \MediatioVIII{fi}[lhos,] a fim de que a geração futura fique sabendo, os filhos que \TerminatioVIII{vão}[ ]{nas}{cer}. \Antiphona{\Elisio{O} Espírito}[\LinkPT]},
  % 6a
  {\Inchoatio{Ex}{súr}gent et narrábunt fíliis \MediatioVIII{su}[is,] ut ponant in De\TerminatioVIII{o}[ ]{spem}[ ]{su}am. \Antiphona{Spíritus}[\LinkLA]}%
    {\Inchoatio{E}{les} se levantarão e a transmitirão a seus \MediatioVIII{fi}[lhos,] a fim de que ponham em Deus sua \TerminatioVIII{es}{pe}{ran}ça. \Antiphona{\Elisio{O} Espírito}[\LinkPT]},
  % 6b
  {\Inchoatio{Et}[ ]{non} obliviscántur óperum \MediatioVIII{De}[i,] et mandáta e\TerminatioVIII{ius}[ ]{cu}{stó}diant. \Antiphona{Spíritus}[\LinkLA]}%
    {\Inchoatio{Das}[ ]{o}bras de Deus não se es\MediatioVIII{que}[çam] e guardem seus \TerminatioVIII{man}{\-da}{\-men}\-tos. \Antiphona{\Elisio{O} Espírito}[\LinkPT]},
  % 7a
  {\Inchoatio{Ec}{ce} percússit petram et fluxárunt \MediatioVIII{a}[quæ,] et torréntes i\TerminatioVIII{nun}{da}{vé}runt. \Antiphona{Spíritus}[\LinkLA]}%
    {\Inchoatio{Eis}[ ]{que} feriu a rocha e escorreu \MediatioVIII{á}[gua,] e as torrentes \TerminatioVIII{trans}{bor}{da}ram. \Antiphona{\Elisio{O} Espírito}[\LinkPT]},
  % 20
  {\Inchoatio{Num}{quid} et panem póterit \MediatioVIII{da}[re,] aut paráre carnes pó\TerminatioVIII{pu}{lo}[ ]{su}o? \Antiphona{Spíritus}[\LinkLA]}%
    {``\Inchoatio{Po}{de}rá ele dar-nos pão tam\MediatioVIII{bém}[,] ou preparar carne para \TerminatioVIII{o}[ ]{seu}[ ]{po}vo?'' \Antiphona{\Elisio{O} Espírito}[\LinkPT]},
  % 23
  {\Inchoatio{Ve}{rúm}tamen mandávit núbibus \MediatioVIII{dé}[super,] et iánuas cæ\TerminatioVIII{li}[ ]{a}{pé}ruit. \Antiphona{Spíritus}[\LinkLA]}%
    {\Inchoatio{Deu}[ ]{or}dens às nuvens do \MediatioVIII{al}[to] e abriu as por\TerminatioVIII{tas}[ ]{do}[ ]{céu}. \Antiphona{\Elisio{O} Espírito}[\LinkPT]},
  % 24
  {\Inchoatio{Et}[ ]{plu}it illis manna ad mandu\MediatioVIII{cán}[dum,] et panem cæli \TerminatioVIII{de}{dit}[ ]{e}is. \Antiphona{Spíritus}[\LinkLA]}%
    {\Inchoatio{Fez}[ ]{cho}ver sobre eles o maná, para co\MediatioVIII{me}[rem,] e deu-lhes o \TerminatioVIII{pão}[ ]{do}[ ]{céu}. \Antiphona{\Elisio{O} Espírito}[\LinkPT]},
  % 25
  {\Inchoatio{Pa}{nem} angelórum manducávit \MediatioVIII{ho}[mo,] cibária misit eis ad \TerminatioVIII{a}{bun}{dán}tiam. \Antiphona{Spíritus}[\LinkLA]}%
    {\Inchoatio{O}[ ]{ho}mem comeu o pão dos \MediatioVIII{an}[jos;] ele enviou-lhes alimento em \TerminatioVIII{a}{bun}{dân}cia. \Antiphona{\Elisio{O} Espírito}[\LinkPT]},
  % 29
  {\Inchoatio{Et}[ ]{man}ducavérunt et saturáti sunt \MediatioVIII{ni}[mis,] et desidérium eórum át\TerminatioVIII{tu}{lit}[ ]{e}is. \Antiphona{Spíritus}[\LinkLA]}%
    {\Inchoatio{E}{les} comeram e se far\MediatioVIII{ta}[ram,] tendo ele atendido ao \TerminatioVIII{seu}[ ]{de}\MediatioVIII{se}jo. \Antiphona{\Elisio{O} Espírito}[\LinkPT]}
}