\SetVerseAnnotation{Sl 45(44),1.8.11--12}

\SetLinkTextL{\Antiphona{Afferéntur in lætítia}}
\SetLinkTextR{\Antiphona{Serão trazidas com alegria}}

\SetVersePairs{
  % 8
  {\Inchoatio{Di}{le}xísti iustítiam et odísti iniqui\Flexa{tá}[tem,] proptérea unxit te Deus, Deus \MediatioII{tu}[us,] óleo lætítiæ præ consórti\TerminatioII{bus}[ ]{tu}is.}%
    {\Inchoatio{A}{mas}tes a justiça e odiastes a iniqui\Flexa{da}[de:] por isso, Deus vos ungiu, o vosso \MediatioII{Deus}[,] com o óleo da alegria, de preferência a vossos com\TerminatioII{pa}{nhei}ros.},
  % 11
  {\Inchoatio{Au}{di}, fília, et \Flexa{vi}[de,] et inclína aurem \MediatioII{tu}[am,] et oblivíscere pópulum tuum et domum pa\TerminatioII{tris}[ ]{tu}i.}%
    {\Inchoatio{Ou}{vi}, filha, \Flexa{ve}[de,] inclinai o ou\MediatioII{vi}[do,] esquecei o vosso povo e a casa de vos\TerminatioII{so}[ ]{pai}.},
  % 12
  {\Inchoatio{Et}[ ]{con}cupíscet rex spéciem \MediatioII{tu}[am,] quóniam ipse est dóminus tuus, et adó\TerminatioII{ra}[ ]{e}um.}%
    {\Inchoatio{E}[ ]{o} rei se apaixonará com a vossa be\MediatioII{le}[za.] Pois ele é vosso senhor, prostrai-vos dian\TerminatioII{te}[ ]{de}le.}
}