% !TeX root = ../../../main.tex
\DeclareDocumentCommand{\Annotation}{}{
  \MakeAnnotation{\CantusID{002089}[Sl 60(59),12]}{Sl 60(59),3--7.12cd--13}
}

\def\LinkLA{tempus-quadragesimae/dominica-1/communio/antiphona}
\def\LinkPT{\LinkLA-pt}

\def\VersePairs{
  % 3
  {\Inchoatio{In}{tén}de voci clamóris \MediatioVIII{me}[i,] rex meus et \TerminatioVIII{De}{us}[ ]{me}us. \Antiphona{Intéllege}[\LinkLA]}%
    {\Inchoatio{Fi}{cai} atento à voz do meu cla\MediatioVIII{mor}[,] meu Rei \TerminatioVIII{e}[ ]{meu}[ ]{Deus}. \Antiphona{Atendei}[\LinkPT]},
  % 4
  {\Inchoatio{Quó}{ni}am ad te orábo, \Flexa{Dó}[mine,] mane exáudies vocem \MediatioVIII{me}[am;] mane astábo tibi, et \TerminatioVIII{ex}{spec}{tá}bo. \Antiphona{Intéllege}[\LinkLA]}%
    {\Inchoatio{Pois}[ ]{é} a vós que eu rezo, Se\Flexa{nhor}[,] e de manhã ouvireis a minha \MediatioVIII{voz}[;] de manhã estarei na vossa presença e a\TerminatioVIII{guar}{da}{rei}. \Antiphona{Atendei}[\LinkPT]},
  % 5--6
  {\Inchoatio{Quó}{ni}am non Deus volens iniquitátem \Flexa{tu}[ es;] neque habitábit iuxta te ma\MediatioVIII{lí}[gnus,] neque permanébunt iniústi ante ó\TerminatioVIII{cu}{los}[ ]{tu}os. \Antiphona{Intéllege}[\LinkLA]}%
    {\Inchoatio{Pois}[ ]{vós} não sois um Deus que ame a iniqui\Flexa{da}[de;] e o perverso não habitará junto de \MediatioVIII{vós}[,] nem os injustos permanecerão ante os \TerminatioVIII{vos}{sos}[ ]{o}lhos. \Antiphona{Atendei}[\LinkPT]},
  % 7
  {\Inchoatio{O}{dí}sti omnes qui operántur iniqui\Flexa{tá}[tem,] perdes omnes qui loquúntur men\MediatioVIII{dá}[cium;] virum sánguinum et dolósum abominá\TerminatioVIII{bi}{tur}[ ]{Dó}minus. \Antiphona{Intéllege}[\LinkLA]}%
    {\Inchoatio{O}{di}ais a todos os que praticam a iniqui\Flexa{da}[de,] e fazeis perecer os que falam a men\MediatioVIII{ti}[ra;] o Senhor abomina quem derrama sangue ou co\TerminatioVIII{me}{te}[ ]{frau}de. \Antiphona{Atendei}[\LinkPT]},
  % 8
  {\Inchoatio{E}{go} autem in multitúdine misericórdiæ \Flexa{tu}[æ] introíbo in domum \MediatioVIII{tu}[am;] adorábo ad templum sanctum tuum in ti\TerminatioVIII{mó}{re}[ ]{tu}o. \Antiphona{Intéllege}[\LinkLA]}%
    {\Inchoatio{Eu}[, ]{po}rém, confiando na vossa grande miseri\Flexa{cór}[dia,] entrarei na vossa \MediatioVIII{ca}[sa;] cheio do vosso temor, me prostrarei ao vosso \TerminatioVIII{san}{to}[ ]{tem}plo. \Antiphona{Atendei}[\LinkPT]},
  % 9
  {\Inchoatio{Dó}{mi}ne, deduc me in iustítia vossa propter inimícos \MediatioVIII{me}[os,] dírige in conspéctu meo \TerminatioVIII{vi}{am}[ ]tuam. \Antiphona{Intéllege}[\LinkLA]}%
    {\Inchoatio{Se}{nhor}, guiai-me com a vossa justiça, contra meus ini\MediatioVIII{mi}[gos,] aplanai à minha frente o vos\TerminatioVIII{so}[ ]{ca}{mi}nho. \Antiphona{Atendei}[\LinkPT]},
  % 12ab
  {\Inchoatio{Et}[ ]{om}nes qui sperant in te læ\MediatioVIII{tén}[tur,] in ætér\TerminatioVIII{num}[ ]{ex}{súl}\-tent. \Antiphona{Intéllege}[\LinkLA]}%
    {\Inchoatio{Mas}[ ]{se} alegrem todos os que esperam em \MediatioVIII{vós}[,] e exultem \TerminatioVIII{pa}{ra}[ ]{sem}pre. \Antiphona{Atendei}[\LinkPT]},
  % 12cd
  {\Inchoatio{Ob}{um}brábis eis, et gloriabúntur \MediatioVIII{in}[ te] qui díligunt \TerminatioVIII{no}{men}[ ]{tu}um. \Antiphona{Intéllege}[\LinkLA]}%
    {\Inchoatio{Da}{rás} sombra para eles, e se gloriarão em \MediatioVIII{vós} os que amam o \TerminatioVIII{vos}{so}[ ]{no}me; \Antiphona{Atendei}[\LinkPT]},
  % 13
  {\Inchoatio{Quó}{ni}am tu benedíces iusto, \MediatioVIII{Dó}[mine;] quasi scuto, bona voluntáte coro\TerminatioVIII{ná}{bis}[ ]{e}um. \Antiphona{Intéllege}[\LinkLA]}%
    {\Inchoatio{Pois}[ ]{vós} abençoareis o justo, Se\MediatioVIII{nhor}[,] e com o escudo do vosso favor o pro\TerminatioVIII{te}{ge}reis. \Antiphona{Atendei}[\LinkPT]}
}