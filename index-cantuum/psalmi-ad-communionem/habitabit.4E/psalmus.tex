\SetVerseAnnotation{Sl 15(14)}

\SetLinkTextL{\Antiphona{Habitábit}}
\SetLinkTextR{\Antiphona{Habitará}}

\SetVersePairs{
  {\Inchoatio{Qui}[ ]{in}gréditur sine mácula, et operá\MediatioIV{tur}[ ]{ius}{tí}[tiam;] qui lóquitur veritátem \TerminatioIV{in}[ ]{cor}{de}[ ]{su}o.}%
    {\Inchoatio{A}{que}le que caminha sem mancha e pratica \MediatioIV{a}[ ]{jus}{ti}[\-ça,] fala a verdade que está \TerminatioIV{no}[ ]{co}{ra}{ção}.},
  %
  {\Inchoatio{Qui}[ ]{non} egit dolum in lingua \Flexa{su}[a,] nec fecit próximo \MediatioIV{su}{o}[ ]{ma}[lum,] et oppróbrium non íntulit \TerminatioIV{pró}{xi}{mo}[ ]{su}o.}%
    {\Inchoatio{Quem}[ ]{não} provoca engano com a sua \Flexa{lín}[gua,] não faz o mal \MediatioIV{a}[ ]{seu}[ ]{pró}[ximo,] e não insul\TerminatioIV{ta}[ ]{seu}[ ]{vi}{zi}nho.},
  %
  {\Inchoatio{Ad}[ ]{ní}hilum reputátus est in conspéctu e\MediatioIV{ius}[ ]{ma}{lí}[\-gnus,] timéntes autem Dó\TerminatioIV{mi}{num}[ ]{glo}{rífi}cat.}%
    {\Inchoatio{E}{le} tem por desprezível \MediatioIV{o}[ ]{mal}{va}[do,] mas honra os que te\TerminatioIV{mem}[ ]{o}[ ]{Se}{nhor}.},
  %
  {\Inchoatio{Qui}[ ]{iu}rat in detriméntum suum et non \Flexa{mu}[tat,] qui pecúniam suam non dedit \MediatioIV{ad}[ ]{u}{sú}[ram,] et múnera super innocén\TerminatioIV{tem}[ ]{non}[ ]{ac}{cé}pit.}%
    {\Inchoatio{Em}{bo}ra, ao jurar, tenha sido prejudicado, não \Flexa{mu}[da;] ele empresta seu dinheiro \MediatioIV{sem}[ ]{u}{su}[ra,] e não aceita suborno contra \TerminatioIV{o}[ ]{i}{no}{cen}te.},
  %
  {\Inchoatio{Qui}[ ]{fa}cit hæc, {\GreStar} non movébi\TerminatioIV{tur}[ ]{in}[ ]{æ}{tér}num.}%
    {\Inchoatio{Quem}[ ]{faz} \MediatioIV{es}{tas}[ ]{coi}[sas,] jamais se\TerminatioIV{rá}[ ]{a}{ba}{la}do.}
}