% !TeX root = ../../a4.tex
% chktex-file 1

\subsection{Entrada}\label{subsection:tempus-quadragesimae/dominica-1/introitus}
\MakeChantAntiphonPsalm{tunc-invocabis.7a}{psalmi-ad-introitum}

\AllowPageFlush

\subsection[Salmo Responsorial I]{Salmo Responsorial I \textmd{E 5}}\label{subsection:tempus-quadragesimae/dominica-1/psalmus-responsorius-1}
\MakeChantPsalmTwoVerses{psalmi-responsorii}{scapulis-suis.E5}

\AllowPageFlush

\subsection[Salmo Responsorial II]{Salmo Responsorial II \textmd{D 1 g}}\label{subsection:tempus-quadragesimae/dominica-1/psalmus-responsorius-2}
\MakeChantPsalmOneVerse{psalmi-responsorii}{ego-dixi.D1g}

\AllowPageFlush

\subsection{Antífona de aclamação}\label{subsection:tempus-quadragesimae/dominica-1/antiphona-acclamationis}
\MakeChantAntiphon{vobis-datum-est.6F}
\begin{rubrica}
  Então canta-se pelo menos um verso do salmo responsorial anterior que não foi utilizado.
\end{rubrica}

\subsection{Trato}\label{subsection:tempus-quadragesimae/dominica-1/tractus}
\MakeChantPsalmOneVerse{tractus}{ego-dixi}

\AllowPageFlush

\subsection{Ofertório}\label{subsection:tempus-quadragesimae/dominica-1/offertorium}
\MakeChantAntiphonPsalm{cum-facis-eleemosynam.1f}{psalmi-ad-offertorium}

\AllowPageFlush

\subsection{Comunhão}\label{subsection:tempus-quadragesimae/dominica-1/communio}
\MakeChantAntiphonPsalm{intellege.8c}{psalmi-ad-communionem}