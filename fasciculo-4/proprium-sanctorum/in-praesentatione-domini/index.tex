% !TeX root = ../../main.tex
% chktex-file 1
\def\Prefix{proprium-sanctorum/in-praesentatione-domini/}

\subsection{Antífona para a benção das velas}\label{subsection:proprium-sanctorum/in-praesentatione-domini/ad-benedictionem-candelarum}
\def\OfficePart{ad-benedictionem-candelarum/}

\begin{annotation}
  \MakeAnnotation{\CantusID{002508}[Is 35,4--5; Mt 24,30; Mc 13,26; 1 Esr 9,8]}{}
\end{annotation}

\iftoggle{compact}{%
  \GregorioImportL{\Prefix}{\OfficePart}{antiphona}{antiphona-pt}

  \nobreaksubsection{Convite para a procissão}

  \begin{rubrica}
    \begin{center}
      O díacono ou o celebrante diz:
    \end{center}
  \end{rubrica}

  \vspace{-20pt}

  \GregorioImportL{\Prefix}{\OfficePart}{ad-processionem}{ad-processionem-pt}

  \begin{rubrica}
    \begin{center}
      Todos respondem:
    \end{center}
  \end{rubrica}

  \vspace{-20pt}

  \GregorioImportL{\Prefix}{\OfficePart}{ad-processionem-responsorium}{ad-processionem-responsorium-pt}

  \clearpage

  \GregorioImportR{\Prefix}{\OfficePart}{antiphona}{antiphona-pt}

  \nobreaksubsection{Convite para a procissão}

  \begin{rubrica}
    \begin{center}
      O díacono ou o celebrante diz:
    \end{center}
  \end{rubrica}

  \vspace{-20pt}

  \GregorioImportR{\Prefix}{\OfficePart}{ad-processionem}{ad-processionem-pt}

  \begin{rubrica}
    \begin{center}
      Todos respondem:
    \end{center}
  \end{rubrica}

  \vspace{-20pt}

  \GregorioImportR{\Prefix}{\OfficePart}{ad-processionem-responsorium}{ad-processionem-responsorium-pt}
}{%
  \begin{paracol}{2}
    \GregorioImportLR{\Prefix}{\OfficePart}{antiphona}{antiphona-pt}
  \end{paracol}

  \nobreaksubsection{Convite para a procissão}

  \begin{rubrica}
    \begin{center}
      O díacono ou o celebrante diz:
    \end{center}
  \end{rubrica}

  \vspace{-20pt}

  \begin{paracol}{2}
    \GregorioImportLR{\Prefix}{\OfficePart}{ad-processionem}{ad-processionem-pt}
  \end{paracol}

  \begin{rubrica}
    \begin{center}
      Todos respondem:
    \end{center}
  \end{rubrica}

  \vspace{-20pt}

  \begin{paracol}{2}
    \GregorioImportLR{\Prefix}{\OfficePart}{ad-processionem-responsorium}{ad-processionem-responsorium-pt}
  \end{paracol}
}

\AllowPageFlush

\subsection{Procissão I}\label{subsection:proprium-sanctorum/in-praesentatione-domini/ad-processionem-1}
\MakeChantAntiphonPsalm{\Prefix}{ad-processionem-1/}

\AllowPageFlush

\subsection{Procissão II}\label{subsection:proprium-sanctorum/in-praesentatione-domini/ad-processionem-2}
\MakeChantAntiphonPsalm{\Prefix}{ad-processionem-2/}

\subsection{Procissão III}
\begin{rubrica}
  Ver Solenidade de São José, Esposo da Virgem Maria, página~\pageref{subsection:proprium-sanctorum/sancti-ioseph-sponsi-bmv/psalmus-responsorius}.
\end{rubrica}

\AllowPageFlush

\subsection{Entrada}\label{subsection:proprium-sanctorum/in-praesentatione-domini/introitus}
\MakeChantAntiphonPsalm{\Prefix}{introitus/}

\nobreaksubsection{Salmo Responsorial}

\begin{rubrica}
  Ver Solenidade da Imaculada Conceição de Nossa Senhora, página~\pageref{subsection:proprium-sanctorum/in-conceptione-immaculata-bmv/psalmus-responsorius}.
\end{rubrica}

\subsection{Aleluia}\label{subsection:proprium-sanctorum/in-praesentatione-domini/alleluia}
\MakeChantAntiphonPsalm{\Prefix}{alleluia/}

\AllowPageFlush

\subsection[Salmo Aleluiático]{Salmo Aleluiático \textmd{E 2 d}}\label{subsection:proprium-sanctorum/in-praesentatione-domini/psalmus-alleluiaticus}
\MakeChantPsalmOneVerse{\Prefix}{psalmus-alleluiaticus/}

\AllowPageFlush

\subsection{Ofertório}\label{subsection:proprium-sanctorum/in-praesentatione-domini/offertorium}
\MakeChantAntiphonPsalm{\Prefix}{offertorium/}

\AllowPageFlush

\subsection{Comunhão}\label{subsection:proprium-sanctorum/in-praesentatione-domini/communio}
\MakeChantAntiphonPsalm{\Prefix}{communio/}[%
  \begin{rubrica}
    Pode-se complementar com um cântico de comunhão à escolha da página~\pageref{appendix:communio-ad-libitum/communio-1} e seguintes.
  \end{rubrica}%
]