\SetVerseAnnotation{Sl 19(18),1--4.6}

\SetLinkTextL{\Antiphona{In omnem terram}}
\SetLinkTextR{\Antiphona{Por tod\Elisio{a} a terra}}

\SetVersePairs{
  % 2
  {\Inchoatio{Di}{es} diéi erúctat \MediatioII{ver}[bum,] et nox nocti índicat \TerminatioII{sci}{\-én}\-tiam.}%
    {\Inchoatio{O}[ ]{di}a transmite ao dia a men\MediatioII{sa}[gem] e a noite dá conhecimento a ou\TerminatioII{tra}[ ]{noi}te.},
  % 3
  {\Inchoatio{Non}[ ]{sunt} loquélæ neque ser\MediatioII{mó}[nes,] quorum non intellegán\TerminatioII{tur}[ ]{vo}ces.}%
    {\Inchoatio{Não}[ ]{são} falas, nem dis\MediatioII{cur}[sos,] nem se ouve a su\TerminatioII{a}[ ]{voz}.},
  % 4
  {\Inchoatio{So}{li} pósuit tabernáculum in \Flexa{e}[is,] et ipse tamquam sponsus procédens de thálamo \MediatioII{su}[o,] exsultávit ut gigas ad currén\TerminatioII{dam}[ ]{vi}am.}%
    {\Inchoatio{A}{li} armou uma tenda para o \Flexa{sol} e sai como um noivo do quarto nupci\MediatioII{al}[,] e exulta como um gigante a percorrer o seu \TerminatioII{ca}{mi}nho.},
  % 6
  {\Inchoatio{A}[ ]{fí}nibus cælórum egréssio \Flexa{e}[ius,] et occúrsus eius usque ad fines e\MediatioII{ó}[rum,] nec est quod abscondátur a caló\TerminatioII{re}[ ]{e}ius.}%
    {\Inchoatio{A}[ ]{su}a saída é desde os confins dos \Flexa{céus} e o seu percurso vai até o outro ex\MediatioII{tre}[mo,] e nada pode subtrair-se ao seu \TerminatioII{ca}{lor}.}
}