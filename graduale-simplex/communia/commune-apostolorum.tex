% !TeX root = ../../a4.tex
% chktex-file 1

\subsection{Entrada}\label{subsection:communia/commune-apostolorum/introitus}
\SetAntiphonAnnotation{\CantusID{003262}[Sl 19(18),5]}
\SetAntiphonScore{in-omnem-terram.2D/}
\MakeChantAntiphonPsalm{introitus/}

\nobreaksubsection{Salmo Responsorial}
\begin{rubrica}
  Ver Solenidade de São Pedro e São Paulo, Apóstolos, página~\pageref{subsection:proprium-sanctorum/sanctorum-petri-et-pauli-apostolorum/psalmus-responsorius}.
\end{rubrica}
\vspace{-2mm}

\nobreaksubsection{Aleluia}
\begin{rubrica}
  Ver Solenidade de São Pedro e São Paulo, Apóstolos, página~\pageref{subsection:proprium-sanctorum/sanctorum-petri-et-pauli-apostolorum/alleluia}.
\end{rubrica}
\vspace{-2mm}

\nobreaksubsection{Salmo Aleluiático}
\begin{rubrica}
  Ver Solenidade de São Pedro e São Paulo, Apóstolos, página~\pageref{subsection:proprium-sanctorum/sanctorum-petri-et-pauli-apostolorum/psalmus-alleluiaticus}.
\end{rubrica}
\vspace{-2mm}

\subsection{Ofertório}\label{subsection:communia/commune-apostolorum/offertorium}
\SetAntiphonAnnotation{\CantusID{004999}[Lc 4,18]}
\SetAntiphonScore{spiritus-domini-super-me.2D/}
\MakeChantAntiphonPsalm{offertorium/}

\AllowPageBreak

\subsection{Comunhão}\label{subsection:communia/commune-apostolorum/communio}
\SetAntiphonAnnotation{\CantusID{005502}[Mt 19,28]}
\SetAntiphonScore{vos-qui-secutis-estis-me.1g/}
\MakeChantAntiphonPsalm{communio/}