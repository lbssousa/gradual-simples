\SetVerseAnnotation{Sl 119(118),1.10--11.30--32.40.44--45.47}

\SetLinkTextL{\Responsorium{Allelúia}}
\SetLinkTextR{\Responsorium{Aleluia}}

\def\VersePairs{
  % 10
  {In toto corde meo exqui\MediatioE{sí}[vi te,] ne erráre me fácias a præcéptis \TerminatioEII{tu}is.}%
    {Com todo o meu coração tenho-vos bus\MediatioE{ca}[do,] não me faças desviar dos vossos manda\TerminatioEII{men}tos.},
  % 11
  {In corde meo abscóndi elóquia \MediatioE{tu}[a,] ut non peccem \MediatioE{ti}bi.}%
    {Encerrei no meu coração a vossa pro\MediatioE{mes}[sa,] para não pecar contra \MediatioE{vós}.},
  % 30
  {Viam veritátis e\MediatioE{lé}[gi,] iudícia tua propósui \TerminatioEII{mi}hi.}%
    {Escolhi o caminho da ver\MediatioE{da}[de,] propus para mim os vossos julga\TerminatioEII{men}tos.},
  % 31
  {Adhǽsi testimóniis tuis, \MediatioE{Dó}[mine;] noli me con\-\TerminatioEII{fún}\-de\-re.}%
    {Aderi aos vossos ensina\MediatioE{men}[tos;] que eu não seja envergo\TerminatioEII{\-nha}\-do.},
  % 32
  {Viam mandatórum tuórum \MediatioE{cur}[ram,] quia dilatásti cor \TerminatioEII{me}um.}%
    {Correrei pelo caminho dos vossos manda\MediatioE{men}[tos;] pois dilatastes o meu cora\TerminatioEII{ção}.},
  % 40
  {Ecce concupívi mandáta \MediatioE{tu}[a;] in iustítia tua vivífi\TerminatioEII{ca} me.}%
    {Desejo vossos pre\MediatioE{cei}[tos;] na vossa justiça, fazei-me vi\TerminatioEII{ver}.},
  % 44
  {Et custódiam legem tuam \MediatioE{sem}[per,] in sǽculum et in sǽculum \TerminatioEII{sǽ}culi.}%
    {Eu guardarei sempre a vossa \MediatioE{Lei}[,] para sempre e eterna\TerminatioEII{\-men}\-te.},
  % 45
  {Et ambulábo in lati\MediatioE{tú}[dine,] quia mandáta tua exqui\TerminatioEII{sí}vi.}%
    {Andarei por um caminho espa\MediatioE{ço}[so,] pois procurei vossos pre\TerminatioEII{cei}tos.},
  % 47
  {Et delectábor in præcéptis \MediatioE{tu}[is,] quæ di\TerminatioEII{lé}xi.}%
    {Sinto prazer nos vossos manda\MediatioE{men}[tos,] que tanto eu \TerminatioEII{a}mo.}
}