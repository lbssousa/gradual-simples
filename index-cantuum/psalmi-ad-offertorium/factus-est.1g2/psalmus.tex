\SetLinkTextL{\Antiphona{Factus est}}
\SetLinkTextR{\Antiphona{Meu Deus}}

\SetFirstVersePair{
  {\VSup{2}\Inchoatio{Ex}{al}tábo te, Dómine, quóniam \MediatioI{ex}{tra}{xí}[sti me,] nec delectásti inimícos \TerminatioI{me}{os}[ ]{su}per me.}%
    {\Inchoatio{Se}{nhor}, vos exaltarei porque \MediatioI{me}{ li}{vras}[tes,] e não deixastes que meus inimigos se divertissem à \TerminatioI{mi}{nha}[ ]{cus}ta.}
}

\SetVersePairs{
  {\VSup{3}\Inchoatio{Dó}{mi}ne Deus meus, cla\MediatioI{má}{vi }{ad}[ te,] \TerminatioI{et}[ ]{sa}{nás}ti me.}%
    {\Inchoatio{Se}{nhor}, meu Deus, a \MediatioI{vós}{ cla}{mei} e \TerminatioI{me}[ ]{cu}{ras}tes.},
  {\VSup{10c}\Inchoatio{Num}{quid} confitébitur \MediatioI{ti}{bi }{pul}[vis,] aut annuntiábit veri\TerminatioI{tá}{tem}[ ]{tu}am?}%
    {\Inchoatio{A}{ca}so o pó \MediatioI{vai}{ lou}{var}[-vos] e proclamar a vossa fi\TerminatioI{de}{li}{da}de?},
  % Nota: o texto original da Bíblia Oficial da CNBB está no modo verbal 
  %       indicativo, mas aqui nós mudamos a conjugação verbal para o modo
  %       imperativo, em conformidade com o texto correspondente
  %       no original em latim.
  {\VSup{11}\Inchoatio{Au}{di}, Dómine, et mise\MediatioI{ré}{re }{me}[i,] Dómine, esto mi\TerminatioI{hi}[ ]{ad}{iú}tor.}%
    {\Inchoatio{Ou}{vi}, Senhor, e tende compai\MediatioI{xão}{ de }{mim}[;] Senhor, tor\-nai-vos o meu \TerminatioI{pro}{te}{tor}.}
}