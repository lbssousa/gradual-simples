\SetVerseAnnotation{Sl 129(128)}

\SetLinkTextL{\Responsorium{Sæpe}}
\SetLinkTextR{\Responsorium{Muitas vezes}}

\SetVersePairs{
  % 3
  {Supra dorsum meum aravérunt \MediatioCS{a}{ra}{tó}[res,] prolongavérunt sulcos \TerminatioCS{su}os.}%
    {Sobre as minhas costas araram os \MediatioCS{la}{vra}{do}[res,] a\-brin\-do seus longos \TerminatioCS{sul}cos.},
  % 4
  {Dóminus \MediatioCS{au}{tem}[ ]{iu}[stus] concídit cervíces pec\-ca\TerminatioCS{\-tó}\-rum.}%
    {O Senhor, porém, \MediatioCS{que}[ ]{é}[ ]{jus}[to,] livrou-me do poder dos \TerminatioCS{ím}pios.},
  % 5
  {Confundántur et convertán\MediatioCS{tur}[ ]{re}{trór}[sum] omnes, qui odérunt \TerminatioCS{Si}on.}%
    {Fiquem confundidos e voltem \MediatioCS{pa}{ra}[ ]{trás} todos os que odeiam Si\TerminatioCS{ão}.},
  % 6
  {Fiant sicut fe\MediatioCS{num}[ ]{tec}{tó}[rum,] quod, priúsquam evellátur, ex\TerminatioCS{á}ruit.}%
    {Sejam como a erva \MediatioCS{dos}[ ]{te}{lha}[dos,] que antes de arrancada já está \TerminatioCS{se}ca.},
  % 7
  {De quo non implévit manum su\MediatioCS{am}[, ]{qui}[ ]{me}[tit,] et sinum suum, qui manípulos \TerminatioCS{cól}ligit.}%
    {Com ela, o ceifador não en\MediatioCS{che}[ ]{a}[ ]{mão}[;] nem o colo, o que recolhe os \TerminatioCS{fei}xes.},
  % 8
  {Et non dixérunt, qui præte\Flexa*{rí}{bant:} ``Benedíctio Dó\MediatioCS{mi}{ni}[ ]{su}[per vos,] benedícimus vobis in nómine \TerminatioCS{Dó}mini''.}%
    {Não lhes digam os que \Flexa*{pas}[sam:] ``A bênção do Senhor venha \MediatioCS{so}{bre}[ ]{vós}[,] nós vos abençoamos em nome do Se\TerminatioCS{nhor}!''}
}