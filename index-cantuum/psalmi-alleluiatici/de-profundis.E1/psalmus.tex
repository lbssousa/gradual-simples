\SetVerseAnnotation{Sl 130(129),1--5.6b--8}




\SetVersePairs{
  %
  {Fiant aures tuæ inten\MediatioE{dén}[tes] in vocem deprecatió\TerminatioEI{nis}[ ]{me}æ.~\Responsorium}%
    {Vossos ouvidos estejam bem a\MediatioE{ten}[tos] ao clamor da mi\TerminatioEI{nha}[ ]{pre}ce.~\Responsorium},
  %
  {Si iniquitátes observáveris, \MediatioE{Dó}[mine,] Dómine, quis su\TerminatioEI{sti}{\-né}\-bit?~\Responsorium}%
    {Se considerardes as iniquidades, Se\MediatioE{nhor}[,] Senhor, quem poderá sub\TerminatioEI{sis}{tir}?~\Responsorium},
  %
  {Quia apud te propitiáti\MediatioE{o}[ est,]  et timé\TerminatioEI{bi}{mus} te.~\Responsorium}%
    {Mas em vós se encontra o per\MediatioE{dão}[,] para que sejais \TerminatioEI{te}{mi}do.~\Responsorium},
  %
  {Sustínui te, \Flexa*{Dó}[mine,] sustínuit ánima mea in verbo \MediatioE{e}[ius,] sperávit ánima mea \TerminatioEI{in}[ ]{Dó}mino.~\Responsorium}%
    {Eu aguardo o Se\Flexa*{nhor}[!] Aguardo com toda a minha alma e espero na sua pa\MediatioE{la}[vra;] a minha alma espera pelo \TerminatioEI{Se}{nhor}.~\Responsorium},
  %
  {Magis quam custódes au\MediatioE{ró}[ram,] speret Israel \TerminatioEI{in}[ ]{Dó}mino.~\Responsorium}%
    {Mais do que os vigias pela au\MediatioE{ro}[ra,] espere Israel pelo \TerminatioEI{Se}{nhor}.~\Responsorium},
  %
  {Quia apud Dóminum miseri\MediatioE{cór}[dia,] et copiósa apud eum \TerminatioEI{red}{émp}tio.~\Responsorium}%
    {Pois no Senhor está a miseri\MediatioE{cór}[dia] e junto a ele, copiosa re\TerminatioEI{den}{ção}!~\Responsorium},
  %
  {Et ipse rédimet \MediatioE{Is}[rael] ex ómnibus iniquitáti\TerminatioEI{bus}[ ]{e}ius.~\Responsorium}%
    {Ele redimirá Isra\MediatioE{el} de todas as suas ini\TerminatioEI{qui}{da}des.~\Responsorium}
}