% !TeX root = ../../../a4.tex
\DeclareDocumentCommand{\Annotation}{}{
  \MakeAnnotation{\CantusID{005214}[Sl 119(118),4]}{Sl 119(118),1.52.81.89.105.114.116.133.138.145}
}

\def\LinkLA{tempus-per-annum/missa-7/communio/antiphona}
\def\LinkPT{\LinkLA-pt}

\def\VersePairs{
  % 52
  {\Inchoatio{Me}{mor} fui iudiciórum tuórum a \MediatioVII{sǽ}{culo, }{Dó}[mine,] et \TerminatioVII{con}{so}{lá}tus sum. \Antiphona{Tu mandásti, Dómine}[\LinkLA]}%
    {\Inchoatio{Lem}{bro}-me, Senhor, dos vossos julga\MediatioVII{men}{tos an}{ti}[\-gos,] e sinto-me \TerminatioVII{con}{so}{la}do. \Antiphona{Vós mandastes, ó Senhor}[\LinkPT]},
  % 81
  {\Inchoatio{De}{fé}cit in salutáre tuum \MediatioVII{á}{nima }{me}[a,] et in verbum tuum su\TerminatioVII{per}{spe}{rá}vi. \Antiphona{Tu mandásti, Dómine}[\LinkLA]}%
    {\Inchoatio{Mi}{nha} alma se consome aguardando vossa \MediatioVII{sal}{va}{ção}[;] toda a minha esperança está na \TerminatioVII{vos}{sa pa}{la}vra. \Antiphona{Vós mandastes, ó Senhor}[\LinkPT]},
  % 89
  {\Inchoatio{In}[ ]{æ}\MediatioVII{ter}{num, }{Dó}[mine,] verbum tuum constitútum \TerminatioVII{est}{ in }{cæ}lo. \Antiphona{Tu mandásti, Dómine}[\LinkLA]}%
    {\Inchoatio{Pa}{ra} \MediatioVII{sem}{pre, Se}{nhor}[,] vossa palavra está \TerminatioVII{fir}{me no }{céu}. \Antiphona{Vós mandastes, ó Senhor}[\LinkPT]},
  % 105
  {\Inchoatio{Lu}{cér}na pédibus meis \MediatioVII{ver}{bum }{tu}[um,] et lumen se\TerminatioVII{mí}{tis }{me}is. \Antiphona{Tu mandásti, Dómine}[\LinkLA]}%
    {\Inchoatio{Lâm}{pa}da para os meus pés é a \MediatioVII{vos}{sa pa}{la}[vra,] e luz para \TerminatioVII{mi}{nhas ve}{re}das. \Antiphona{Vós mandastes, ó Senhor}[\LinkPT]},
  % 114
  {\Inchoatio{Teg}{men} et scutum \MediatioVII{me}{um }{es}[ tu,] et in verbum tuum su\TerminatioVII{per}{spe}{rá}vi. \Antiphona{Tu mandásti, Dómine}[\LinkLA]}%
    {\Inchoatio{Meu}[ ]{pro}tetor e meu es\MediatioVII{cu}{do sois }{vós}[,] pus toda a minha esperança na \TerminatioVII{vos}{sa pa}{la}vra. \Antiphona{Vós mandastes, ó Senhor}[\LinkPT]},
  % 116
  {\Inchoatio{Sú}{sci}pe me secúndum elóquium \MediatioVII{tu}{um, et }{vi}[vam;] et non confúndas me ab exspectati\TerminatioVII{ó}{ne }{me}a. \Antiphona{Tu mandásti, Dómine}[\LinkLA]}%
    {\Inchoatio{Re}{ce}bei-me segundo a vossa promessa, e \MediatioVII{vi}{ve}{rei}[;] não me decepcioneis na minha \TerminatioVII{es}{pe}{ran}ça. \Antiphona{Vós mandastes, ó Senhor}[\LinkPT]},
  % 133
  {\Inchoatio{Gres}{sus} meos dírige secúndum e\MediatioVII{ló}{quium }{tu}[um,] et non dominétur mei \TerminatioVII{om}{nis i}{ní}quitas. \Antiphona{Tu mandásti, Dómine}[\LinkLA]}%
    {\Inchoatio{Di}{ri}gi meus passos segundo a \MediatioVII{vos}{sa pro}{mes}[sa,] e nenhuma iniquidade dominará \TerminatioVII{so}{bre }{mim}. \Antiphona{Vós mandastes, ó Senhor}[\LinkPT]},
  % 138
  {\Inchoatio{Man}{dá}sti in iustítia testi\MediatioVII{mó}{nia }{tu}[a,] et in veri\TerminatioVII{tá}{te }{ni}mis. \Antiphona{Tu mandásti, Dómine}[\LinkLA]}%
    {\Inchoatio{Com}[ ]{jus}tiça promulgastes os vossos en\MediatioVII{si}{na}{men}[tos,] e com suprema fi\TerminatioVII{de}{li}{da}de. \Antiphona{Vós mandastes, ó Senhor}[\LinkPT]},
  % 145
  {\Inchoatio{Cla}{má}vi in toto corde, ex\MediatioVII{áu}{di me, }{Dó}[mine;] iustificatiónes \TerminatioVII{tu}{as ser}{vá}bo. \Antiphona{Tu mandásti, Dómine}[\LinkLA]}%
    {\Inchoatio{Cla}{mo} de todo o coração: aten\MediatioVII{dei}{-me, Se}{nhor}[;] observarei os vossos \TerminatioVII{jus}{tos de}{cre}tos. \Antiphona{Vós mandastes, ó Senhor}[\LinkPT]}
}