% !TeX root = ../../a4.tex
% chktex-file 1

\subsection{Entrada}\label{subsection:proprium-sanctorum/in-nativitate-sancti-ioannes-baptistae/introitus}
\MakeChantAntiphonPsalm{dominus-ab-utero.8G}{introitus}

\AllowPageFlush

\subsection[Salmo Responsorial]{Salmo Responsorial \textmd{E 5}}\label{subsection:proprium-sanctorum/in-nativitate-sancti-ioannes-baptistae/psalmus-responsorius}
\MakeChantPsalmThreeVerses{psalmi-responsorii}{in-te-domine-speravi.E5}

\subsection{Aleluia}\label{subsection:proprium-sanctorum/in-nativitate-sancti-ioannes-baptistae/alleluia}
\MakeChantAlleluiaPsalm{1g2/}{ad-te-domine.1g2}

\AllowPageFlush

\subsection[Salmo Aleluiático]{Salmo Aleluiático \textmd{C 4}}\label{subsection:proprium-sanctorum/in-nativitate-sancti-ioannes-baptistae/psalmus-alleluiaticus}
\begin{rubrica}
  O primeiro {\normalfont\Rbar} pode ser cantado apenas pelo grupo de cantores ou por todos. O segundo {\normalfont\Rbar} é cantado por todos.
\end{rubrica}
\MakeChantPsalmOneVerse{psalmi-alleluiatici}{ad-te-domine.C4}

\AllowPageFlush

\subsection{Ofertório}\label{subsection:proprium-sanctorum/in-nativitate-sancti-ioannes-baptistae/offertorium}
\MakeChantAntiphonPsalm{puer.7d}{offertorium}

\AllowPageFlush

\subsection{Comunhão}\label{subsection:proprium-sanctorum/in-nativitate-sancti-ioannes-baptistae/communio}
\MakeChantAntiphonPsalm{tu-puer.3a}{communio}