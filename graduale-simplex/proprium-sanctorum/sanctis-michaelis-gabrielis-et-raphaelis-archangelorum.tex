% !TeX root = ../../a4.tex
% chktex-file 1

\subsection{Entrada}\label{subsection:proprium-sanctorum/sanctis-michaelis-gabrielis-et-raphaelis-archangelorum/introitus}
\SetAntiphonAnnotation{\CantusID{001700}[Sl 103(102),20; Tb 13,10]}
\SetAntiphonScore{benedicite-dominum.8G/}
\MakeChantAntiphonPsalm{introitus/}

\AllowPageFlush

\subsection[Salmo Responsorial]{Salmo Responsorial \textmd{E 5 \protect\GreStar}}\label{subsection:proprium-sanctorum/sanctis-michaelis-gabrielis-et-raphaelis-archangelorum/psalmus-responsorius}
\MakeChantPsalmTwoVerses{psalmi-responsorii/}{laudate-dominum-de-caelis.E5/}

\subsection{Aleluia}\label{subsection:proprium-sanctorum/sanctis-michaelis-gabrielis-et-raphaelis-archangelorum/alleluia}
\SetAntiphonScore{alleluia.3g/}
\MakeChantAntiphonPsalm{alleluia/}[benedic-anima-mea-domino.3g/]

\AllowPageFlush

\subsection[Salmo Aleluiático]{Salmo Aleluiático \textmd{C 4}}\label{subsection:proprium-sanctorum/sanctis-michaelis-gabrielis-et-raphaelis-archangelorum/psalmus-alleluiaticus}
\begin{rubrica}
  O primeiro {\normalfont\Rbar} pode ser cantado apenas pelo grupo de cantores ou por todos. O segundo {\normalfont\Rbar} é cantado por todos.
\end{rubrica}
\MakeChantPsalmOneVerse{psalmi-alleluiatici/}{benedic-anima-mea-domino.C4/}

\AllowPageFlush

\subsection{Ofertório}\label{subsection:proprium-sanctorum/sanctis-michaelis-gabrielis-et-raphaelis-archangelorum/offertorium}
\SetAntiphonAnnotation{\CantusID{005029}[Ap 8,3]}
\SetAntiphonScore{stetit-angelus.4A/}
\MakeChantAntiphonPsalm{offertorium/}

\AllowPageFlush

\subsection{Comunhão}\label{subsection:proprium-sanctorum/sanctis-michaelis-gabrielis-et-raphaelis-archangelorum/communio}
\SetAntiphonAnnotation{\CantusID{004116}[Sl 148,1]}
\SetAntiphonScore{omnes-angeli-eius.5a/}
\MakeChantAntiphonPsalm{communio/}