% chktex-file 1
\subsection{Antífona de Entrada}\label{subsection:tempus-per-annum/sanctissimi-corporis-et-sanguinis-christi/introitus}
\MakeChantAntiphon{sacerdos-in-aeternum.1f}
\nobreaksubsection{Salmo 109(110)}
\MakeChantPsalm{psalmi-ad-introitum}{sacerdos-in-aeternum.1f}

\AllowPageFlush

\subsection[Salmo Responsorial]{Salmo Responsorial}\label{subsection:tempus-per-annum/sanctissimi-corporis-et-sanguinis-christi/psalmus-responsorius}
\subsubsection{Salmo 144(145)}
\MakeChantPsalmMulti{psalmi-responsorii}{exaltabo-te.E5}{
  {psalmus-v1}{psalmus-v1-pt},
  {psalmus-v2}{psalmus-v2-pt}
}

\AllowPageFlush

\subsection{Sequência}\label{subsection:tempus-per-annum/sanctissimi-corporis-et-sanguinis-christi/sequentia}
\begin{rubrica}
  A forma breve está entre colchetes.
\end{rubrica}
\MakeChantHymn{sequentiae}{lauda-sion}

\subsection{Aleluia}\label{subsection:tempus-per-annum/sanctissimi-corporis-et-sanguinis-christi/alleluia}
\MakeChantAlleluia{3g}
\nobreaksubsection{Salmo 77(78)}
\MakeChantPsalm{psalmi-ad-alleluia}{attendite-popule-meus.3g}

\AllowPageFlush

\subsection[Salmo Aleluiático]{Salmo Aleluiático}\label{subsection:tempus-per-annum/sanctissimi-corporis-et-sanguinis-christi/psalmus-alleluiaticus}
\nobreaksubsection{Salmo 77(78)}
\begin{rubrica}
  O primeiro {\normalfont\Rbar} pode ser cantado apenas pelo grupo de cantores ou por todos. O segundo {\normalfont\Rbar} é cantado por todos.
\end{rubrica}
\MakeChantPsalm{psalmi-alleluiatici}{attendite-popule-meus.C4}

\AllowPageFlush

\subsection{Antífona de Ofertório}\label{subsection:tempus-per-annum/sanctissimi-corporis-et-sanguinis-christi/offertorium}
\MakeChantAntiphon{angelorum-esca.2D}
\nobreaksubsection{Salmo 83(84)}
\MakeChantPsalm{psalmi-ad-offertorium}{angelorum-esca.2D}

\AllowPageFlush

\subsection{Antífona de Comunhão}\label{subsection:tempus-per-annum/sanctissimi-corporis-et-sanguinis-christi/communio}
\MakeChantAntiphon{ego-sum-panis-vivus.1f}
\nobreaksubsection{Salmo 22(23)}
\MakeChantPsalm{psalmi-ad-communionem}{ego-sum-panis-vivus.1f}

\subsection{Procissão}\label{subsection:tempus-per-annum/sanctissimi-corporis-et-sanguinis-christi/ad-processionem}
\begin{rubrica}
  Cantam-se os versos 1 a 4 abaixo, repetidamente ou seguidos de outros hinos eucarísticos, durante a procissão, se necessário. Após a chegada ao altar, cantam-se os versos 5 e 6 do hino.
\end{rubrica}
\MakeChantHymnMulti{hymni}{pange-lingua}{
  {psalmus-v1}{psalmus-v1-pt},
  {psalmus-v2}{psalmus-v2-pt},
  {psalmus-v3}{psalmus-v3-pt},
  {psalmus-v4}{psalmus-v4-pt},
  {psalmus-v5}{psalmus-v5-pt},
  {psalmus-v6}{psalmus-v6-pt}
}