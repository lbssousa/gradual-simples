\SetLinkTextL{\Responsorium{Quia}}
\SetLinkTextR{\Responsorium{Pois}}

\SetFirstVersePair{
  {\VSup{5}Ego dixi: Dómine, miserére \MediatioD{me}[i;] sana ánimam \TerminatioDI{me}am.}%
    {Eu disse: tende piedade de mim, Se\MediatioD{nhor}[,] curai minha \TerminatioDI{al}ma.}
}

\SetVersePairs{
  {\VSup{6}Inimíci mei dixérunt mala \MediatioD{mi}[hi:] ``Quando moriétur, et períbit nomen, \TerminatioDI{e}ius''?}%
    {Meus inimigos dizem maldades a meu res\MediatioD{pei}[to:] ``Quan\-do é que vai morrer, e se extinguirá o seu \TerminatioDI{no}me''?},
  {\VSup{7}Et si ingrediebátur, ut visitáret, vana loque\Flexa*{bá}[tur;] cor eius congregábat iniquitátem \MediatioD{si}[bi,] egrediebátur foras et detra\TerminatioDI{hé}bat.}%
    {Se alguém entrava, para me visitar, falava \Flexa*{fal}[so:] seu coração se enchia de mal\MediatioD{da}[de,] e, ao sair, ainda caluni\TerminatioDI{a}va.},
  {\VSup{8}Simul advérsum me susurrábant omnes inimíci \MediatioD{me}[i;] advérsum me cogitábant mala \TerminatioDI{mi}{hi:}}%
    {Juntos murmuravam contra mim meus ini\MediatioD{mi}[gos,] contra mim planejavam o \TerminatioDI{mal}:},
  % Nota: substituído o termo hebraico original
  % da Bíblia oficial da CNBB "algo de Belial" por "um mal mortal",
  % tal como se encontra na Bíblia Ave-Maria.
  {\VSup{9}``Malefícium effúsum est in \MediatioD{e}[o;] et qui decúmbit non adíciet ut re\TerminatioDI{súr}gat''.}%
    {``Um mal mortal o atin\MediatioD{giu}[;] agora que deitou, não vai mais levan\TerminatioDI{tar}-se''.},
  {\VSup{10}Sed et homo pacis meæ, in quo spe\MediatioD{rá}[vi,] qui edébat panem meum, levávit contra me cal\TerminatioDI{cá}neum.}%
    {Até meu amigo pessoal, no qual eu confi\MediatioD{a}[va,] que comia à minha mesa, levantou o calcanhar contra \TerminatioDI{mim}.}
}