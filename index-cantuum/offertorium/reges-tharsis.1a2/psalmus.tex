\SetVerseAnnotation{Sl 72(71),7--9.15--16ab}

\SetLinkTextL{\Antiphona{Reges Tharsis}}
\SetLinkTextR{\Antiphona{Os reis de Társis}}

\SetVersePairs{
  % 8
  {\Inchoatio{Et}[ ]{do}minábitur a mari \MediatioI{us}{que ad }{ma}[re,] et a Flúmine usque ad términos or\TerminatioI{bis}[ ]{ter}{rá}rum.}%
    {\Inchoatio{E}{le} dominará de \MediatioI{mar}{ a }{mar}[,] e desde o rio até os con\TerminatioI{fins}[ ]{da}[ ]{ter}ra.},
  % 9
  {\Inchoatio{Co}{ram} illo prócident ínco\MediatioI{læ}{ de}{sér}[ti,] et inimíci eius \TerminatioI{ter}{ram}[ ]{lin}gent.}%
    {\Inchoatio{Di}{an}te dele se prostrarão os habitantes \MediatioI{do}{ de}{ser}[to,] e seus inimigos lambe\TerminatioI{rão}[ ]{a}[ ]{ter}ra.},
  % 15
  {\Inchoatio{Et}[ ]{vi}vet, et dábitur ei de auro A\Flexa{rá}[biæ,] et orábunt pro \MediatioI{ip}{so }{sem}[per;] tota die bene\TerminatioI{dí}{cent}[ ]{e}i.}%
    {\Inchoatio{E}{le} viverá, e o ouro de Sabá lhe será \Flexa{da}[do;] orarão \MediatioI{sem}{pre por }{e}[le,] e o dia todo o \TerminatioI{ben}{di}{rão}.},
  % 16ab
  {\Inchoatio{Et}[ ]{e}rit ubértas fru\MediatioI{mén}{ti in }{ter}[ra,] in summis móntium \TerminatioI{fluc}{tu}{á}bit.}%
    {\Inchoatio{Ha}{ve}rá fartura de \MediatioI{tri}{go na }{ter}[ra,] ondulando no alto \TerminatioI{das}[ ]{mon}\MediatioI{ta}nhas.}
}