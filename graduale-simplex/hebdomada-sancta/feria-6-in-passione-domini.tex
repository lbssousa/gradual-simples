% !TeX root = ../../a4.tex
% chktex-file 1

\subsection[Salmo Responsorial I]{Salmo Responsorial I \textmd{C 3 a}}\label{subsection:hebdomada-sancta/feria-6-in-passione-domini/psalmus-responsorius-1}
\MakeChantPsalmThreeVerses{psalmi-responsorii}{in-manus-tuas-domine.C3a}

\subsection[Salmo Responsorial II]{Salmo Responsorial II \textmd{C 2 g}}\label{subsection:hebdomada-sancta/feria-6-in-passione-domini/psalmus-responsorius-1}
\MakeChantPsalmOneVerse{psalmi-responsorii}{spera-in-deo.C2g}

\AllowPageFlush

\subsection{Apresentação da Santa Cruz}\label{subsection:hebdomada-sancta/feria-6-in-passione-domini/ad-detegendam-sanctam-crucem}
\begin{rubrica}
  O sacerdote repete o cântico a seguir três vezes, em tons ascendentes:
\end{rubrica}
\MakeChantPsalm{varia}{ecce-lignum-crucis}

\subsection{Adoração da Santa Cruz I}\label{subsection:hebdomada-sancta/feria-6-in-passione-domini/ad-adoratione-sanctam-crucem-1}
\MakeChantAntiphonPsalm{popule-meus.4A}{psalmi-ad-adoratione-sanctam-crucem}

\subsection{Adoração da Santa Cruz II}\label{subsection:hebdomada-sancta/feria-6-in-passione-domini/ad-adoratione-sanctam-crucem-2}
\MakeChantLongPsalm{psalmi-ad-adoratione-sanctam-crucem}{crucem-tuam-adoramus.4E}{
  {psalmus-v1}{psalmus-v1-pt},
  {psalmus-v2}{psalmus-v2-pt}
}

\nobreaksubsection{Comunhão}
\begin{rubrica}
  Enquanto é trazido o Santíssimo Sacramento para o altar, todos silenciam-se. Durante a distribuição da sagrada Comunhão, pode-se cantar um cântico apropriado.
\end{rubrica}