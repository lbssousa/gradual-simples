\SetVerseAnnotation{Sl 46(45),2--11}

\SetLinkTextL{\Antiphona{Virgo María}}
\SetLinkTextR{\Antiphona{Virgem Maria}}

\SetFirstVersePair{
  {\Inchoatio{De}{us} est nobis refúgi\MediatioVII{um}{ et}{vir}[tus,] adiutórium in tribulatiónibus in\TerminatioVII{vén}{tus est }{ni}mis.}%
    {\Inchoatio{O}[ ]{Se}nhor para nós é refúgio \MediatioVII{e}{ vi}{gor}[,] sempre pronto, mostrou-se um socorro \TerminatioVII{na}{ an}{gús}tia.}
}

\SetVersePairs{
  {\VSup{3}\Inchoatio{Prop}{té}rea non timébimus, dum tur\MediatioVII{bá}{tur }{ter}[ra,] et transferéntur montes \TerminatioII{in}{ cor }{ma}ris.}%
    {\VSup{3}\Inchoatio{As}{sim} não tememos, se a terra \MediatioVII{es}{tre}{me}[ce,] se os montes desabam, ca\TerminatioVII{in}{do nos }{ma}res.},
  {\VSup{4}\Inchoatio{Fre}{mant} et intuméscant \MediatioVII{a}{quæ }{ei}[us,] conturbéntur montes in elati\TerminatioVII{ó}{ne }{e}ius.}%
    {\VSup{4}\Inchoatio{Se}[ ]{as} águas trovejam e as ondas \MediatioVII{se}{ a}{gi}[tam,] se, em feroz tempestade, as montanhas \TerminatioVII{se}{ a}{ba}lam.},
  {\VSup{5}\Inchoatio{Flú}{mi}nis rivi lætíficant civi\MediatioVII{tá}{tem }{De}[i,] sancta tabernácu\TerminatioVII{\-la}{ Al}{tís}simi.}%
    {\VSup{5}\Inchoatio{Os}[ ]{bra}ços de um rio vêm trazer \MediatioVII{a}{le}{gri}[a,] à Cidade de Deus, à morada \TerminatioVII{do}{ Al}{tís}simo.},
  {\VSup{6}\Inchoatio{De}{us} in médio eius, non \MediatioVII{com}{mo}{vé}[bitur;] adiavábit eam Deus \TerminatioVII{ma}{ne di}{lú}culo.}%
    {\VSup{6}\Inchoatio{Quem}[ ]{a} pode abalar? Deus está \MediatioVII{no}{ seu }{mei}[o!] Já bem antes da aurora, ele vem \TerminatioVII{a}{ju}{dá}-la.},
  {\VSup{7}\Inchoatio{Fre}{mu}érunt gentes, com\MediatioVII{mó}{ta sunt }{re}[gna;] dedit vocem suam, lique\TerminatioVII{fác}{ta est }{ter}ra.}%
    {\VSup{7}\Inchoatio{Os}[ ]{po}vos se agitam, os \MediatioVII{rei}{nos de}{sa}[bam;] troveja sua voz e a terra \TerminatioVII{es}{tre}{me}ce.},
  {\VSup{8}\Inchoatio{Dó}{mi}nus vir\MediatioVII{tú}{tum no}{bís}[cum,] refúgium nobis \TerminatioVII{De}{\-us }{Ia}cob.}%
    {\VSup{8}\Inchoatio{Co}{nos}co está o Senhor do \MediatioVII{u}{ni}{ver}[so!] O nosso refúgio é o \MediatioVII{Deus}{ de Ja}{có}!},
  {\VSup{9}\Inchoatio{Ve}{ní}te, et vidéte \MediatioVII{ó}{pera }{Dó}[mini] quæ pósuit prodígia \TerminatioVII{su}{per }{ter}ram.}%
    {\VSup{9}\Inchoatio{Vin}{de} ver, contemplai os pro\MediatioVII{dí}{gios de }{Deus} e a obra estupenda que fez no \TerminatioVII{u}{ni}{ver}so.},
  {\VSup{10}\Inchoatio{Áu}{fe}ret bella usque ad finem \Flexa{ter}[ræ,] arcum cónteret et con\MediatioVII{frín}{get }{ar}[ma,] et scuta com\TerminatioVII{bú}{ret }{i}gni.}%
    {\VSup{10}\Inchoatio{Re}{pri}me as guerras na face da \Flexa{ter}[ra,] ele quebra os arcos, as \MediatioVII{lan}{ças des}{trói}[;] e queima no fogo os escudos \TerminatioVII{e}{ as }{ar}mas.},
  {\VSup{11}\Inchoatio{Va}{cá}te, et vidéte quóniam \MediatioVII{e}{go sum }{De}[us;] exaltábor in géntibus, exal\TerminatioVII{tá}{bor in }{ter}ra.}%
    {\VSup{11}``\Inchoatio{Pa}{rai} e sabei, conhecei que \MediatioVII{eu}{ sou }{Deus}[,] que domino as nações, que do\TerminatioVII{mi}{no a }{ter}ra!''}
}