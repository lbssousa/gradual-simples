% !TeX root = ../fasciculo-2/a4.tex
% chktex-file 1

\subsection{Entrada I}\label{subsection:tempus-quadragesimae/feria-4-cinerum/introitus-1}
\MakeChantAntiphonPsalm{miserere-mei-deus.8G}{psalmi-ad-introitum}

\subsection{Entrada II}\label{subsection:tempus-quadragesimae/feria-4-cinerum/introitus-2}
\MakeChantAntiphonPsalm{advenerunt-nobis.8G}{psalmi-ad-introitum}

\subsection[Salmo Responsorial I]{Salmo Responsorial I \textmd{C 2 g}}\label{subsection:tempus-quadragesimae/feria-4-cinerum/psalmus-responsorius-1}
\MakeChantPsalmOneVerse{psalmi-responsorii}{miserere-mei-deus.C2g}

\AllowPageFlush

\subsection[Salmo Responsorial II]{Salmo Responsorial II \textmd{E 1}}\label{subsection:tempus-quadragesimae/feria-4-cinerum/psalmus-responsorius-2}
\MakeChantPsalmOneVerse{psalmi-responsorii}{adiuva-nos.E1}

\AllowPageFlush

\subsection{Trato}\label{subsection:tempus-quadragesimae/feria-4-cinerum/tractus}
\MakeChantPsalmOneVerse{tractus}{adiuva-nos}

\AllowPageBreak

\subsection{Bênção e imposição das cinzas}\label{subsection:tempus-quadragesimae/feria-4-cinerum/ad-benedictionem-et-impositionem-cinerum}
\MakeChantAntiphonPsalm{dele-domine.4E}{psalmi-ad-benedictionem-et-impositionem-cinerum}

\AllowPageFlush

\subsection{Ofertório}\label{subsection:tempus-quadragesimae/feria-4-cinerum/offertorium}
\MakeChantAntiphonPsalm{factus-est.1g2}{psalmi-ad-offertorium}

\AllowPageFlush

\subsection{Comunhão}\label{subsection:tempus-quadragesimae/feria-4-cinerum/communio}
\MakeChantAntiphonPsalm{da-nobis-domine.2D}{psalmi-ad-communionem}