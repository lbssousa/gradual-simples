\SetVerseAnnotation{Sl 19(18),2--7}

\def\VersePairs{
  % 2
  {\Inchoatio{Di}{es} diéi e\MediatioVII{rú}{ctat }{ver}[bum,] et nox nocti índi\TerminatioVII{cat}{ sci}{\-én}\-tiam. \Antiphona{Misit Dóminus}[\LinkLA]}%
    {\Inchoatio{O}[ ]{di}a transmite ao dia \MediatioVII{a}{ men}{sa}[gem] e a noite dá conhecimento a \TerminatioVII{ou}{tra }{noi}te. \Antiphona{O Senhor enviou}[\LinkPT]},
  % 3
  {\Inchoatio{Non}[ ]{sunt} loquélæ \MediatioVII{ne}{que ser}{mó}[nes,] quorum non intelle\TerminatioVII{gán}{tur }{vo}ces. \Antiphona{Misit Dóminus}[\LinkLA]}%
    {\Inchoatio{Não}[ ]{são} falas, \MediatioVII{nem}{ dis}{cur}[sos,] nem se ouve a \TerminatioVII{su}{a }{voz}. \Antiphona{O Senhor enviou}[\LinkPT]},
  % 4
  {\Inchoatio{In}[ ]{om}nem terram exívit \MediatioVII{so}{nus e}{ó}[rum,] et in fines orbis terræ \TerminatioVII{ver}{ba e}{ó}rum. \Antiphona{Misit Dóminus}[\LinkLA]}%
    {\Inchoatio{Por}[ ]{to}da a terra se difunde \MediatioVII{o}{ seu }{som}[,] e até os confins do mundo vai a \TerminatioVII{su}{a men}{sa}gem. \Antiphona{O Senhor enviou}[\LinkPT]},
  % 5
  {\Inchoatio{So}{li} pósuit tabernáculum in \Flexa{e}[is,] et ipse tamquam sponsus procédens de \MediatioVII{thá}{lamo }{su}[o,] exsultávit ut gigas ad cur\TerminatioVII{rén}{dam }{vi}am. \Antiphona{Misit Dóminus}[\LinkLA]}%
    {\Inchoatio{A}{li} armou uma tenda para o \Flexa{sol} e sai como um noivo do quarto \MediatioVII{nup}{ci}{al}[,] e exulta como um gigante a percorrer o \TerminatioVII{seu}{ ca}{mi}nho. \Antiphona{O Senhor enviou}[\LinkPT]},
  % 6
  {\Inchoatio{A}[ ]{fí}nibus cælórum egréssio \Flexa{e}[ius,] et occúrsus eius usque ad \MediatioVII{fi}{nes e}{ó}[rum,] nec est quod abscondátur a ca\TerminatioVII{ló}{re }{e}ius. \Antiphona{Misit Dóminus}[\LinkLA]}%
    {\Inchoatio{A}[ ]{su}a saída é desde os confins dos \Flexa{céus} e o seu percurso vai até o \MediatioVII{ou}{tro ex}{tre}[mo,] e nada pode subtrair-se ao \TerminatioVII{seu}{ ca}{lor}. \Antiphona{O Senhor enviou}[\LinkPT]}
}%