% !TeX root = ../../../a4.tex
\DeclareDocumentCommand{\Annotation}{}{
  \SetVerseAnnotation{\CantusID{002894}[Is 58,7]}{Sl 51(50),3a.11--14}
}

\def\LinkLA{tempus-quadragesimae/dominica-3/offertorium/antiphona}
\def\LinkPT{\LinkLA-pt}

\def\VersePairs{
  % 11
  {\Inchoatio{A}{vér}te fáciem tuam a pec\MediatioIV{cá}{tis}[ ]{me}[is,] et omnes iniquitá\TerminatioIV{tes}[ ]{me}{as}[ ]{de}le. \Antiphona{Frange esuriénti}[\LinkLA]}%
    {\Inchoatio{Dos}[ ]{meus} pecados, desviai \MediatioIV{vos}{sa}[ ]{fa}[ce] e apagai todas as minhas \TerminatioIV{i}{ni}{qui}{da}des. \Antiphona{Reparte com os famintos}[\LinkPT]},
  % 12
  {\Inchoatio{Cor}[ ]{mun}dum crea \MediatioIV{in}[ ]{me}[, ]{De}[us,] et spíritum firmum ínnova in vi\TerminatioIV{scé}{ri}{bus}[ ]{me}is. \Antiphona{Frange esuriénti}[\LinkLA]}%
    {\Inchoatio{Cri}{ai} em mim um coração pu\MediatioIV{ro}[, ]{ó}[ ]{Deus}[,] e renovai em minhas entranhas um espíri\TerminatioIV{to}[ ]{re}{so}{lu}to. \Antiphona{Reparte com os famintos}[\LinkPT]},
  % 13
  {\Inchoatio{Ne}[ ]{pro}ícias me a fá\TerminatioIV{ci}{e}[ ]{tu}[a,] et spíritum sanctum tuum ne \TerminatioIV{áu}{fe}{ras}[ ]{a} me. \Antiphona{Frange esuriénti}[\LinkLA]}%
    {\Inchoatio{Não}[ ]{me} rejeiteis da \MediatioIV{vos}{sa}[ ]{fa}[ce] e não retireis de mim o espírito de vos\TerminatioIV{sa}[ ]{san}{ti}{da}de. \Antiphona{Reparte com os famintos}[\LinkPT]},
  % 14
  {\Inchoatio{Red}{de} mihi lætítiam salu\MediatioIV{tá}{ris}[ ]{tu}[i,] et spíritus promptís\TerminatioIV{si}{mo}[ ]{con}{fírma} me. \Antiphona{Frange esuriénti}[\LinkLA]}%
    {\Inchoatio{De}{vol}vei-me a alegria da vossa \MediatioIV{sal}{va}{ção} e confir\-mai-me com um espíri\TerminatioIV{to}[ ]{ge}{ne}{ro}so. \Antiphona{Reparte com os famintos}[\LinkPT]}
}