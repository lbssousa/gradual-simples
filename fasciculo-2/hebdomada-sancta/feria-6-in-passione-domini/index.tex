% !TeX root = ../../main.tex
% chktex-file 1
\def\Prefix{hebdomada-sancta/feria-6-in-passione-domini/}

\subsection[Salmo Responsorial I]{Salmo Responsorial I \textmd{C 3 a}}\label{subsection:hebdomada-sancta/feria-6-in-passione-domini/psalmus-responsorius-1}
\MakeChantPsalmThreeVerses{\Prefix}{psalmus-responsorius-1/}

\subsection[Salmo Responsorial II]{Salmo Responsorial II \textmd{C 2 g}}\label{subsection:hebdomada-sancta/feria-6-in-passione-domini/psalmus-responsorius-2}
\MakeChantPsalmOneVerse{\Prefix}{psalmus-responsorius-2/}

\subsection{Apresentação da Santa Cruz}\label{subsection:hebdomada-sancta/feria-6-in-passione-domini/ad-detegendam-sanctam-crucem}
\begin{rubrica}
  O sacerdote repete o cântico a seguir três vezes, em tons ascendentes:
\end{rubrica}

\MakeChantPsalm{\Prefix}{ad-detegendam-sanctam-crucem/}

\subsection{Adoração da Santa Cruz I}\label{subsection:hebdomada-sancta/feria-6-in-passione-domini/ad-adoratione-sanctam-crucem-1}
\MakeChantAntiphonPsalm{\Prefix}{ad-adoratione-sanctam-crucem-1/}

\subsection{Adoração da Santa Cruz II}\label{subsection:hebdomada-sancta/feria-6-in-passione-domini/ad-adoratione-sanctam-crucem-2}
\begin{annotation}
  \textcolor{gregoriocolor}{\Vbar.} Sl 67(66),2
\end{annotation}

\MakeChantLongPsalm{\Prefix}{ad-adoratione-sanctam-crucem-2/}{
  {psalmus-v1}{psalmus-v1-pt},
  {psalmus-v2}{psalmus-v2-pt}
}

\nobreaksubsection{Comunhão}

\begin{rubrica}
  Enquanto é trazido o Santíssimo Sacramento para o altar, todos silenciam-se. Durante a distribuição da sagrada Comunhão, pode-se cantar um cântico apropriado.
\end{rubrica}