% !TeX root = ../../a4.tex
% chktex-file 1
\def\Prefix{hebdomada-sancta/missa-chrismatis}

\subsection{Entrada}\label{subsection:hebdomada-sancta/missa-chrismatis/introitus}
\MakeChantAntiphonPsalm{\Prefix}{introitus}

\subsection[Salmo Responsorial I]{Salmo Responsorial I \textmd{E 3}}\label{subsection:hebdomada-sancta/missa-chrismatis/psalmus-responsorius-1}
\MakeChantPsalmTwoVerses{\Prefix}{psalmus-responsorius-1}

\subsection[Salmo Responsorial II]{Salmo Responsorial II \textmd{C 2 g}}\label{subsection:hebdomada-sancta/missa-chrismatis/psalmus-responsorius-2}
\MakeChantPsalmTwoVerses{\Prefix}{psalmus-responsorius-2}

\subsection{Ofertório}\label{subsection:hebdomada-sancta/missa-chrismatis/offertorium}
\MakeChantLongPsalm{\Prefix}{offertorium}{
    {psalmus-v1}{psalmus-v1-pt},
    {psalmus-v2}{psalmus-v2-pt},
    {psalmus-v3}{psalmus-v3-pt},
    {psalmus-v4}{psalmus-v4-pt},
    {psalmus-v5}{psalmus-v5-pt},
    {psalmus-v6}{psalmus-v6-pt},
    {psalmus-v7}{psalmus-v7-pt}
}

\nottoggle{compact}{\clearpage\hbox{}\newpage}{}

\subsection{Comunhão}\label{subsection:hebdomada-sancta/missa-chrismatis/communio}
\MakeChantAntiphonPsalm{\Prefix}{communio}