\def\PsalmAnnotation{Ct 1,15.2,11–12.13b.14b.4,11–12.5,2b.6,4}

\def\VersePairs{
  % 2,11
  {Iam enim hiems \MediatioE{tráns}[iit,] imber ábiit et re\TerminatioEII{cés}sit. \Responsorium{Allelúia}[\LinkLA]}%
    {O inverno pas\MediatioE{sou}[,] as chuvas cessaram e já se \TerminatioEII{fo}ram. \Responsorium{Aleluia}[\LinkPT]},
  % 12
  {Flores apparuérunt in terra \MediatioE{no}[stra,] tempus putatiónis ad\TerminatioEII{vé}nit. \Responsorium{Allelúia}[\LinkLA]}%
    {Apareceram as flores no \MediatioE{cam}[po,] chegou o tempo da \TerminatioEII{po}\-da. \Responsorium{Aleluia}[\LinkPT]},
  % 13b
  {Vox túrturis audíta est in terra \MediatioE{no}[stra,] víneæ floréntes dedérunt odórem \TerminatioEII{su}um. \Responsorium{Allelúia}[\LinkLA]}%
    {A rola já faz ouvir seu canto em nossa \MediatioE{ter}[ra.] Soltam perfume as vinhas em \MediatioE{flor}. \Responsorium{Aleluia}[\LinkPT]},
  % 14b
  {Sonet vox tua in áuribus \MediatioE{me}[is:] vox enim tua dulcis, et fácies tua de\TerminatioEII{có}ra. \Responsorium{Allelúia}[\LinkLA]}%
    {A tua voz ressoe aos meus ou\MediatioE{vi}[dos,] pois a tua voz é suave e o teu rosto é \TerminatioEII{lin}do! \Responsorium{Aleluia}[\LinkPT]},
  % 4,11a
  {Favus distíllans lábia tua \MediatioE{spon}[sa,] mel et lac sub língua \TerminatioEII{tu}a. \Responsorium{Allelúia}[\LinkLA]}%
    {Teus lábios, minha esposa, são favo que destila o \MediatioE{mel}[;] sob a tua língua há mel e \TerminatioEII{lei}te. \Responsorium{Aleluia}[\LinkPT]},
  % 11b
  {Odor vestimentórum tu\MediatioE{ó}[rum] super ómnia a\TerminatioEII{ró}\-ma\-ta. \Responsorium{Allelúia}[\LinkLA]}%
    {O perfume de tuas \MediatioE{ves}[tes] é como o perfume do \TerminatioEII{Lí}ba\-no. \Responsorium{Aleluia}[\LinkPT]},
  % 12
  {Hortus conclúsus, soror mea, \MediatioE{spon}[sa,] hortus conclúsus, fons si\TerminatioEII{gná}tus. \Responsorium{Allelúia}[\LinkLA]}%
    {És um jardim fechado, minha irmã e es\MediatioE{po}[sa,] jardim fechado e fonte la\TerminatioEII{cra}da. \Responsorium{Aleluia}[\LinkPT]},
  % 5,2b
  {Aperi mihi, soror mea, amíca \MediatioE{me}[a,] colúmba mea, immaculáta \MediatioE{me}a. \Responsorium{Allelúia}[\LinkLA]}%
    {Abre-me, ó minha irmã e a\MediatioE{ma}[da,] minha pomba, minha imacu\MediatioE{la}da. \Responsorium{Aleluia}[\LinkPT]},
  % 6,4
  {Pulchra es, amíca \Flexa*{me}[a,] suávis et decóra sicut Ie\MediatioE{\-rú}[\-sa\-lem,] terríbilis ut castrórum ácies ordi\TerminatioEII{ná}ta. \Responsorium{Allelúia}[\LinkLA]}%
    {Tu és bela, minha a\Flexa*{ma}[da,] formosa como Jerusa\MediatioE{\-lém}[,] terrível como um exército em ordem de ba\TerminatioEII{ta}\-lha. \Responsorium{Aleluia}[\LinkPT]}
}