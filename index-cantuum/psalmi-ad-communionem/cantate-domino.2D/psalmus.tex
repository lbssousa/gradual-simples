\SetLinkTextL{\Antiphona{Cantáte Dómino}}
\SetLinkTextR{\Antiphona{Cantai ao Senhor}}

\SetVersePairs{
  {\item \VSup{3}\Inchoatio{An}{nun}tiáte inter gentes glóriam \MediatioII{e}[ius,] in ómnibus pópulis mirabíli\TerminatioII{a}[ ]{e}ius.}%
    {\item \Inchoatio{Ma}{ni}festai a sua glória entre \MediatioII{as}[ nações,] e entre os povos do universo seus \TerminatioII{pro}{dí}gios!},
  {\item \VSup{6}\Inchoatio{Ma}{gni}ficéntia et pulchritúdo in conspéctu \MediatioII{e}[ius,] poténtia et decor in sanctuári\TerminatioVII{o}{ e}ius.}%
    {\item \VSup{6}\Inchoatio{Di}{an}te dele vão a glória e a majes\MediatioII{ta}[de,] e o seu templo, que beleza \TerminatioII{e}[ ]{es}plendor!},
  {\item \VSup{7}\Inchoatio{Af}{fér}te Dómino, famíliæ popu\Flexa{ló}[rum,] afférte Dómino glóriam et po\MediatioII{tén}[tiam,] \VSup{8}afférte Dómino glóriam nómi\TerminatioII{nis}[ ]{e}ius.}%
    {\item \VSup{7}\Inchoatio{Ó}[ ]{fa}mília das nações, dai \Flexa{ao}[ Senhor,] ó nações, dai ao Senhor poder e \MediatioII{gló}[ria,] \VSup{8}dai-lhe a glória que é devida ao \TerminatioII{seu}[ ]{no}me!},
  {\item \Inchoatio{Tól}{li}te hóstias et introíte in átria \MediatioII{e}[ius,] \VSup{9}adoráte Dóminum in splendó\TerminatioII{re}{ sanc}to.}%
    {\item \Inchoatio{O}{fe}recei um sacrifício nos seus \MediatioII{á}[trios,] \VSup{9}adorai-o no esplendor da san\TerminatioII{ti}{da}de!},
  {\item \Inchoatio{Con}{tre}míscite a fácie eius, univérsa \MediatioII{ter}[ra;] \VSup{10}dícite in géntibus: ``Dóminus \TerminatioII{re}{gná}vit!''.}%
    {\item \Inchoatio{Ter}{ra} inteira, estremecei diante \MediatioII{de}[le!] \VSup{10}Publicai entre as nações: ``Rei\TerminatioII{na}[ ]{o} Senhor!''},
  {\item \Inchoatio{Et}{e}nim corréxit orbem terræ, qui non commo\MediatioII{\-vé}[\-bitur;] iudicábit pópulos in æ\TerminatioII{qui}{tá}te.}%
    {\item \Inchoatio{E}{le} firmou o universo inaba\MediatioII{lá}[vel,] e os povos ele julga com \TerminatioII{jus}{ti}ça.},
  {\item \VSup{11}\Inchoatio{Læ}{tén}tur cæli, et exsúltet \Flexa{ter}[ra,] sonet mare et plenitúdo \MediatioII{e}[ius;] \VSup{12}gaudébunt campi et ómnia, quæ \TerminatioII{in}[ ]{e}is sunt.}%
    {\item \VSup{11}\Inchoatio{O}[ ]{céu} se rejubile e exulte a \Flexa{ter}[ra,] aplauda o mar com o que vive em suas \MediatioII{á}[guas;] \VSup{12}os campos com seus frutos re\TerminatioII{ju}{bi}lem.},
  {\item \Inchoatio{Tunc}[ ]{ex}sultábunt ómnia ligna sil\Flexa{vá}[rum] \VSup{13}a fácie Dómini, quia \MediatioII{ve}[nit,] quóniam venit iudicá\TerminatioII{re}[ ]{ter}ram.}%
    {\item \Inchoatio{E}[ ]{e}xultem as florestas e as \Flexa{ma}[tas] \VSup{13}na presença do Senhor, pois \MediatioII{e}[le vem,] porque vem para julgar a terra \TerminatioII{in}{tei}ra.},
  {\item \Inchoatio{Iu}{di}cábit orbem terræ in iu\MediatioII{stí}[tia] et pópulos in veritá\TerminatioII{te}{ su}a.}%
    {\item \Inchoatio{Go}{ver}nará o mundo todo com jus\MediatioII{ti}[ça] e os povos julgará com le\TerminatioII{al}{da}de.}
}