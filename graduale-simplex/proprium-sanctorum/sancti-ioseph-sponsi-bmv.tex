% !TeX root = ../../a4.tex
% chktex-file 1
\def\Prefix{proprium-sanctorum/sancti-ioseph-sponsi-bmv/}

\subsection{Entrada}\label{subsection:proprium-sanctorum/sancti-ioseph-sponsi-bmv/introitus}
\SetAntiphonAnnotation{\CantusID{003507}[Mt 1,20]}
\def\AntiphonScore{ioseph-fili-david.7a/}
\MakeChantAntiphonPsalm*{introitus/}{\AntiphonScore}

\subsection{Salmo Responsorial \textmd{D 1 b}}\label{subsection:proprium-sanctorum/sancti-ioseph-sponsi-bmv/psalmus-responsorius}
\MakeChantPsalmOneVerse{psalmi-responsorii/}{quam-dilecta-tabernacula-tua.D1b/}

\subsection{Aleluia}\label{subsection:proprium-sanctorum/sancti-ioseph-sponsi-bmv/alleluia}
\def\AntiphonScore{alleluia.4E.1/}
\MakeChantAntiphonPsalm{alleluia/}{unum-petii-a-domino.4E/}

\AllowPageFlush

\subsection[Salmo Aleluiático]{Salmo Aleluiático \textmd{C 1}}\label{subsection:proprium-sanctorum/sancti-ioseph-sponsi-bmv/psalmus-alleluiaticus}
\MakeChantPsalmOneVerse{psalmi-alleluiatici/}{dominus-illuminatio-mea.C1/}

\AllowPageFlush

\subsection{Trato}\label{subsection:proprium-sanctorum/sancti-ioseph-sponsi-bmv/tractus}
\begin{rubrica}
  O Graduale Simplex, originalmente, não propõe um trato para esta solenidade. A proposta aqui apresentada é derivada do trato do Graduale Romanum para esta solenidade, aplicando-se a melodia do tra\-to do Graduale Simplex para o I Domingo do Tempo da Quaresma.
\end{rubrica}
\MakeChantPsalmOneVerse{tractus/}{beatus-vir-qui-timet-dominum/}

\AllowPageFlush

\subsection{Ofertório}\label{subsection:proprium-sanctorum/sancti-ioseph-sponsi-bmv/offertorium}
\SetAntiphonAnnotation{\CantusID{000000}[Lc 2,51]}
\def\AntiphonScore{descendit.8G/}
\MakeChantAntiphonPsalm*{offertorium/}{\AntiphonScore}

\AllowPageFlush

\subsection{Comunhão}\label{subsection:proprium-sanctorum/sancti-ioseph-sponsi-bmv/communio}
\SetAntiphonAnnotation{\CantusID{a02794}[Mt 24,45]}
\def\AntiphonScore{ecce-fidelis-servus-et-prudens.8G/}
\MakeChantAntiphonPsalm*{communio/}{\AntiphonScore}