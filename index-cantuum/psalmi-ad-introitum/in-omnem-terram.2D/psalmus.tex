\SetLinkTextL{\Antiphona{In omnem terram}}
\SetLinkTextR{\Antiphona{Por tod\Elisio{a} a terra}}

\SetVersePairs{
  {\VSup{3}\Inchoatio{Di}{es} diéi erúctat \MediatioII{ver}[bum,] et nox nocti índicat \TerminatioII{sci}{\-én}\-tiam.}%
    {\VSup{3}\Inchoatio{O}[ ]{di}a ao dia transmite esta men\MediatioII{sa}[gem,] a noite à noite publica esta \TerminatioII{no}{tí}cia.},
  {\VSup{4}\Inchoatio{Non}[ ]{sunt} loquélæ neque ser\MediatioII{mó}[nes,] quorum non intellegán\TerminatioII{tur}[ ]{vo}ces.}%
    {\VSup{4}\Inchoatio{Não}[ ]{são} discursos nem frases ou pa\MediatioII{la}[vras,] nem são vozes que possam ser \TerminatioII{ou}{vi}das.},
  {\VSup{6}\Inchoatio{So}{li} pósuit tabernáculum in \Flexa{e}[is,] et ipse tamquam sponsus procédens de thálamo \MediatioII{su}[o,] exsultávit ut gigas ad currén\TerminatioII{dam}[ ]{vi}am.}%
    {\VSup{6}\Inchoatio{Ar}{mou} no alto uma tenda para \MediatioII{o}[ sol;] ele desponta no céu e se \TerminatioII{le}{van}ta \Sequitur{co}mo um noivo do quarto \MediatioII{nup}[cial,] como um herói exultante em seu \TerminatioII{ca}{mi}nho.},
  {\VSup{7}\Inchoatio{A}[ ]{fí}nibus cælórum egréssio \Flexa{e}[ius,] et occúrsus eius usque ad fines e\MediatioII{ó}[rum,] nec est quod abscondátur a caló\TerminatioII{re}[ ]{e}ius.}%
    {\VSup{7}\Inchoatio{De}[ ]{um} extremo do céu põe-se a \MediatioII{cor}[rer] e vai traçando o seu raio lu\TerminatioII{mi}{no}so, \Sequitur{a}té que possa chegar ao outro ex\MediatioII{tre}[mo,] e nada pode fugir ao \TerminatioII{seu}[ ]{ca}lor.}
}