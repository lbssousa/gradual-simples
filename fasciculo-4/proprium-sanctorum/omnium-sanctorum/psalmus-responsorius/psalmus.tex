% !TeX root = ../../../main.tex
\DeclareDocumentCommand{\Annotation}{}{
  \MakeAnnotation{}{Sl 150}
}

\def\LinkLA{proprium-sanctorum/omnium-sanctorum/psalmus-responsorius/psalmus-v1}
\def\LinkPT{\LinkLA-pt}

\def\VersePairs{
  %
  {Laudáte eum in sono \MediatioC{tu}[bæ,] laudáte eum in psaltério et \TerminatioCII{cí}thara. \Responsorium{In Sanctis eius}}%
    {Louvai-o com o som da trom\MediatioC{be}[ta,] Louvai-o com a harpa e a \TerminatioCII{cí}tara. \Responsorium{Louvai a Deus}[\LinkPT]},
  %
  {\InchoatioC[Lau]{dá}te eum in týmpano et \MediatioC{cho}[ro,] laudáte eum in chordis \TerminatioCI{et}[ ]{ór}gano. \Responsorium{In Sanctis eius}}%
    {\InchoatioC[Lou]{vai}-o com o tamborim e a \MediatioC{dan}[ça,] Louvai-o com as cordas e \TerminatioCI{as}[ ]{flau}tas. \Responsorium{Louvai a Deus}[\LinkPT]},
  %
  {Laudáte eum in cýmbalis beneso\Flexa*{nán}[tibus,] laudáte eum in cýmbalis iubilati\MediatioC{ó}[nis:] omne quod spirat, laudet \TerminatioCII{Dó}\-mi\-num. \Responsorium{In Sanctis eius}}%
    {Louvai-o com os címbalos so\Flexa*{no}[ros,] louvai-o com os címbalos jubi\MediatioC{lo}[sos;] Tudo o que respira, louve o Se\TerminatioCII{nhor}. \Responsorium{Louvai a Deus}[\LinkPT]}
}