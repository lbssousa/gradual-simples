% !TeX root = ../../../main.tex
\DeclareDocumentCommand{\Annotation}{}{
  \MakeAnnotation{\CantusID{001281}[Sl 119(118),117]}{Sl 119(118),1.2.12.17.25.27.35--36.48--49}
}

\def\LinkLA{tempus-per-annum/missa-6/communio/antiphona}
\def\LinkPT{\LinkLA-pt}

\def\VersePairs{
  % 2
  {\Inchoatio{Be}{á}ti qui servant testimónia \MediatioVIII{e}[ius,] in toto corde ex\TerminatioVIII{quí}{runt}[ ]{e}um. \Antiphona{Adiúva me}[\LinkLA]}%
    {\Inchoatio{Fe}{li}zes os que guardam seus ensina\MediatioVIII{men}[tos] e de todo cora\TerminatioVIII{ção}[ ]{o}[ ]{bus}cam. \Antiphona{Sustentai-me, Senhor}[\LinkPT]},
  % 12
  {\Inchoatio{Be}{ne}díctus es, \MediatioVIII{Dó}[mine;] doce me iustificati\TerminatioVIII{ó}{nes}[ ]{tu}as. \Antiphona{Adiúva me}[\LinkLA]}%
    {\Inchoatio{Ben}{di}to sois vós, Se\MediatioVIII{nhor}[!] Ensinai-me os vossos jus\TerminatioVIII{tos}[ ]{de}{cre}tos. \Antiphona{Sustentai-me, Senhor}[\LinkPT]},
  % 17
  {\Inchoatio{Bé}{ne}fac servo tuo, et \MediatioVIII{vi}[vam,] et custódiam ser\TerminatioVIII{mó}{\-nem}[ ]{tu}um. \Antiphona{Adiúva me}[\LinkLA]}%
    {\Inchoatio{Fa}{zei} o bem a vosso servo, e vive\MediatioVIII{rei}[,] e guardarei a vos\TerminatioVIII{sa}[ ]{pa}{la}vra. \Antiphona{Sustentai-me, Senhor}[\LinkPT]},
  % 25
  {\Inchoatio{Ad}{hǽ}sit púlveri ánima \MediatioVIII{me}[a;] vivifíca me secúndum \TerminatioVIII{ver}{\-bum}[ ]{tu}um. \Antiphona{Adiúva me}[\LinkLA]}%
    {\Inchoatio{Mi}{nha} alma está grudada ao \MediatioVIII{pó}[;] segundo a vossa palavra, fazei-\TerminatioVIII{me}[ ]{vi}{ver}. \Antiphona{Sustentai-me, Senhor}[\LinkPT]},
  % 27
  {\Inchoatio{Vi}{am} mandatórum tuórum fac me intel\MediatioVIII{lé}[gere,] et exercébor in mirabí\TerminatioVIII{li}{bus}[ ]{tu}is. \Antiphona{Adiúva me}[\LinkLA]}%
    {\Inchoatio{Fa}{zei}-me entender o caminho dos vossos pre\MediatioVIII{cei}[tos,] e meditarei nas vossas \TerminatioVIII{ma}{ra}{vi}lhas. \Antiphona{Sustentai-me, Senhor}[\LinkPT]},
  % 35
  {\Inchoatio{De}{duc} me in sémitam præceptórum tu\MediatioVIII{ó}[rum,] quia \TerminatioVIII{ip}{sam}[ ]{vo}lui. \Antiphona{Adiúva me}[\LinkLA]}%
    {\Inchoatio{Gui}{ai}-me pela trilha dos vossos manda\MediatioVIII{men}[tos,] pois nela te\TerminatioVIII{nho}[ ]{pra}{zer}. \Antiphona{Sustentai-me, Senhor}[\LinkPT]},
  % 36
  {\Inchoatio{In}{clí}na cor meum in testimónia \MediatioVIII{tu}[a,] et non in \TerminatioVIII{a}{va}{rí}tiam. \Antiphona{Adiúva me}[\LinkLA]}%
    {\Inchoatio{In}{cli}nai meu coração para os vossos ensina\MediatioVIII{men}[tos,] e não para a \TerminatioVIII{a}{va}{re}za. \Antiphona{Sustentai-me, Senhor}[\LinkPT]},
  % 48
  {\Inchoatio{Et}[ ]{le}vábo manus meas ad præcépta tua, quæ di\MediatioVIII{lé}[xi;] et exercébor in iustificatió\TerminatioVIII{ni}{bus}[ ]{tu}is. \Antiphona{Adiúva me}[\LinkLA]}%
    {\Inchoatio{Le}{van}tarei minhas mãos para os vossos manda\MediatioVIII{men}[\-tos] \TerminatioVIII{que}[ ]{eu}[ ]{a}mo. \Antiphona{Sustentai-me, Senhor}[\LinkPT]},
  % 49
  {\Inchoatio{Me}{mor} esto verbi tui servo \MediatioVIII{tu}[o,] in quo mihi \allowbreak\TerminatioVIII{spem}[ ]{de}{dí}sti. \Antiphona{Adiúva me}[\LinkLA]}%
    {\Inchoatio{Lem}{brai}-vos da vossa palavra ao vosso \MediatioVIII{ser}[vo,] pela qual me destes \TerminatioVIII{es}{pe}{ran}ça. \Antiphona{Sustentai-me, Senhor}[\LinkPT]}
}