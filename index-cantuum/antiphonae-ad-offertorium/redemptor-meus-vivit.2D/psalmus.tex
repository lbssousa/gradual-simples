% !TeX root = ../../../a4.tex
\DeclareDocumentCommand{\Annotation}{}{
    \MakeAnnotation{\CantusID[?]{004588}[Jó 19,25]}{Sl 18(17),1.6--7 \\ Adaptado originalmente por Lincoln Haas Hein}
}

\def\LinkLA{liturgia-defunctorum/missa-pro-defunctis/offertorium-1/antiphona}
\def\LinkPT{\LinkLA-pt}

\def\VersePairs{
    %
    {\Inchoatio{Fu}{nes} inférni circumde\MediatioII{dé}[runt me,] præoccupavérunt me láque\TerminatioII{i}[ ]{mor}tis. \Antiphona{Redémptor meus vivit}[\LinkLA]}
        {\Inchoatio{Cor}{das} dos infernos me amar\MediatioII{ra}[ram,] surpreende\-ram-me os laços \TerminatioII{da}[ ]{Mor}te. \Antiphona{Meu redentor vive}[\LinkPT]},
    % 
    {\Inchoatio{In}[ ]{tri}bulatióne mea invocávi \MediatioII{Dó}[minum,] et ad Deum meum \TerminatioII{cla}{má}vi. \Antiphona{Redémptor meus vivit}[\LinkLA]}
        {\Inchoatio{Na}[ ]{mi}nha angústia invoquei o Se\MediatioII{nhor}[,] e clamei ao \TerminatioII{meu}[ ]{Deus}. \Antiphona{Meu redentor vive}[\LinkPT]},
    %
    {\Inchoatio{Ex}{au}dívit de templo suo vocem \MediatioII{me}[am,] et clamor meus in conspéctu eius introívit in au\TerminatioII{res}[ ]{e}ius. \Antiphona{Redémptor meus vivit}[\LinkLA]}
        {\Inchoatio{Do}[ ]{seu} Templo, ele ouviu a minha \MediatioII{voz} e o meu clamor na sua presença entrou em seus \TerminatioII{ou}{vi}dos. \Antiphona{Meu redentor vive}[\LinkPT]}
}