% !TeX root = ../../../main.tex
\DeclareDocumentCommand{\Annotation}{}{
  \MakeAnnotation{\CantusID{a02794}[Mt 24,45]}{Sl 21(20),2--8.14}
}

\def\LinkLA{proprium-sanctorum/in-praesentatione-domini/communio/antiphona}
\def\LinkPT{\LinkLA-pt}

\def\VersePairs{
  % 3
  {\Inchoatio{De}{si}dérium cordis eius tribuísti \MediatioVIII{e}[i,] et voluntátem labiórum eius non \TerminatioVIII{de}{ne}{gá}sti. \Antiphona{Ecce fidélis servus et prudens}[\LinkLA]}%
    {\Inchoatio{Re}{a}lizastes o desejo do seu cora\MediatioVIII{ção} e não lhe recusastes o pedido \TerminatioVIII{de}[ ]{seus}[ ]{lá}bios. \Antiphona{Este é o servo fiel e prudente}[\LinkPT]},
  % 4
  {\Inchoatio{Quó}{ni}am prævenísti eum in benedictiónibus dul\MediatioVIII{cé}[\-dinis;] posuísti in cápite eius corónam de au\TerminatioVIII{ro}[ ]{pu}{\-rís}simo. \Antiphona{Ecce fidélis servus et prudens}[\LinkLA]}%
    {\Inchoatio{Vi}{es}tes ao encontro dele com bênçãos gene\MediatioVIII{ro}[sas;] pusestes em sua cabeça uma coroa de ou\TerminatioVIII{ro}[ ]{pu}{rí}ssimo. \Antiphona{Este é o servo fiel e prudente}[\LinkPT]},
  % 5
  {\Inchoatio{Vi}{tam} pétiit a te, et tribuísti \MediatioVIII{e}[i,] longitúdinem diérum in sǽculum et in sǽ\TerminatioVIII{cu}{lum}[ ]{sǽ}culi. \Antiphona{Ecce fidélis servus et prudens}[\LinkLA]}%
    {\Inchoatio{E}{le} pediu-vos vida, e vós lhe conce\MediatioVIII{des}[tes] abundância de dias, sempre e \TerminatioVIII{pa}{ra}[ ]{sem}pre. \Antiphona{Este é o servo fiel e prudente}[\LinkPT]},
  % 6
  {\Inchoatio{Ma}{gna} est glória eius in salutári \MediatioVIII{tu}[o,] magnificéntiam et decórem impónes \TerminatioVIII{su}{per}[ ]{e}um. \Antiphona{Ecce fidélis servus et prudens}[\LinkLA]}%
    {\Inchoatio{Gran}{de} é a sua glória pela vossa salva\MediatioVIII{ção}[,] e vós o cobrireis de magnificência e \TerminatioVIII{es}{plen}{dor}. \Antiphona{Este é o servo fiel e prudente}[\LinkPT]},
  % 7
  {\Inchoatio{Quó}{ni}am pones eum benedictiónem in sǽculum \MediatioVIII{sǽ}[culi,] lætificábis eum in gáudio ante \TerminatioVIII{vul}{tum}[ ]{tu}\-um. \Antiphona{Ecce fidélis servus et prudens}[\LinkLA]}%
    {\Inchoatio{Pois}[ ]{fa}reis dele uma bênção para \MediatioVIII{sem}[pre] e o alegrareis com exultação na vos\TerminatioVIII{sa}[ ]{pre}{sen}ça. \Antiphona{Este é o servo fiel e prudente}[\LinkPT]},
  % 8
  {\Inchoatio{Quó}{ni}am rex sperat in \MediatioVIII{Dó}[mino] et in misericórdia Altíssimi non \TerminatioVIII{com}{mo}{vé}bitur. \Antiphona{Ecce fidélis servus et prudens}[\LinkLA]}%
    {\Inchoatio{O}[ ]{rei} espera no Se\MediatioVIII{nhor}[,] e graças à misericórdia do Altíssimo não será \TerminatioVIII{a}{ba}{la}do. \Antiphona{Este é o servo fiel e prudente}[\LinkPT]},
  % 14
  {\Inchoatio{Ex}{al}táre, Dómine, in virtúte \MediatioVIII{tu}[a;] cantábimus et psallémus vir\TerminatioVIII{tú}{tes}[ ]{tu}as. \Antiphona{Ecce fidélis servus et prudens}[\LinkLA]}%
    {\Inchoatio{Le}{van}tai-vos, Senhor, com vossa \MediatioVIII{for}[ça;] cantaremos e salmodiaremos ao vos\TerminatioVIII{so}[ ]{po}{der}. \Antiphona{Este é o servo fiel e prudente}[\LinkPT]}
}