% !TeX root = ../../main.tex
% chktex-file 1
\def\Prefix{proprium-sanctorum/sancti-ioseph-sponsi-bmv/}

\subsection{Entrada}\label{subsection:proprium-sanctorum/sancti-ioseph-sponsi-bmv/introitus}
\MakeChantAntiphonPsalm{\Prefix}{introitus/}

\subsection{Salmo Responsorial \textmd{D 1 b}}\label{subsection:proprium-sanctorum/sancti-ioseph-sponsi-bmv/psalmus-responsorius}
\MakeChantPsalmOneVerse{\Prefix}{psalmus-responsorius/}

\subsection{Aleluia}\label{subsection:proprium-sanctorum/sancti-ioseph-sponsi-bmv/alleluia}
\MakeChantAntiphonPsalm{\Prefix}{alleluia/}

\AllowPageFlush

\subsection[Salmo Aleluiático]{Salmo Aleluiático \textmd{C 1}}\label{subsection:proprium-sanctorum/sancti-ioseph-sponsi-bmv/psalmus-alleluiaticus}
\MakeChantPsalmOneVerse{\Prefix}{psalmus-alleluiaticus/}

\AllowPageFlush

\subsection{Trato}\label{subsection:proprium-sanctorum/sancti-ioseph-sponsi-bmv/tractus}
\begin{rubrica}
  O Graduale Simplex, originalmente, não propõe um trato para esta solenidade. A proposta aqui apresentada é derivada do trato do Graduale Romanum para esta solenidade, aplicando-se a melodia do tra\-to do Graduale Simplex para o I Domingo do Tempo da Quaresma.
\end{rubrica}
\MakeChantPsalmOneVerse{\Prefix}{tractus/}

\AllowPageFlush

\subsection{Ofertório}\label{subsection:proprium-sanctorum/sancti-ioseph-sponsi-bmv/offertorium}
\MakeChantAntiphonPsalm{\Prefix}{offertorium/}

\AllowPageFlush

\subsection{Comunhão}\label{subsection:proprium-sanctorum/sancti-ioseph-sponsi-bmv/communio}
\MakeChantAntiphonPsalm{\Prefix}{communio/}