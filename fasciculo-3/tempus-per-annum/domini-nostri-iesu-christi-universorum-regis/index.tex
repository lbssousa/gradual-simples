% !TeX root = ../../main.tex
% chktex-file 1
\def\Prefix{tempus-per-annum/domini-nostri-iesu-christi-universorum-regis/}

\subsection{Entrada}\label{subsection:tempus-per-annum/domini-nostri-iesu-christi-universorum-regis/introitus}
\MakeChantAntiphonPsalm{\Prefix}{introitus/}

\AllowPageFlush

\subsection[Salmo Responsorial]{Salmo Responsorial \textmd{E 5}}\label{subsection:tempus-per-annum/domini-nostri-iesu-christi-universorum-regis/psalmus-responsorius}
\MakeChantPsalmTwoVerses{\Prefix}{psalmus-responsorius/}

\AllowPageFlush

\subsection{Aleluia}
\MakeChantAntiphonPsalm{\Prefix}{alleluia/}

\AllowPageFlush

\subsection[Salmo Aleluiático]{Salmo Aleluiático \textmd{C 4}}\label{subsection:tempus-per-annum/domini-nostri-iesu-christi-universorum-regis/psalmus-alleluiaticus}
\begin{center}
    \begin{rubrica}
      O primeiro {\normalfont\Rbar} pode ser cantado apenas pelo grupo de cantores ou por todos. O segundo {\normalfont\Rbar} é cantado por todos.
    \end{rubrica}
  \end{center}
\MakeChantPsalmOneVerse{\Prefix}{psalmus-alleluiaticus/}

\AllowPageFlush

\subsection{Ofertório}\label{subsection:tempus-per-annum/domini-nostri-iesu-christi-universorum-regis/offertorium}
\MakeChantAntiphonPsalm{\Prefix}{offertorium/}

\AllowPageFlush

\subsection{Comunhão}\label{subsection:tempus-per-annum/domini-nostri-iesu-christi-universorum-regis/communio}
\MakeChantAntiphonPsalm{\Prefix}{communio/}