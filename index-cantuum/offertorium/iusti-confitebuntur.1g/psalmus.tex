\SetVerseAnnotation{Sl 119(118),1.17.23.46}

\def\VersePairs{
  {\Inchoatio{Bé}{ne}fac servo \MediatioI{tu}{o, et }{vi}[vam,] et custódiam ser\TerminatioI{mó}{\-nes}[ ]{tu}os. \Antiphona{Iusti}[\LinkLA]}%
    {\Inchoatio{Fa}{zei} o bem a vosso servo, e \MediatioI{vi}{ve}{rei} e guardarei a vos\TerminatioI{sa}[ ]{pa}{la}vra. \Antiphona{Os justos}[\LinkPT]},
  %
  {\Inchoatio{Et}{si} príncipes sedent et advérsum \MediatioI{me}{ lo}{quún}[tur,] servus tamen tuus exercétur in iustificatió\TerminatioI{ni}{bus}[ ]{tu}is. \Antiphona{Iusti}[\LinkLA]}%
    {\Inchoatio{Em}{bo}ra os príncipes no tribunal falem \MediatioI{con}{tra }{mim}[,] o vosso servo medita nos vossos jus\TerminatioI{tos}[ ]{de}{cre}tos. \Antiphona{Os justos}[\LinkPT]},
  %
  {\Inchoatio{Et}[ ]{lo}quar de testimóniis tuis in con\MediatioI{spéc}{tu }{re}[gum,] et \TerminatioI{non}[ ]{con}{fún}dar. \Antiphona{Iusti}[\LinkLA]}%
    {\Inchoatio{Fa}{la}rei dos vossos ensinamentos di\MediatioI{an}{te dos }{reis}[,] e não me enver\TerminatioI{go}{nha}{rei}. \Antiphona{Os justos}[\LinkPT]}
}