\SetVerseAnnotation{Sl 19,2--7(18A)}

\SetLinkTextL{\Antiphona{Roráte cæli désuper}}
\SetLinkTextR{\Antiphona{Orvalhai, ó céus, do alto}}

\SetVersePairs{
  % 3
  {\Inchoatio{Di}{es} diéi e\MediatioIV{rúc}{tat}[ ]{ver}[bum,] et nox nocti ín\TerminatioIV{di}{cat}[ ]{sci}{én}\-tiam.}%
    {\Inchoatio{O}[ ]{di}a transmite ao dia \MediatioIV{a}[ ]{men}{sa}[gem] e a noite dá conhecimento \TerminatioIV{a}[ ]{ou}{tra}[ ]{noi}te.},
  % 4
  {\Inchoatio{Non}[ ]{sunt} loquélæ ne\MediatioIV{que}[ ]{ser}{mó}[nes] quorum non intel\TerminatioIV{le}{gán}{tur}[ ]{vo}ces.}%
    {\Inchoatio{Não}[ ]{são} falas, \MediatioIV{nem}[ ]{dis}{cur}[sos,] nem se ouve \TerminatioIV{a}[ ]{sua}[ ]{voz}.},
  % 5
  {\Inchoatio{In}[ ]{om}nem terram exívit so\MediatioIV{nus}[ ]{e}{ó}[rum,] et in fines orbis terræ \TerminatioIV{ver}{ba}[ ]{e}{ó}rum.}%
    {\Inchoatio{Por to}da a terra se difunde \MediatioIV{o}[ ]{seu}[ ]{som}[,] e até os confins do mundo vai \TerminatioIV{su}{a}[ ]{men}{sa}gem.},
  % 6
  {\Inchoatio{So}{li} pósuit tabernáculum in \Flexa{e}[is,] et ipse tamquam sponsus procédens de thá\MediatioIV{la}{mo}[ ]{su}[o,] exsultávit ut gigas ad cur\TerminatioIV{rén}{dam}[ ]{vi}am.}%
    {\Inchoatio{A}{li} armou uma tenda para o \Flexa{sol} e sai como um noivo do quarto \MediatioIV{nup}{ci}{al}[,] e exulta como um gigante a percorrer \TerminatioIV{o}[ ]{seu}[ ]{ca}{mi}nho.},
  % 7
  {\Inchoatio{A}[ ]{fí}nibus cælórum egréssio \Flexa{e}[ius,] et occúrsus eius usque ad fi\MediatioIV{nes}[ ]{e}{ó}[rum,] nec est quod abscondátur a \TerminatioIV{ca}{ló}{re}[ ]{e}ius.}%
    {\Inchoatio{A}[ ]{su}a saída é desde os confins dos \Flexa{céus} e o seu percurso vai até o ou\MediatioIV{tro}[ ]{ex}{tre}[mo,] e nada pode subtrair-se \TerminatioIV{ao}[ ]{seu}[ ]{ca}{lor}.}
}