% !TeX root = ../../main.tex
% chktex-file 1
\def\Prefix{proprium-sanctorum/in-annuntiatione-domini/}

\subsection{Entrada}\label{subsection:proprium-sanctorum/in-annuntiatione-domini/introitus}
\MakeChantAntiphonPsalm{\Prefix}{introitus/}

\nobreaksubsection{Salmo Responsorial}

\begin{rubrica}
  Ver Solenidade da Imaculada Conceição de Nossa Senhora, página~\pageref{subsection:proprium-sanctorum/in-conceptione-immaculata-bmv/psalmus-responsorius}
\end{rubrica}

\nobreaksubsection{Aleluia}

\begin{rubrica}
  Ver Solenidade da Imaculada Conceição de Nossa Senhora, página~\pageref{subsection:proprium-sanctorum/in-conceptione-immaculata-bmv/alleluia}
\end{rubrica}

\nobreaksubsection{Salmo Aleluiático}

\begin{rubrica}
  Ver Solenidade da Imaculada Conceição de Nossa Senhora, página~\pageref{subsection:proprium-sanctorum/in-conceptione-immaculata-bmv/psalmus-alleluiaticus}
\end{rubrica}

%\subsection{Trato}\label{subsection:proprium-sanctorum/in-annuntiatione-domini/tractus}
%\begin{rubrica}
%  O Graduale Simplex, originalmente, não propõe um trato para esta solenidade. %A proposta aqui apresentada é derivada do trato do Graduale Romanum para esta %solenidade, aplicando-se a melodia do tra\-to do Graduale Simplex para o I %Domingo do Tempo da Quaresma.
%\end{rubrica}
%\MakeChantPsalmOneVerse{\Prefix}{tractus/}

\AllowPageFlush

\subsection{Ofertório}\label{subsection:proprium-sanctorum/in-annuntiatione-domini/offertorium}
\MakeChantAntiphonPsalm{\Prefix}{offertorium/}

\subsection{Comunhão}\label{subsection:proprium-sanctorum/in-annuntiatione-domini/communio}
\MakeChantAntiphonPsalm{\Prefix}{communio/}