% !TeX root = ../../main.tex
% chktex-file 1
\def\Prefix{tempus-quadragesimae/feria-iv-cinerum/}

\subsection{Entrada I}\label{subsection:tempus-quadragesimae/feria-iv-cinerum/introitus-1}
\MakeChantAntiphonPsalm{\Prefix}{introitus-1/}

\subsection{Entrada II}\label{subsection:tempus-quadragesimae/feria-iv-cinerum/introitus-2}
\MakeChantAntiphonPsalm{\Prefix}{introitus-2/}

\subsection[Salmo Responsorial I]{Salmo Responsorial I \textmd{C 2 g}}\label{subsection:tempus-quadragesimae/feria-iv-cinerum/psalmus-responsorius-1}
\MakeChantPsalmOneVerse{\Prefix}{psalmus-responsorius-1/}

\subsection[Salmo Responsorial II]{Salmo Responsorial II \textmd{E 1}}\label{subsection:tempus-quadragesimae/feria-iv-cinerum/psalmus-responsorius-2}
\MakeChantPsalmOneVerse{\Prefix}{psalmus-responsorius-2/}

\subsection{Trato}\label{subsection:tempus-quadragesimae/feria-iv-cinerum/tractus}
\MakeChantPsalmOneVerse{\Prefix}{tractus/}

%\nottoggle{compact}{\clearpage\hbox{}\newpage}{}

\subsection{Bênção e imposição das cinzas}\label{subsection:tempus-quadragesimae/feria-iv-cinerum/ad-benedictionem-et-impositionem-cinerum}
\MakeChantAntiphonPsalm{\Prefix}{ad-benedictionem-et-impositionem-cinerum/}

\subsection{Ofertório}\label{subsection:tempus-quadragesimae/feria-iv-cinerum/offertorium}
\MakeChantAntiphonPsalm{\Prefix}{offertorium/}

\subsection{Comunhão}\label{subsection:tempus-quadragesimae/feria-iv-cinerum/communio}
\MakeChantAntiphonPsalm{\Prefix}{communio/}