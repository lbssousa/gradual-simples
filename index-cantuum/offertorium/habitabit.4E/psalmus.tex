\SetVerseAnnotation{Sl 15(14),2--3.5ab}

\def\VersePairs{
  %
  {\Inchoatio{Qui}[ ]{ló}quitur veritátem in \MediatioIV{cor}{de}[ ]{su}[o,] qui non egit dolum \TerminatioIV{in}[ ]{lin}{gua}[ ]{su}a. \Antiphona{Habitábit}[\LinkLA]}%
    {\Inchoatio{Quem}[ ]{fa}la a verdade que está no \MediatioIV{co}{ra}{ção} e não provoca engano com \TerminatioIV{a}[ ]{su}{a}[ ]{lín}gua. \Antiphona{Habitará}[\LinkPT]},
  %
  {\Inchoatio{Nec}[ ]{fe}cit próximo \MediatioIV{su}{o}[ ]{ma}[lum,] et ppróbrium non íntulit \TerminatioIV{pró}{xi}{mo}[ ]{su}o. \Antiphona{Habitábit}[\LinkLA]}%
    {\Inchoatio{Quem}[ ]{não} faz o mal \MediatioIV{a}[ ]{seu}[ ]{pró}[ximo,] e não insul\TerminatioIV{ta}[ ]{seu}[ ]{vi}{zi}nho. \Antiphona{Habitará}[\LinkPT]},
  %
  {\Inchoatio{Qui}[ ]{fa}cit hæc,\MediatioIV{}{}{} non movébi\TerminatioIV{tur}[ ]{in}[ ]{æ}{tér}num. \Antiphona{Habitábit}[\LinkLA]}%
    {\Inchoatio{Quem}[ ]{faz} \MediatioIV{es}{tas}[ ]{coi}[sas,] jamais se\TerminatioIV{rá}[ ]{a}{ba}{la}do. \Antiphona{Habitará}[\LinkPT]}
}