\SetVerseAnnotation{Sl 23(22)}

\def\VersePairs{
  %
  {\Inchoatio{Su}{per} aquas qui\MediatioI{é}{tis e}{dú}[xit me,] ánimam me\TerminatioI{am}[ ]{re}{fé}cit. \Antiphona{Ego sum panis vivus}[\LinkLA]}%
    {\Inchoatio{Pa}{ra} águas repousantes \MediatioI{me}{ con}{duz} e reconforta \TerminatioI{mi}{nha}[ ]{al}ma. \Antiphona{Eu sou o pão vivo}[\LinkPT]},
  %
  {\Inchoatio{De}{dú}xit me super sémi\MediatioI{tas}{ iu}{stí}[tiæ] propter \TerminatioI{no}{men}[ ]{su}um. \Antiphona{Ego sum panis vivus}[\LinkLA]}%
    {\Inchoatio{Por}[ ]{ve}redas de jus\MediatioI{ti}{ça me }{gui}[a,] por amor \TerminatioI{do}[ ]{seu}[ ]{no}me. \Antiphona{Eu sou o pão vivo}[\LinkPT]},
  %
  {\Inchoatio{Nam}[ ]{et} si ambulavéro in valle umbræ mortis, non ti\MediatioI{mé}{bo }{ma}[la,] quóni\TerminatioI{am}[ ]{tu}[ ]{me}cum es. \Antiphona{Ego sum panis vivus}[\LinkLA]}%
    {\Inchoatio{Mes}{mo} se eu tiver de andar por um vale de sombra mortal, não teme\MediatioI{rei}{ os }{ma}[les,] porque es\TerminatioI{tais}[ ]{co}{mi}go. \Antiphona{Eu sou o pão vivo}[\LinkPT]},
  %
  {\Inchoatio{Vir}{ga} tua et \MediatioI{bá}{culus }{tu}[us] ipsa me \TerminatioI{con}{so}{lá}ta sunt. \Antiphona{Ego sum panis vivus}[\LinkLA]}%
    {\Inchoatio{O}[ ]{vos}so bordão e o \MediatioI{vos}{so ca}{ja}[do,] são eles que \TerminatioI{me}[ ]{con}{for}\-tam. \Antiphona{Eu sou o pão vivo}[\LinkPT]},
  %
  {\Inchoatio{Pa}{rá}sti in conspéctu \MediatioI{me}{o }{men}[sam] advérsus eos qui \TerminatioI{trí}{bu}{lant} me. \Antiphona{Ego sum panis vivus}[\LinkLA]}%
    {\Inchoatio{Di}{an}te de mim prepa\MediatioI{rais}{ a }{me}[sa] em frente dos que \TerminatioI{me}[ ]{a}{fli}gem. \Antiphona{Eu sou o pão vivo}[\LinkPT]},
  %
  {\Inchoatio{Im}{pin}guásti in óleo \MediatioI{ca}{put }{me}[um,] et calix me\TerminatioI{us}[ ]{re}{dún}dat. \Antiphona{Ego sum panis vivus}[\LinkLA]}%
    {\Inchoatio{Com}[ ]{ó}leo me un\MediatioI{gis}{ a ca}{be}[ça] e meu cáli\TerminatioI{ce}[ ]{trans}{bor}\-da. \Antiphona{Eu sou o pão vivo}[\LinkPT]},
  %
  {\Inchoatio{É}{te}nim benígnitas et misericórdia \MediatioI{sub}{se}{quén}[tur me] ómnibus diébus \TerminatioI{vi}{tæ}[ ]{me}æ. \Antiphona{Ego sum panis vivus}[\LinkLA]}%
    {\Inchoatio{Pois}[ ]{a} bondade e a misericórdia me \TerminatioI{se}{gui}{rão} todos os dias de \TerminatioI{mi}{nha}[ ]{vi}da. \Antiphona{Eu sou o pão vivo}[\LinkPT]},
  %
  {\Inchoatio{Et}[ ]{in}habitábo in \MediatioI{do}{mo }{Dó}[mini] in longitúdi\TerminatioI{nem}[ ]{di}{é}rum. \Antiphona{Ego sum panis vivus}[\LinkLA]}%
    {\Inchoatio{E}[ ]{ha}bitarei na casa \MediatioI{do}{ Se}{nhor} por di\TerminatioI{as}[ ]{sem}[ ]{fim}. \Antiphona{Eu sou o pão vivo}
      [\LinkPT]}
}