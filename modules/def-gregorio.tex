% !TeX root = ./a4.tex
% chktex-file 1
\gresetbarspacing{old}
\grechangedim{baselineskip}{66pt plus 5pt minus 5pt}{scalable}
\grechangedim{commentaryraise}{1cm}{scalable}
\grechangestyle{initial}{\fontsize{40}{40}\selectfont\color{gregoriocolor}}
\grechangestyle{abovelinestext}{\fontsize{9}{9}\selectfont\bfseries}
\grechangedim{abovelinestextraise}{0.1cm}{scalable}
\gresetheadercapture{commentary}{grecommentary}{string}
\gresetheadercapture{phantom-commentary}{grephantomcommentary}{string}

\newcommand{\grephantomcommentary}[1]{\grecommentary{\phantom{#1}}}

\newcommand{\VSup}[1]{\(^{\mathrm{#1}}\)}

%\setlength{\columnseprule}{0.4pt}

\ExplSyntaxOn
\DeclareExpandableDocumentCommand{\IfNoValueOrEmptyTF}{m m m}{%
  \IfNoValueTF{#1}%
  {#2}% true
  {\tl_if_empty:nTF{#1}{#2}{#3}}% false
}
\ExplSyntaxOff

\NewDocumentEnvironment{rubrica}{}{%
  \color{gregoriocolor}\slshape%
}{}
\NewDocumentEnvironment{greenumerate}{}{%
  \begin{enumerate}[label=\textcolor{gregoriocolor}{\protect\Vbar. \arabic*.}]%
    }{%
  \end{enumerate}%
}

\NewDocumentCommand{\Antiphona}{s m}{%
  \allowbreak\textcolor{gregoriocolor}{\Abar.}~#2\IfBooleanF{#1}{.}%
}

\NewDocumentCommand{\Responsorium}{s m}{%
  \allowbreak\textcolor{gregoriocolor}{\Rbar.}~#2\IfBooleanF{#1}{.}%
}

\NewDocumentCommand{\ResponsoriumC}{m}{%
  \allowbreak\textcolor{gregoriocolor}{\Rbar.}~#1~\GreStar%
}

\NewDocumentCommand{\Versum}{s m}{%
  \allowbreak\textcolor{gregoriocolor}{\Vbar.}~#2\IfBooleanF{#1}{.}%
}

\NewDocumentEnvironment{annotation}{}{\begin{flushright}\iftoggle{compact}{\scriptsize}{\footnotesize}}{\end{flushright}}

\DeclareDocumentCommand{\Annotation}{}{}

\newcommand{\MakeAntiphonAnnotation}[1]{\ifdefempty{#1}{}{\Antiphona*{#1}}}
\newcommand{\MakeVerseAnnotation}[1]{\ifdefempty{#1}{}{\Versum*{#1}}}

\NewDocumentCommand{\Elisio}{m}{\textit{\small#1}}

\NewDocumentCommand{\CantusID}{o m o}{%
  Cantus ID \href{http://cantusindex.org/id/#2}{%
    #2%
    \IfValueT{#1}{(#1)}%
  }
  \IfValueT{#3}{\textit{cf.} #3}%
}

\NewDocumentCommand{\Inchoatio}{m o m}{%
  \IfNoValueTF{#2}{%
    \textit{#1#3}%
  }{%
    \textit{#1}#2\textit{#3}%
  }%
}

\NewDocumentCommand{\Flexa}{s m o}{%
  \IfBooleanTF{#1}
  {\IfValueTF{#3}{#2#3}{#2}~{\large'}}
  {\IfValueTF{#3}{\textbf{#2}#3}{\textbf{#2}}~\GreDagger}%
}

\NewDocumentCommand{\Sequitur}{m}{%
  \textcolor{gregoriocolor}{\gretilde}~#1%
}

\NewDocumentCommand{\MediatioNil}{m}{#1~\GreStar}

\NewDocumentCommand{\MediatioI}{m m m O{}}{%
  \textbf{#1}#2\textbf{#3}#4~\GreStar%
}
\NewDocumentCommand{\TerminatioI}{m O{} m O{} m}{%
  \IfNoValueTF{#2}{%
    \textit{#1#3}%
  }{%
    \textit{#1}#2\textit{#3}%
  }%
  #4\textbf{#5}%
}

\NewDocumentCommand{\MediatioII}{m O{}}{%
  \textbf{#1}#2~\GreStar%
}
\NewDocumentCommand{\TerminatioII}{m O{} m}{%
  \textit{#1}#2\textbf{#3}%
}

\NewDocumentCommand{\MediatioIII}{m m m O{}}{%
  \textbf{#1}#2\textbf{#3}#4~\GreStar%
}
\NewDocumentCommand{\TerminatioIIIa}{m O{} m}{%
  \textit{#1}#2\textbf{#3}%
}
\NewDocumentCommand{\TerminatioIIIg}{m o m O{} m}{%
  \TerminatioI{#1}[#2]{#3}[#4]{#5}%
}

\NewDocumentCommand{\MediatioIV}{m o m O{} m O{}}{%
  \IfNoValueTF{#2}{%
    \textit{#1#3}%
  }{%
    \textit{#1}#2\textit{#3}%
  }%
  #4\textbf{#5}#6~\GreStar%
}
\NewDocumentCommand{\TerminatioIV}{m o m o m O{} m}{%
  \IfNoValueTF{#2}{%
    \IfNoValueTF{#4}{%
      \textit{#1#3#5}%
    }{%
      \textit{#1#3}#4\textit{#5}%
    }%
  }{%
    \IfNoValueTF{#4}{%
      \textit{#1}#2\textit{#3#5}%
    }{%
      \textit{#1}#2\textit{#3}#4\textit{#5}%
    }%
  }%
  #6\textbf{#7}%
}
\NewDocumentCommand{\TerminatioIVc}{m}{%
  \textbf{#1}%
}

\NewDocumentCommand{\MediatioV}{m O{}}{%
  \MediatioII{#1}[#2]%
}
\NewDocumentCommand{\TerminatioV}{m m m}{%
  \textbf{#1}#2\textbf{#3}%
}

\NewDocumentCommand{\MediatioVI}{m O{} m O{}}{%
  \textit{#1}#2\textbf{#3}#4~\GreStar%
}
\NewDocumentCommand{\TerminatioVI}{m o m O{} m}{%
  \TerminatioI{#1}[#2]{#3}[#4]{#5}%
}

\NewDocumentCommand{\MediatioVII}{m m m O{}}{%
  \textbf{#1}#2\textbf{#3}#4~\GreStar%
}
\NewDocumentCommand{\TerminatioVII}{m m m}{%
  \TerminatioV{#1}{#2}{#3}%
}

\NewDocumentCommand{\MediatioVIII}{m o}{%
  \MediatioII{#1}[#2]%
}
\NewDocumentCommand{\TerminatioVIII}{m o m O{} m}{%
  \TerminatioI{#1}[#2]{#3}[#4]{#5}%
}

\NewDocumentCommand{\MediatioC}{m O{}}{%
  \MediatioII{#1}[#2]%
}
\NewDocumentCommand{\TerminatioCI}{m O{} m}{%
  \TerminatioII{#1}[#2]{#3}%
}
\NewDocumentCommand{\TerminatioCII}{m}{%
  \textbf{#1}%
}
\NewDocumentCommand{\InchoatioC}{O{} m}{%
  #1\textbf{#2}%
}

\NewDocumentCommand{\MediatioCS}{m O{} m O{} m}{%
  \MediatioIV{#1}[#2]{#3}[#4]{#5}%
}
\NewDocumentCommand{\TerminatioCS}{m}{%
  \textbf{#1}%
}

\NewDocumentCommand{\MediatioD}{m O{}}{%
  \MediatioII{#1}[#2]%
}
\NewDocumentCommand{\TerminatioDI}{m}{%
  \textbf{#1}%
}
\NewDocumentCommand{\TerminatioDIe}{m O{} m O{} m}{%
  \TerminatioI{#1}[#2]{#3}[#4]{#5}%
}
\NewDocumentCommand{\TerminatioDS}{m O{} m}{%
  \TerminatioII{#1}[#2]{#3}%
}

\NewDocumentCommand{\MediatioE}{m O{}}{%
  \MediatioII{#1}[#2]%
}
\NewDocumentCommand{\TerminatioEI}{m O{} m}{%
  \textit{#1}#2\textbf{#3}%
}

\NewDocumentCommand{\TerminatioEII}{m}{%
  \textbf{#1}%
}

\NewDocumentCommand{\TerminatioES}{m O{} m}{%
  \TerminatioEI{#1}[#2]{#3}%
}

\NewDocumentCommand{\InchoatioE}{O{} m}{%
  #1\textbf{#2}%
}

\NewDocumentCommand{\MediatioTA}{m o m O{} m O{}}{%
  \MediatioIV{#1}[#2]{#3}[#4]{#5}[#6]%
}
\NewDocumentCommand{\TerminatioTA}{m}{%
  \textbf{#1}%
}

\NewDocumentCommand{\MediatioTB}{m m m O{}}{%
  \MediatioVII{#1}{#2}{#3}[#4]%
}
\NewDocumentCommand{\TerminatioTB}{m m m}{%
  \TerminatioV{#1}{#2}{#3}%
}

\NewDocumentCommand{\MediatioTC}{m O{}}{%
  \MediatioII{#1}[#2]%
}
\NewDocumentCommand{\TerminatioTC}{m O{} m}{%
  \TerminatioII{#1}[#2]{#3}%
}