% !TeX root = ../../main.tex
% chktex-file 1
\def\Prefix{tempus-per-annum/sanctissimi-corporis-et-sanguinis-christi/}

\subsection{Entrada}\label{subsection:tempus-per-annum/sanctissimi-corporis-et-sanguinis-christi/introitus}
\MakeChantAntiphonPsalm{\Prefix}{introitus/}

\AllowPageFlush

\subsection[Salmo Responsorial]{Salmo Responsorial \textmd{E 5}}\label{subsection:tempus-per-annum/sanctissimi-corporis-et-sanguinis-christi/psalmus-responsorius}
\MakeChantPsalmTwoVerses{\Prefix}{psalmus-responsorius/}

\subsection{Sequência}\label{subsection:tempus-per-annum/sanctissimi-corporis-et-sanguinis-christi/sequentia}
\begin{annotation}
  Missal Romano, 2ª Edição Típica (CNBB)
\end{annotation}
\begin{center}
  \begin{rubrica}
    A forma breve está entre colchetes.
  \end{rubrica}
\end{center}
\MakeChantLongPsalm{\Prefix}{sequentia/}{
  {psalmus-v1}{psalmus-v1-pt}
}

\subsection{Aleluia}\label{subsection:tempus-per-annum/sanctissimi-corporis-et-sanguinis-christi/alleluia}
\MakeChantAntiphonPsalm{\Prefix}{alleluia/}

\AllowPageFlush

\subsection[Salmo Aleluiático]{Salmo Aleluiático \textmd{C 4}}\label{subsection:tempus-per-annum/sanctissimi-corporis-et-sanguinis-christi/psalmus-alleluiaticus}
\begin{center}
  \begin{rubrica}
    O primeiro {\normalfont\Rbar} pode ser cantado apenas pelo grupo de cantores ou por todos. O segundo {\normalfont\Rbar} é cantado por todos.
  \end{rubrica}
\end{center}
\MakeChantPsalmOneVerse{\Prefix}{psalmus-alleluiaticus/}

\AllowPageFlush

\subsection{Ofertório}\label{subsection:tempus-per-annum/sanctissimi-corporis-et-sanguinis-christi/offertorium}
\MakeChantAntiphonPsalm{\Prefix}{offertorium/}

\AllowPageFlush

\subsection{Comunhão}\label{subsection:tempus-per-annum/sanctissimi-corporis-et-sanguinis-christi/communio}
\MakeChantAntiphonPsalm{\Prefix}{communio/}

\subsection{Procissão}\label{subsection:tempus-per-annum/sanctissimi-corporis-et-sanguinis-christi/ad-processionem}
\begin{annotation}
  Missal Romano, 2ª Edição Típica (CNBB)
\end{annotation}
\MakeChantLongPsalm{\Prefix}{ad-processionem/}{
  {psalmus-v1}{psalmus-v1-pt},
  {psalmus-v2}{psalmus-v2-pt},
  {psalmus-v3}{psalmus-v3-pt},
  {psalmus-v4}{psalmus-v4-pt}
}
\begin{rubrica}
  Repetem-se os versos acima, ou cantam-se outros hinos eucarísticos, durante a procissão, se necessário. Após a chegada ao altar, cantam-se os dois últimos versos do hino, a seguir:
\end{rubrica}
\MakeChantLongPsalm*{\Prefix}{ad-processionem/}{
  {psalmus-v5}{psalmus-v5-pt},
  {psalmus-v6}{psalmus-v6-pt}
}