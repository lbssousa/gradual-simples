% !TeX root = ../../a4.tex
% chktex-file 1

\subsection{Entrada}\label{subsection:proprium-sanctorum/omnium-sanctorum/introitus}
\MakeChantAntiphonPsalm{diligite-dominum.2D}{psalmi-ad-introitum}

\subsection[Salmo Responsorial]{Salmo Responsorial \textmd{C 3 g}}\label{subsection:proprium-sanctorum/omnium-sanctorum/psalmus-responsorius}
\MakeChantPsalmTwoVerses{psalmi-responsorii}{laudate-dominum-in-sanctis-eius.C3g}

\AllowPageFlush

\subsection{Aleluia}\label{subsection:proprium-sanctorum/omnium-sanctorum/alleluia}
\MakeChantAlleluiaPsalm{8c.1}{exsultate-iusti-in-domino.8c}

\subsection[Salmo Aleluiático]{Salmo Aleluiático \textmd{C 4}}\label{subsection:proprium-sanctorum/omnium-sanctorum/psalmus-alleluiaticus}
\begin{rubrica}
  O primeiro {\normalfont\Rbar} pode ser cantado apenas pelo grupo de cantores ou por todos. O segundo {\normalfont\Rbar} é cantado por todos.
\end{rubrica}
\MakeChantPsalmOneVerse{psalmi-alleluiatici}{exsultate-iusti-in-domino.C4}

\AllowPageFlush

\subsection{Ofertório}\label{subsection:proprium-sanctorum/omnium-sanctorum/offertorium}
\MakeChantAntiphonPsalm{nomen-sempiternum.7a}{psalmi-ad-offertorium}

\AllowPageFlush

\subsection{Comunhão}\label{subsection:proprium-sanctorum/omnium-sanctorum/communio}
\MakeChantAntiphonPsalm{beati-pacifici.1f}{psalmi-ad-communionem}