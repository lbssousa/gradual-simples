% !TeX root = ../a4.tex
% chktex-file 1

\subsection{Comunhão I}\label{subsection:appendices/communio-ad-libitum/communio-1}

\begin{annotation}
  \MakeAnnotation{}{Sl 34(33)}
\end{annotation}

\grechangedim{baselineskip}{55pt plus 5pt minus 5pt}{scalable}
\MakeChantLongPsalm*{communio-ad-libitum}{communio-1}{
  {psalmus-v1}{psalmus-v1-pt},
  {psalmus-v02}{psalmus-v02-pt},
  {psalmus-v03}{psalmus-v03-pt},
  {psalmus-v04}{psalmus-v04-pt},
  {psalmus-v05}{psalmus-v05-pt},
  {psalmus-v06}{psalmus-v06-pt},
  {psalmus-v07}{psalmus-v07-pt},
  {psalmus-v08}{psalmus-v08-pt},
  {psalmus-v09}{psalmus-v09-pt},
  {psalmus-v10}{psalmus-v10-pt},
  {psalmus-v11}{psalmus-v11-pt},
  {psalmus-v12}{psalmus-v12-pt},
  {psalmus-v13}{psalmus-v13-pt},
  {psalmus-v14}{psalmus-v14-pt},
  {psalmus-v15}{psalmus-v15-pt},
  {psalmus-v16}{psalmus-v16-pt},
  {psalmus-v17}{psalmus-v17-pt},
  {psalmus-v18}{psalmus-v18-pt},
  {psalmus-v19}{psalmus-v19-pt},
  {psalmus-v20}{psalmus-v20-pt},
  {psalmus-v21}{psalmus-v21-pt},
  {psalmus-v22}{psalmus-v22-pt},
  {gloria}{gloria-pt}
}

\AllowPageFlush

\subsectioncomplex{Comunhão II}{}{Para uso preferencial durante o Tempo da Quaresma}\label{subsection:appendices/communio-ad-libitum/communio-2}

\grechangedim{baselineskip}{66pt plus 5pt minus 5pt}{scalable}
\MakeChantPsalmOneVerse*{communio-ad-libitum}{communio-2}

\nobreaksubsection{Comunhão III}
\begin{rubrica}
  Ver Solenidade do Santíssimo Corpo e Sangue de Cristo, página~\pageref{subsection:tempus-per-annum/sanctissimi-corporis-et-sanguinis-christi/communio}.
\end{rubrica}

\AllowPageFlush

\subsection{Comunhão IV}\label{appendix:communio-ad-libitum/communio-4}
\MakeChantAntiphonPsalm{sanctum-nomen-domini.1g}{communio}

\AllowPageFlush

\subsection{Comunhão V}\label{appendix:communio-ad-libitum/communio-5}
\begin{annotation}
  Missal Romano, 2ª Edição Típica (CNBB)
\end{annotation}
\MakeChantLongPsalm{varia}{ubi-caritas}{
  {psalmus-v1}{psalmus-v1-pt},
  {psalmus-v2}{psalmus-v2-pt},
  {psalmus-v3}{psalmus-v3-pt}
}