% !TeX root = ../../../main.tex
\DeclareDocumentCommand{\Annotation}{}{
  \MakeAnnotation{\CantusID{001764}[Sl 96(95),2]}{Sl 96(95),1.3.6--13}
}

\def\LinkLA{tempus-paschale/missa-1/communio/antiphona}
\def\LinkPT{\LinkLA-pt}

\def\VersePairs{
  % 3
  {\Inchoatio{An}{nun}tiáte inter gentes glóriam \MediatioII{e}[ius,] in ómnibus pópulis mirabíli\TerminatioII{a}[ ]{e}ius. \Antiphona{Cantáte Dómino}[\LinkLA]}%
    {\Inchoatio{Pro}{cla}mai entre as nações a sua \MediatioII{gló}[ria,] entre todos os povos, as suas ma\TerminatioII{ra}{vi}lhas! \Antiphona{Cantai ao Senhor}[\LinkPT]},
  % 6
  {\Inchoatio{Ma}{gni}ficéntia et pulchritúdo in conspéctu \MediatioII{e}[ius,] poténtia et decor in sanctuári\TerminatioVII{o}{ e}ius. \Antiphona{Cantáte Dómino}[\LinkLA]}%
    {\Inchoatio{Ma}{jes}tade e magnificência estão diante \MediatioII{de}[le,] poder e esplendor estão no seu san\TerminatioII{tu}{á}rio. \Antiphona{Cantai ao Senhor}[\LinkPT]},
  % 7--8a
  {\Inchoatio{Af}{fér}te Dómino, famíliæ popu\Flexa{ló}[rum,] afférte Dómino glóriam et po\MediatioII{tén}[tiam,] afférte Dómino glóriam nómi\TerminatioII{nis}[ ]{e}ius. \Antiphona{Cantáte Dómino}[\LinkLA]}%
    {\Inchoatio{Tri}{bu}tai ao Senhor, ó famílias dos \Flexa{po}[vos,] tributai ao Senhor poder e \MediatioII{gló}[ria,] tributai ao Senhor a glória do \TerminatioII{seu}[ ]{no}me! \Antiphona{Cantai ao Senhor}[\LinkPT]},
  % 8b--9a
  {\Inchoatio{Tól}{li}te hóstias et introíte in átria \MediatioII{e}[ius,] adoráte Dóminum in splendó\TerminatioII{re}{ sanc}to. \Antiphona{Cantáte Dómino}[\LinkLA]}%
    {\Inchoatio{Tra}{zei} oferendas e entrai nos seus \MediatioII{á}[trios,] adorai o Senhor em seu santo es\TerminatioII{plen}{dor}. \Antiphona{Cantai ao Senhor}[\LinkPT]},
  % 9b--10a
  {\Inchoatio{Con}{tre}míscite a fácie eius, univérsa \MediatioII{ter}[ra;] dícite in géntibus: ``Dóminus \TerminatioII{re}{gná}vit!''. \Antiphona{Cantáte Dómino}[\LinkLA]}%
    {\Inchoatio{Tre}{me} diante dele, ó terra in\MediatioII{tei}[ra.] Dizei entre as nações: ``O Se\TerminatioII{nhor}[ ]{rei}na!'' \Antiphona{Cantai ao Senhor}[\LinkPT]},
  % 10bc
  {\Inchoatio{Et}{e}nim corréxit orbem terræ, qui non commo\MediatioII{\-vé}[\-bitur;] iudicábit pópulos in æ\TerminatioII{qui}{tá}te. \Antiphona{Cantáte Dómino}[\LinkLA]}%
    {\Inchoatio{E}{le} consolidou o orbe da terra, para que não va\MediatioII{ci}[le.] Ele julgará os povos com re\TerminatioII{ti}{dão}! \Antiphona{Cantai ao Senhor}[\LinkPT]},
  % 11--12a
  {\Inchoatio{Læ}{tén}tur cæli, et exsúltet \Flexa{ter}[ra,] sonet mare et plenitúdo \MediatioII{e}[ius;] gaudébunt campi et ómnia, quæ \TerminatioII{in}[ ]{e}is sunt. \Antiphona{Cantáte Dómino}[\LinkLA]}%
    {\Inchoatio{A}{le}grem-se os céus e exulte a \Flexa{ter}[ra,] reboe o mar e tudo o que ele con\MediatioII{tém}[;] rejubilem os campos e tudo o que neles \TerminatioII{e}{xis}te. \Antiphona{Cantai ao Senhor}[\LinkPT]},
  % 12b--13ab
  {\Inchoatio{Tunc}[ ]{ex}sultábunt ómnia ligna sil\Flexa{vá}[rum] a fácie Dómini, quia \MediatioII{ve}[nit,] quóniam venit iudicá\TerminatioII{re}[ ]{ter}ram. \Antiphona{Cantáte Dómino}[\LinkLA]}%
    {\Inchoatio{En}{tão} exultarão todas as árvores das \Flexa{ma}[tas] diante do Senhor, porque ele \MediatioII{vem}[!] Ele vem julgar \TerminatioII{a}[ ]{ter}ra. \Antiphona{Cantai ao Senhor}[\LinkPT]},
  % 13cd
  {\Inchoatio{Iu}{di}cábit orbem terræ in iu\MediatioII{stí}[tia] et pópulos in veritá\TerminatioII{te}{ su}a. \Antiphona{Cantáte Dómino}[\LinkLA]}%
    {\Inchoatio{Jul}{ga}rá o orbe da terra com jus\MediatioII{ti}[ça] e, na sua verdade, \TerminatioII{os}[ ]{po}vos. \Antiphona{Cantai ao Senhor}[\LinkPT]}
}