% !TeX root = ../fasciculo-1/a4.tex
% chktex-file 1

\subsection{Entrada}\label{subsection:tempus-nativitatis/in-nativitatis-domini/introitus}
\MakeChantAntiphonPsalm{dominus-dixit-ad-me.2D}{psalmi-ad-introitum}

\AllowPageFlush

\subsection[Salmo Responsorial]{Salmo Responsorial \textmd{C 3 g}}\label{subsection:tempus-nativitatis/in-nativitatis-domini/psalmus-responsorius}
\MakeChantPsalmTwoVerses{psalmi-responsorii}{tecum-principatus.C3g}

\AllowPageFlush

\subsection{Aleluia}\label{subsection:tempus-nativitatis/in-nativitatis-domini/alleluia}
\MakeChantAlleluiaPsalm{3g}{cantate-domino-canticum-novum.3g}

\subsection[Salmo Aleluiático]{Salmo Aleluiático \textmd{C 4}}\label{subsection:tempus-nativitatis/in-nativitatis-domini/psalmus-alleluiaticus}
\begin{rubrica}
  O primeiro {\normalfont\Rbar} pode ser cantado apenas pelo grupo de cantores ou por todos. O segundo {\normalfont\Rbar} é cantado por todos.
\end{rubrica}
\MakeChantPsalmOneVerse{psalmi-alleluiatici}{cantate-domino-canticum-novum.C4}

\subsection{Ofertório}\label{subsection:tempus-nativitatis/in-nativitatis-domini/offertorium}
\MakeChantAntiphonPsalm{laetentur-caeli.4A}{psalmi-ad-offertorium}

\AllowPageFlush

\subsection{Comunhão}\label{subsection:tempus-nativitatis/in-nativitatis-domini/communio}
\MakeChantAntiphonPsalm{viderunt-omnes.7a}{psalmi-ad-communionem}