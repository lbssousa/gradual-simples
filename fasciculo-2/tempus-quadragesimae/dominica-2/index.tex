% !TeX root = ../../a4.tex
% chktex-file 1
\def\Prefix{tempus-quadragesimae/dominica-2}

\subsection{Entrada}\label{subsection:tempus-quadragesimae/dominica-2/introitus}
\MakeChantAntiphonPsalm{\Prefix}{introitus}

\subsection[Salmo Responsorial I]{Salmo Responsorial I \textmd{E 5}}\label{subsection:tempus-quadragesimae/dominica-2/psalmus-responsorius-1}
\MakeChantPsalmTwoVerses{\Prefix}{psalmus-responsorius-1}

\iftoggle{compact}{\nobreaksubsection{Salmo Responsorial II}}{\subsection{Salmo Responsorial II}}
\begin{rubrica}
    Ver III Domingo do Tempo da Quaresma, página~\pageref{subsection:tempus-quadragesimae/dominica-3/psalmus-responsorius-2}.
\end{rubrica}

\nobreaksubsection{Antífona de aclamação}
\begin{rubrica}
    Ver III Domingo do Tempo da Quaresma, página~\pageref{subsection:tempus-quadragesimae/dominica-3/antiphona-acclamationis}.
\end{rubrica}

\nobreaksubsection{Trato}
\begin{rubrica}
    Ver III Domingo do Tempo da Quaresma, página~\pageref{subsection:tempus-quadragesimae/dominica-3/tractus}.
\end{rubrica}

\subsection{Ofertório}\label{subsection:tempus-quadragesimae/dominica-2/offertorium}
\MakeChantAntiphonPsalm{\Prefix}{offertorium}

\nottoggle{compact}{\clearpage\hbox{}\newpage}{}

\subsection{Comunhão}\label{subsection:tempus-quadragesimae/dominica-2/communio}
\MakeChantAntiphonPsalm{\Prefix}{communio}