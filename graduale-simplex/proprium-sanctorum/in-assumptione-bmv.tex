% !TeX root = ../../a4.tex
% chktex-file 1

\subsection{Entrada}\label{subsection:proprium-sanctorum/in-assumptione-bmv/introitus}
\MakeChantAntiphonPsalm{assumpta-est-maria-in-caelum.7a/}{introitus/}

\AllowPageFlush

\subsection[Salmo Responsorial]{Salmo Responsorial \textmd{E 5}}\label{subsection:proprium-sanctorum/in-assumptione-bmv/psalmus-responsorius}
\MakeChantPsalmTwoVerses{psalmi-responsorii/}{veni-de-libano.E5/}

\AllowPageFlush

\subsection{Aleluia}\label{subsection:proprium-sanctorum/in-assumptione-bmv/alleluia}
\MakeChantAlleluiaPsalm{3g/}{eructavit-cor-meum-verbum-bonum.3g/}

\AllowPageFlush

\subsection{Salmo Aleluiático}\label{subsection:proprium-sanctorum/in-assumptione-bmv/psalmus-alleluiaticus}
\begin{annotation}
  \MakeAnnotation{}{Sl 45(44),1.10b.11--15b.16}
\end{annotation}
\MakeChantLongPsalm*{psalmi-alleluiatici/}{eructavit-cor-meum-verbum-bonum/}{
  {psalmus-v1}{psalmus-v1-pt},
  {psalmus-v2}{psalmus-v2-pt},
  {psalmus-v3}{psalmus-v3-pt},
  {psalmus-v4}{psalmus-v4-pt},
  {psalmus-v5}{psalmus-v5-pt},
  {psalmus-v6}{psalmus-v6-pt},
  {psalmus-v7}{psalmus-v7-pt},
  {psalmus-v8}{psalmus-v8-pt}
}

\subsection{Ofertório}\label{subsection:proprium-sanctorum/in-assumptione-bmv/offertorium}
\MakeChantAntiphonPsalm{paradisi-portae.4A/}{offertorium/}

\nobreaksubsection{Ofertório opcional}
\begin{rubrica}
  Ver Solenidade da Anunciação do Senhor, página~\pageref{subsection:proprium-sanctorum/in-annuntiatione-domini/offertorium}.
\end{rubrica}

\AllowPageFlush

\subsection{Comunhão}\label{subsection:proprium-sanctorum/in-assumptione-bmv/communio}
\MakeChantAntiphonPsalm{beatam-me-dicent.8G/}{communio/}