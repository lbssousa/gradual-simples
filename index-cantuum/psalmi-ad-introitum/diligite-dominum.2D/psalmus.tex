\SetVersePairs{
  {\item \VSup{3ab}\Inchoatio{In}{clí}na ad me aurem \MediatioII{tu}[am,] accélera, ut é\TerminatioII{ru}{as} me.~\Antiphona}%
    {\item \VSup{3ab}\Inchoatio{In}{cli}nai o vosso ouvido \MediatioII{pa}[ra mim,] apressai-vos, ó Senhor, em so\TerminatioII{cor}{rer}-me!~\Antiphona},
  {\item \VSup{3cd}\Inchoatio{E}{sto} mihi in rupem præsídii et in domum mu\MediatioII{ní}[\-tam,] ut salvum \TerminatioII{me}[ ]{fá}cias.~\Antiphona}%
    {\item \VSup{3cd}\Inchoatio{Se}{de} uma rocha protetora \MediatioII{pa}[ra mim,] um abrigo bem seguro que \TerminatioII{me}[ ]{sal}ve!~\Antiphona},
  {\item \VSup{4}\Inchoatio{Quó}{ni}am fortitúdo mea et refúgium meum \MediatioII{es}[ tu,] et propter nomen tuum dedúces me \TerminatioII{et}[ ]{pa}sces me.~\Antiphona}%
    {\item \VSup{4}\Inchoatio{Sim}[, ]{sois} vós a minha rocha e forta\MediatioII{le}[za;] por vossa honra orientai-me e con\TerminatioII{du}{zi}-me.~\Antiphona},
  {\item \VSup{20ab}\Inchoatio{Quam}[ ]{ma}gna multitúdo dulcédinis tuæ, \MediatioII{Dó}[\-mi\-ne,] quam abscondísti timén\TerminatioII{ti}{bus} te.~\Antiphona}%
    {\item \VSup{20ab}\Inchoatio{Co}{mo} é grande, ó Senhor, vossa bon\MediatioII{da}[de,] que reservastes para aqueles que \TerminatioII{vos}[ ]{te}mem!~\Antiphona},
  {\item \VSup{20cd}\Inchoatio{Per}{fe}císti eis qui sperant \MediatioII{in}[ te,] in conspéctu filió\TerminatioII{rum}[ ]{hó}{mi}num.~\Antiphona}%
    {\item \VSup{20cd}\Inchoatio{Pa}{ra} aqueles que em vós se refu\MediatioII{gi}[am,] mostrando, assim, o vosso amor perante \TerminatioII{os}[ ]{ho}mens.~\Antiphona},
  {\item \VSup{21ab}\Inchoatio{Ab}{scón}des eos in abscóndito faciéi \MediatioII{tu}[æ] a conturbatió\TerminatioII{ne}[ ]{hó}minum.~\Antiphona}%
    {\item \VSup{21ab}\Inchoatio{Na}[ ]{pro}teção de vossa face os \MediatioII{de}[fendeis] bem longe das intrigas \TerminatioII{dos}[ ]{mor}tais.~\Antiphona},
  {\item \VSup{21cd}\Inchoatio{Pró}{te}ges eos in taber\MediatioII{ná}[culo] a contradictióne linguárum.~\Antiphona}%
    {\item \VSup{21cd}\Inchoatio{No}[ ]{in}terior de cossa tenda os \MediatioII{es}[condeis,] pro\-te\-gen\-do-os contra as línguas mal\TerminatioII{di}{zen}tes.~\Antiphona},
  {\item \VSup{22}\Inchoatio{Be}{ne}díctus \MediatioII{Dó}[minus,] quóniam mirificávit misericórdiam suam mihi in civitáte \TerminatioII{mu}{ní}ta.~\Antiphona}%
    {\item \VSup{22}\Inchoatio{Se}{ja} bendito o Senhor Deus, que me \MediatioII{mos}[trou] seu grande amor numa cidade pro\TerminatioII{te}{gi}da!~\Antiphona},
  {\item \VSup{25}\Inchoatio{Vi}{rí}liter ágite, et confortétur cor \MediatioII{ve}[strum,] omnes qui sperátis \TerminatioII{in}[ ]{Dó}mino.~\Antiphona}
    {\item \VSup{25}\Inchoatio{For}{ta}lecei os corações, tende co\MediatioII{ra}[gem,] todos vós que ao Senhor \TerminatioII{vos}[ ]{con}fiais!~\Antiphona}
}