\SetVerseAnnotation{Sl 133(132)}

\def\VersePairs{
  % 3b
  {\Inchoatio{Sic}{ut} unguéntum ópti\MediatioIV{mum}[ ]{in}[ ]{cá}[pite,] quod descéndit in bar\TerminatioIV{bam}[, ]{bar}{bam}[ ]{A}aron. \Antiphona{Faciámus hic}[\LinkLA]}%
    {\Inchoatio{É}[ ]{co}mo o óleo precioso sobre \MediatioIV{a}[ ]{ca}{be}[ça,] que desce pela barba, pela barba \TerminatioIV{de }{A}{a}{rão}. \Antiphona{Façamos aqui}[\LinkPT]},
  % 4
  {\Inchoatio{Quod}[ ]{de}scéndit in oram vesti\MediatioIV{mén}{ti}[ ]{e}[ius:] sicut ros Hermon, qui descéndit \TerminatioIV{in}[ ]{mon}{tes}[ ]{Si}on. \Antiphona{Faciámus hic}[\LinkLA]}%
    {\Inchoatio{Que}[ ]{des}ce até a gola \MediatioIV{do}[ ]{seu}[ ]{man}[to.] É como o orvalho do Hermon, que desce sobre os mon\TerminatioIV{tes}[ ]{de}[ ]{Si}{ão}. \Antiphona{Façamos aqui}[\LinkPT]},
  % 5
  {\Inchoatio{Quó}{ni}am illic mandávit Dóminus bene\MediatioIV{dic}{ti}{ó}[nem,] et vitam \TerminatioIV{us}{que}[ ]{in}[ ]{sǽ}culum. \Antiphona{Faciámus hic}[\LinkLA]}%
    {\Inchoatio{Por}{que} o Senhor derrama ali a \MediatioIV{su}{a}[ ]{bên}[ção,] a vi\TerminatioIV{da}[ ]{pa}{ra}[ ]{sem}pre. \Antiphona{Façamos aqui}[\LinkPT]}
}