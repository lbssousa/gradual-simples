% !TeX root = ../../a4.tex
% chktex-file 1

\subsection{Entrada}\label{subsection:proprium-sanctorum/in-annuntiatione-domini/introitus}
\def\AntiphonAnnotation{\CantusID{001414}[Lc 1,30]}
\def\AntiphonScore{angelus-domini.1f/}
\MakeChantAntiphonPsalm*{introitus/}{\AntiphonScore}

\nobreaksubsection{Salmo Responsorial}

\begin{rubrica}
  Ver Solenidade da Imaculada Conceição de Nossa Senhora, página~\pageref{subsection:proprium-sanctorum/in-conceptione-immaculata-bmv/psalmus-responsorius}
\end{rubrica}

\nobreaksubsection{Aleluia}

\begin{rubrica}
  Ver Solenidade da Imaculada Conceição de Nossa Senhora, página~\pageref{subsection:proprium-sanctorum/in-conceptione-immaculata-bmv/alleluia}
\end{rubrica}

\nobreaksubsection{Salmo Aleluiático}

\begin{rubrica}
  Ver Solenidade da Imaculada Conceição de Nossa Senhora, página~\pageref{subsection:proprium-sanctorum/in-conceptione-immaculata-bmv/psalmus-alleluiaticus}
\end{rubrica}

\AllowPageFlush

\subsection{Ofertório}\label{subsection:proprium-sanctorum/in-annuntiatione-domini/offertorium}
\def\AntiphonAnnotation{\CantusID{001539}[Lc 1,28]}
\def\AntiphonScore{ave-maria.1g/}
\MakeChantAntiphonPsalm*{offertorium/}{\AntiphonScore}

\subsection{Comunhão}\label{subsection:proprium-sanctorum/in-annuntiatione-domini/communio}
\def\AntiphonAnnotation{\CantusID{002491}[Lc 1,38]}
\def\AntiphonScore{ecce-ancillae-domini.8c/}
\MakeChantAntiphonPsalm{communio/}{\AntiphonScore}