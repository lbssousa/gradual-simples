% !TeX root = ../../../a4.tex
\DeclareDocumentCommand{\Annotation}{}{
  \SetVerseAnnotation{\CantusID{005243}[Is 58,9]}{Sl 91(90),1--2.4--5a.8--13}
}

\def\LinkLA{tempus-quadragesimae/dominica-1/introitus/antiphona}
\def\LinkPT{\LinkLA-pt}

\def\VersePairs{
  % 2
  {\Inchoatio{Di}{cet} \Flexa{Dó}[mino:] ``Refúgium meum et forti\MediatioVII{tú}{do }{me}[a,] Deus meus, spe\TerminatioVII{rá}{bo in }{e}um''. \Antiphona{Tunc invocábis}[\LinkLA]}%
    {\Inchoatio{E}{le} dirá ao Se\Flexa{nhor}[:] ``Meu refúgio e minha \MediatioVII{for}{ta}{le}[\-za,] meu Deus, em \TerminatioVII{quem}{ con}{fi}o!'' \Antiphona{Quand\Elisio{o} o invocares}[\LinkPT]},
  % 4ab
  {\Inchoatio{A}{lis} suis obum\MediatioVII{brá}{bit }{ti}[bi,] et sub pennas \TerminatioVII{e}{ius con}{\-fú}\-gies. \Antiphona{Tunc invocábis}[\LinkLA]}%
    {\Inchoatio{Com}[ ]{su}as asas te \MediatioVII{co}{bri}{rá}[,] sob suas penas encontra\TerminatioVII{rás}{ re}{fú}gio. \Antiphona{Quand\Elisio{o} o invocares}[\LinkPT]},
  % 4c-5a
  {\Inchoatio{Scu}{tum} et loríca \MediatioVII{vé}{ritas }{e}[ius;] non timébis a ti\TerminatioVII{mó}{re noc}{\-túr}no, \Antiphona{Tunc invocábis}[\LinkLA]}%
    {\Inchoatio{Es}{cu}do e couraça é a sua fi\MediatioVII{de}{li}{da}[de.] Não terás medo do ter\TerminatioVII{ror}{ no}{tur}no. \Antiphona{Quand\Elisio{o} o invocares}[\LinkPT]},
  % 8
  {\Inchoatio{Quod}{si} óculis tuis con\MediatioVII{si}{de}{rá}[veris,] retributiónem pecca\TerminatioVII{\-tó}{rum vi}{dé}bis. \Antiphona{Tunc invocábis}[\LinkLA]}%
    {\Inchoatio{Bas}{ta} que olhes \MediatioVII{com}{ teus }{o}[lhos,] e verás a \TerminatioVII{pa}{ga dos }{ím}\-pios! \Antiphona{Quand\Elisio{o} o invocares}[\LinkPT]},
  % 9
  {\Inchoatio{Quó}{ni}am tu es, Dómine, re\MediatioVII{fú}{gium }{me}[um.] Altíssimum posuísti habi\TerminatioVII{tá}{culum }{tu}um. \Antiphona{Tunc invocábis}[\LinkLA]}%
    {\Inchoatio{``Pois}[ ]{vós}, Senhor, sois o \MediatioVII{meu}{ re}{fú}[gio''.] Fizeste do Altíssimo a \TerminatioVII{tu}{a mo}{ra}da. \Antiphona{Quand\Elisio{o} o invocares}[\LinkPT]},
  % 10
  {\Inchoatio{Non}[ ]{ac}cédet \MediatioVII{ad}{ te }{ma}[lum,] et flagéllum non appropin\-quábit taber\TerminatioVII{ná}{culo }{tu}o. \Antiphona{Tunc invocábis}[\LinkLA]}%
    {\Inchoatio{O}[ ]{mal} não se aproxima\MediatioVII{rá}{ de }{ti} e o flagelo não chegará à \TerminatioVII{tu}{a }{ten}da. \Antiphona{Quand\Elisio{o} o invocares}[\LinkPT]},
  % 11
  {\Inchoatio{Quó}{ni}am ángelis suis man\MediatioVII{dá}{bit }{de}[ te,] ut custódiant te in ómnibus \TerminatioVII{vi}{is }{tu}is. \Antiphona{Tunc invocábis}[\LinkLA]}%
    {\Inchoatio{Pois}[ ]{a} seus anjos ele mandou, a \MediatioVII{teu}{ res}{pei}[to,] que te guardem em todos os \TerminatioVII{teus}{ ca}{mi}nhos. \Antiphona{Quand\Elisio{o} o invocares}[\LinkPT]},
  % 12
  {\Inchoatio{In}[ ]{má}ni\MediatioVII{bus}{ por}{tá}[bunt te,] ne forte offéndas ad lápidem \TerminatioVII{pe}{dem }{tu}um. \Antiphona{Tunc invocábis}[\LinkLA]}%
    {\Inchoatio{E}{les} te levarão em \MediatioVII{su}{as }{mãos}[,] para que não tropeces com o pé nal\TerminatioVII{gu}{ma }{pe}dra. \Antiphona{Quand\Elisio{o} o invocares}[\LinkPT]},
  % 13
  {\Inchoatio{Su}{per} áspidem et basilíscum \MediatioVII{am}{bu}{lá}[bis,] et conculcábis leónem \TerminatioVII{et}{ dra}{có}nem. \Antiphona{Tunc invocábis}[\LinkLA]}%
    {\Inchoatio{An}{da}rás sobre cobras \MediatioVII{e}{ ser}{pen}[tes,] e calcarás leões \TerminatioVII{e}{ dra}{\-gões}. \Antiphona{Quand\Elisio{o} o invocares}[\LinkPT]}
}