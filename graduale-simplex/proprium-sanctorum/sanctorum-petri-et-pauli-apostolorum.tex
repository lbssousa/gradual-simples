% !TeX root = ../../a4.tex
% chktex-file 1

\subsection{Entrada}\label{subsection:proprium-sanctorum/sanctorum-petri-et-pauli-apostolorum/introitus}
\def\AntiphonAnnotation{\CantusID{003783}[At 12,11]}
\def\AntiphonScore{misit-dominus.7c2/}
\MakeChantAntiphonPsalm{introitus/}{\AntiphonScore}

\subsection[Salmo Responsorial]{Salmo Responsorial \textmd{C 3 g}}\label{subsection:proprium-sanctorum/sanctorum-petri-et-pauli-apostolorum/psalmus-responsorius}
\MakeChantPsalmThreeVerses{psalmi-responsorii/}{constitues-eos-principes.C3g/}

\AllowPageFlush

\subsection{Aleluia}\label{subsection:proprium-sanctorum/sanctorum-petri-et-pauli-apostolorum/alleluia}
\def\AntiphonScore{alleluia.3g/}
\MakeChantAntiphonPsalm{alleluia/}{in-convertendo-dominus.3g/}

\subsection[Salmo Aleluiático]{Salmo Aleluiático \textmd{C 4}}\label{subsection:proprium-sanctorum/sanctorum-petri-et-pauli-apostolorum/psalmus-alleluiaticus}
\begin{rubrica}
  O primeiro {\normalfont\Rbar} pode ser cantado apenas pelo grupo de cantores ou por todos. O segundo {\normalfont\Rbar} é cantado por todos.
\end{rubrica}
\MakeChantPsalmOneVerse{psalmi-alleluiatici/}{in-convertendo-dominus.C4/}

\AllowPageFlush

\subsection{Ofertório}\label{subsection:proprium-sanctorum/sanctorum-petri-et-pauli-apostolorum/offertorium}
\def\AntiphonAnnotation{\CantusID{001902}[Sl 45(44),17--18]}
\def\AntiphonScore{constitues-eos-principes.7a/}
\MakeChantAntiphonPsalm{offertorium/}{\AntiphonScore}

\AllowPageBreak

\subsection{Comunhão}\label{subsection:proprium-sanctorum/sanctorum-petri-et-pauli-apostolorum/communio}
\def\AntiphonAnnotation{\CantusID{005208}[Mt 16,18]}
\def\AntiphonScore{tu-es-petrus.7c/}
\MakeChantAntiphonPsalm{communio/}{\AntiphonScore}