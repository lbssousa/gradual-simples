% !TeX root = ../../a4.tex
% chktex-file 1

\subsection{Entrada}\label{subsection:proprium-sanctorum/in-nativitate-bmv/introitus}
\SetAntiphonAnnotation{\CantusID{005453}}
\SetAntiphonScore{virgo-maria.7a/}
\MakeChantAntiphonPsalm{introitus/}

\nobreaksubsection{Salmo Responsorial}

\begin{rubrica}
  Ver Solenidade da Imaculada Conceição de Nossa Senhora, página~\pageref{subsection:proprium-sanctorum/in-conceptione-immaculata-bmv/psalmus-responsorius}.
\end{rubrica}

\nobreaksubsection{Aleluia}

\begin{rubrica}
  Ver Solenidade da Imaculada Conceição de Nossa Senhora, página~\pageref{subsection:proprium-sanctorum/in-conceptione-immaculata-bmv/alleluia}.
\end{rubrica}

\nobreaksubsection{Salmo Aleluiático}

\begin{rubrica}
  Ver Solenidade da Imaculada Conceição de Nossa Senhora, página~\pageref{subsection:proprium-sanctorum/in-conceptione-immaculata-bmv/psalmus-alleluiaticus}.
\end{rubrica}

\nobreaksubsection{Ofertório}

\begin{rubrica}
  Ver Solenidade da Anunciação do Senhor, página~\pageref{subsection:proprium-sanctorum/in-annuntiatione-domini/offertorium}.
\end{rubrica}

\AllowPageFlush

\subsection{Comunhão}\label{subsection:proprium-sanctorum/in-nativitate-bmv/communio}
\SetAntiphonAnnotation{\CantusID{000000}}
\SetAntiphonScore{beata-es.8G/}
\MakeChantAntiphonPsalm{communio/}