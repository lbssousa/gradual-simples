% chktex-file 1
\def\Prefix{subsection:proprium-sanctorum/sancti-ioseph-sponsi-bmv}

\begin{rubrica}
  Nos anos em que o dia 19 de março está situado dentro da Semana Santa, esta solenidade é transferida para o primeiro dia livre após Oitava da Páscoa.
\end{rubrica}

\SectionAntiphonPsalm{Entrada}[\Prefix/introitus]{ioseph-fili-david.7a}{Salmo 91(92)}{psalmi-ad-introitum}

\AllowPageFlush

\SectionPsalm{Responsorial}[\Prefix/psalmus-responsorius]{Salmo 83(84)}{psalmi-responsorii}{quam-dilecta-tabernacula-tua.D1b}

\AllowPageFlush

\SectionAlleluiaPsalm[\Prefix/alleluia]{4E.1}{Salmo 26(27)}{unum-petii-a-domino.4E}

\AllowPageFlush

\SectionPsalm{Aleluiático}[\Prefix/psalmus-alleluiaticus]{Salmo 26(27)}{psalmi-alleluiatici}{dominus-illuminatio-mea.C1}

%\subsection{Trato}\label{\Prefix/tractus}
%\begin{rubrica}
%  O Graduale Simplex, originalmente, não propõe um trato para esta solenidade. A proposta aqui apresentada é derivada do trato do Graduale Romanum para esta solenidade, aplicando-se a melodia do tra\-to do Graduale Simplex para o I Domingo do Tempo da Quaresma.
%\end{rubrica}
%\MakeChantPsalmOneVerse{tractus}{beatus-vir-qui-timet-dominum}

\AllowPageFlush

\SectionAntiphonPsalm{Ofertório}[\Prefix/offertorium]
{descendit.8G}{Salmo 1}{psalmi-ad-offertorium}

\AllowPageFlush

\SectionAntiphonPsalm{Comunhão}[\Prefix/communio]
{ecce-fidelis-servus-et-prudens.8G}{Salmo 20(21)}{psalmi-ad-communionem}

\AllowPageFlush