% !TeX root = ../../../main.tex
\DeclareDocumentCommand{\Annotation}{}{
  \MakeAnnotation{\CantusID{005139}[Sl 76(75),9--10]}{Sl 98(97)}
}

\def\LinkLA{tempus-paschale/missa-2/introitus/antiphona}
\def\LinkPT{\LinkLA-pt}

\def\VersePairs{
  % 2
  {\Inchoatio{No}{tum} fecit Dóminus salu\MediatioVII{tá}{re }{su}[um,] in conspéctu géntium revelávit iu\TerminatioVII{stí}{tiam }{su}am. \Antiphona{Cantáte Dómino}[\LinkLA]}%
    {\Inchoatio{O}[ ]{Se}nhor manifestou sua \MediatioVII{sal}{va}{ção}[,] diante das nações revelou \TerminatioVII{su}{a jus}{ti}ça. \Antiphona{Cantai ao Senhor}[\LinkPT]},
  % 3ab
  {\Inchoatio{Re}{cor}dátus est miseri\MediatioVII{cór}{diæ }{su}[æ] et veritátis suæ \TerminatioVII{dó}{mui }{Is}rael. \Antiphona{Cantáte Dómino}[\LinkLA]}%
    {\Inchoatio{Lem}{brou}-se da sua mi\MediatioVII{se}{ri}{cór}[dia] e da sua fidelidade para com a casa de \MediatioVII{Is}{ra}{el}. \Antiphona{Cantai ao Senhor}[\LinkPT]},
  % 3cd
  {\Inchoatio{Vi}{dé}runt omnes \MediatioVII{tér}{mini }{ter}[ræ] salutáre \TerminatioVII{De}{i }{no}stri. \Antiphona{Cantáte Dómino}[\LinkLA]}%
    {\Inchoatio{To}{dos} os confins da \MediatioVII{ter}{ra }{vi}[ram] a salvação que vem do \MediatioVII{nos}{so }{Deus}. \Antiphona{Cantai ao Senhor}[\LinkPT]},
  % 4
  {\Inchoatio{Iu}{bi}láte Deo, \MediatioVII{om}{nis }{ter}[ra;] erúmpite, exsul\TerminatioVII{tá}{te et }{psál}lite. \Antiphona{Cantáte Dómino}[\LinkLA]}%
    {\Inchoatio{A}{cla}mai a Deus, \MediatioVII{ter}{ra in}{tei}[ra,] prorrompei em cantos, exultai e sal\MediatioVII{mo}{di}{ai}. \Antiphona{Cantai ao Senhor}[\LinkPT]},
  % 5
  {\Inchoatio{Psál}{li}te Dómi\MediatioVII{no}{ in }{cí}[thara,] in cíthara et \TerminatioVII{vo}{ce }{psal}mi. \Antiphona{Cantáte Dómino}[\LinkLA]}%
    {\Inchoatio{Can}{tai} salmos ao Senhor \MediatioVII{com}{ a }{cí}[tara,] com a cítara e ao \TerminatioVII{som}{ da }{mú}sica. \Antiphona{Cantai ao Senhor}[\LinkPT]},
  % 6
  {\Inchoatio{In}[ ]{tu}bis ductílibus et voce \MediatioVII{tu}{bæ }{cór}[neæ,] iubiláte in conspéctu \TerminatioVII{re}{gis }{Dó}mini. \Antiphona{Cantáte Dómino}[\LinkLA]}%
    {\Inchoatio{Com}[ ]{as} trombetas e ao som \MediatioVII{da}{ cor}{ne}[ta,] aclamai diante do Rei, \MediatioVII{o}{ Se}{nhor}. \Antiphona{Cantai ao Senhor}[\LinkPT]},
  % 7
  {\Inchoatio{So}{net} mare et pleni\MediatioVII{tú}{do }{e}[ius,] orbis terrárum et qui hábi\TerminatioVII{tant}{ in }{e}o. \Antiphona{Cantáte Dómino}[\LinkLA]}%
    {\Inchoatio{Re}{bo}e o mar e tudo o que \MediatioVII{ne}{le e}{xis}[te,] o orbe da terra e os que \MediatioVII{o}{ ha}{bi}tam. \Antiphona{Cantai ao Senhor}[\LinkPT]},
  % 8--9a
  {\Inchoatio{Flú}{mi}na plaudent \Flexa{ma}[nu,] simul montes \MediatioVII{ex}{sul}{\-tá}[\-bunt] a conspéctu Dómini, quóniam venit iudi\TerminatioVII{cá}{re }{ter}\-ram. \Antiphona{Cantáte Dómino}[\LinkLA]}%
    {\Inchoatio{Os}[ ]{ri}os batam \Flexa{pal}[mas] e juntos exultem os montes, diante \MediatioVII{do}{ Se}{nhor}[,] porque ele vem jul\TerminatioVII{gar}{ a }{ter}ra. \Antiphona{Cantai ao Senhor}[\LinkPT]},
  % 9bc
  {\Inchoatio{Iu}{di}cábit orbem terrárum \MediatioVII{in}{ iu}{stí}[tia] et pópulos in \TerminatioVII{æ}{qui}{tá}te. \Antiphona{Cantáte Dómino}[\LinkLA]}%
    {\Inchoatio{E}{le} julgará o mundo \MediatioVII{com}{ jus}{ti}[ça] e os povos, com \MediatioVII{re}{ti}{dão}. \Antiphona{Cantai ao Senhor}[\LinkPT]}
}