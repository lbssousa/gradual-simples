% !TeX root = ../../../main.tex
\DeclareDocumentCommand{\Annotation}{}{
  \MakeAnnotation{\CantusID[?]{004769}}{Sl 1,1--3}
}

\def\LinkLA{proprium-sanctorum/in-exaltatione-sanctae-crucis/offertorium/antiphona}
\def\LinkPT{\LinkLA-pt}

\def\VersePairs{
  %
  {\Inchoatio{Sed}[ ]{in} lege Dómini vo\MediatioI{lún}{tas }{e}[ius,] et in lege eius meditátur di\TerminatioI{e}[ ]{ac}[ ]{no}cte. \Antiphona{Sanctum nomen Dómini}[\LinkLA]}%
    {\Inchoatio{Pe}{lo} contrário, seu prazer está na lei \MediatioI{do}{ Se}{nhor}[,] e na sua Lei medita di\TerminatioI{a}[ ]{e}[ ]{noi}te. \Antiphona{O santo nome do Senhor}[\LinkPT]},
  %
  {\Inchoatio{Et}[ ]{e}rit tanquam lignum plantátum secus de\MediatioI{cúr}{sus a}{quá}[\-rum,] quod fructum suum dabit in tém\TerminatioI{po}{re}[ ]{su}o. \Antiphona{Sanctum nomen Dómini}[\LinkLA]}%
    {\Inchoatio{E}{le} será como árvore plantada à \MediatioI{bei}{ra das }{á}[guas,] que dá fruto no de\TerminatioI{vi}{do}[ ]{tem}po. \Antiphona{O santo nome do Senhor}[\LinkPT]},
  %
  {\Inchoatio{Et}[ ]{fó}lium \MediatioI{e}{ius non }{dé}[fluet,] et ómnia quæcúmque fáciet pro\TerminatioI{spe}{ra}{bún}tur. \Antiphona{Sanctum nomen Dómini}[\LinkLA]}%
    {\Inchoatio{Su}{as} \MediatioI{fo}{lhas não }{mur}[cham,] e tudo o que faz \TerminatioI{tem}[ ]{bom}[ ]{ê}xito. \Antiphona{O santo nome do Senhor}[\LinkPT]}
}