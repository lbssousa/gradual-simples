\SetVerseAnnotation{Sl 142(141),1--6a.7--8}

\def\VersePairs{
  %
  {\Inchoatio{Ef}{fún}do in conspéctu eius lamentatiónem \MediatioVIII{me}[am,] et tribulatiónem meam ante ip\TerminatioVIII{sum}[ ]{pro}{nún}tio. \Antiphona{Pórtio mea, Dómine}[\LinkLA]}%
    {\Inchoatio{Di}{an}te dele derramo a minha \MediatioVIII{quei}[xa,] exponho diante dele minha tri\TerminatioVIII{bu}{la}{ção}. \Antiphona{Sois minh\Elisio{a} herança, ó Senhor}[\LinkPT]},
  %
  {\Inchoatio{In}[ ]{de}ficiéndo in me spíritus \MediatioVIII{me}[us,] tu cognovísti sé\TerminatioVIII{mi}{tas}[ ]{me}as. \Antiphona{Pórtio mea, Dómine}[\LinkLA]}%
    {\Inchoatio{En}{quan}to meu espírito em mim desfa\MediatioVIII{le}[ce,] vós conheceis as mi\TerminatioVIII{nhas}[ ]{ve}{re}das. \Antiphona{Sois minh\Elisio{a} herança, ó Senhor}[\LinkPT]},
  %
  {\Inchoatio{In}[ ]{vi}a qua ambu\MediatioVIII{lá}[bam,] abscondérunt lá\TerminatioVIII{que}{um}[ ]{mi}hi. \Antiphona{Pórtio mea, Dómine}[\LinkLA]}%
    {\Inchoatio{No}[ ]{ca}minho onde eu an\MediatioVIII{da}[va,] esconderam uma armadilha \TerminatioVIII{pa}{ra}[ ]{mim}. \Antiphona{Sois minh\Elisio{a} herança, ó Senhor}[\LinkPT]},
  %
  {\Inchoatio{Con}{sí}dera ad déxteram, et \MediatioVIII{vi}[de:] et non est \TerminatioVIII{qui}[ ]{a}{gnó}scat me. \Antiphona{Pórtio mea, Dómine}[\LinkLA]}%
    {\Inchoatio{Eu}[ ]{o}lhava à direita e procu\MediatioVIII{ra}[va:] não havia que me \TerminatioVIII{co}{nhe}{\-ces}\-se. \Antiphona{Sois minh\Elisio{a} herança, ó Senhor}[\LinkPT]},
  %
  {\Inchoatio{Pé}{ri}it fuga \MediatioVIII{a}[ me,] et non est qui requírat á\TerminatioVIII{ni}{mam}[ ]{me}am. \Antiphona{Pórtio mea, Dómine}[\LinkLA]}%
    {\Inchoatio{Não}[ ]{te}nho aonde fu\MediatioVIII{gir}[,] não há quem se interesse por \TerminatioVIII{mi}{nha}[ ]{vi}da. \Antiphona{Sois minh\Elisio{a} herança, ó Senhor}[\LinkPT]},
  %
  {\Inchoatio{Cla}{má}vi ad te, \MediatioVIII{Dó}[mine;] dixi: ``Tu es refú\TerminatioVIII{gi}{um}[ ]{me}um. \Antiphona{Pórtio mea, Dómine}[\LinkLA]}%
    {\Inchoatio{Cla}{mo} a vós, Senhor, di\MediatioVIII{zen}[do:] ``Vós sois o \TerminatioVIII{meu}[ ]{re}\TerminatioVIII{fú}gio. \Antiphona{Sois minh\Elisio{a} herança, ó Senhor}[\LinkPT]},
  %
  {``\Inchoatio{In}{tén}de ad deprecatiónem \MediatioVIII{me}[am,] quia humiliá\TerminatioVIII{tus}[ ]{sum}[ ]{ni}mis. \Antiphona{Pórtio mea, Dómine}[\LinkLA]}%
    {``\Inchoatio{A}{ten}dei à minha \MediatioVIII{sú}[plica,] pois fui por demais \TerminatioVIII{hu}{mi}{lha}\-do. \Antiphona{Sois minh\Elisio{a} herança, ó Senhor}[\LinkPT]},
  %
  {``\Inchoatio{Lí}{be}ra me a persequénti\MediatioVIII{bus}[ me,] quia confortá\TerminatioVIII{ti}[ ]{sunt}[ ]{su}per me. \Antiphona{Pórtio mea, Dómine}[\LinkLA]}%
    {``\Inchoatio{Li}{vrai}-me dos que me per\MediatioVIII{se}[guem,] pois são mais fortes \TerminatioVIII{do}[ ]{que}[ ]{eu}. \Antiphona{Sois minh\Elisio{a} herança, ó Senhor}[\LinkPT]},
  %
  {``\Inchoatio{E}{duc} de custédia ánimam \MediatioVIII{me}[am] ad confitén\-dum nó\TerminatioVIII{mi}{\-ni}[ ]{tu}o. \Antiphona{Pórtio mea, Dómine}[\LinkLA]}%
    {``\Inchoatio{Ti}{rai} minha alma da pri\MediatioVIII{são}[,] e darei graças ao \TerminatioVIII{vos}{so}[ ]{no}me. \Antiphona{Sois minh\Elisio{a} herança, ó Senhor}[\LinkPT]},
  %
  {``\Inchoatio{Cir}{cum}dábunt me \MediatioVIII{iu}[sti,] cum retribú\TerminatioVIII{e}{ris}[ ]{mi}hi''.
      \Antiphona{Pórtio mea, Dómine}[\LinkLA]}%
    {``\Inchoatio{Os}[ ]{jus}tos estarão ao meu re\MediatioVIII{dor}[,] quando me retribuíres com vos\TerminatioVIII{so}[ ]{fa}{vor}''. \Antiphona{Sois minh\Elisio{a} herança, ó Senhor}[\LinkPT]}
}