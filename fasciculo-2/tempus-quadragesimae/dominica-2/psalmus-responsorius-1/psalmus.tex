% !TeX root = ../../../main.tex
\DeclareDocumentCommand{\Annotation}{}{
  \MakeAnnotation{}{Sl 91(90),4.10--16}
}

\def\LinkLA{tempus-quadragesimae/dominica-2/psalmus-responsorius-1/psalmus-v1}
\def\LinkPT{\LinkLA-pt}

\def\VersePairs{
  % 11
  {Desidérium cordis eius tribuísti \MediatioCII{e}[i,] et voluntátem labiórum eius non dene\TerminatioCII{gá}sti. \Responsorium{In salutári tuo}[\LinkLA]}%
    {Pois a seus anjos ele mantou, a teu res\MediatioEV{pei}[to,] que te guardem em todos os teus \TerminatioEV{ca}{mi}nhos. \Responsorium{Sob suas penas}[\LinkPT]},
  % 12
  {Quóniam prævenísti eum in benedictiónibus dul\MediatioCII{cé}[\-dinis;] posuísti in cápite eius corónam de auro pu\TerminatioCII{ríssi}mo. \Responsorium{In salutári tuo}[\LinkLA]}%
    {\InchoatioEV{E}les te levarão em suas \MediatioEV{mãos}[,] para que não tropeces com pé nalgu\TerminatioEV{ma}[ ]{pe}dra. \Responsorium{Sob suas penas}[\LinkPT]},
  % 13
  {Vitam pétiit a te, et tribuísti \MediatioCII{e}[i,] longitúdinem diérum in sǽculum et in sǽculum \TerminatioCII{sǽcu}li. \Responsorium{In salutári tuo}[\LinkLA]}%
    {Andarás sobre cobras e ser\MediatioEV{pen}[tes,] e calcarás leões e \TerminatioEV{dra}{\-gões}. \Responsorium{Sob suas penas}[\LinkPT]},
  % 14
  {Quóniam pones eum benedictiónem in sǽculum \MediatioCII{sǽ}[culi,] lætificábis eum in gáudio ante vultum \TerminatioCII{tu}\-um. \Responsorium{In salutári tuo}[\LinkLA]}%
    {\InchoatioEV[Por]{que} se apegou a mim, eu o livra\MediatioEV{rei}[;] eu o protegerei, pois conhece o \TerminatioEV{meu}[ ]{no}me. \Responsorium{Sob suas penas}[\LinkPT]},
  % 15
  {Quóniam rex sperat in \MediatioCII{Dó}[mino,] et in misericórdia Altíssimi non commo\TerminatioCII{vébi}tur. \Responsorium{In salutári tuo}[\LinkLA]}%
    {Clamará por mim, e eu o atende\Respirare{rei}[;] com ele estarei na tribula\MediatioEV{ção}[,] e o livrarei e o glorifi\TerminatioEV{ca}{rei}. \Responsorium{Sob suas penas}[\LinkPT]},
  % 16
  {Exaltáre, Dómine, in virtúte \MediatioCII{tu}[a;] cantábimus et psallémus virtútes \TerminatioCII{tu}as. \Responsorium{In salutári tuo}[\LinkLA]}%
    {\InchoatioEV{Eu} o saciarei de longos \MediatioEV{di}[as] e lhe mostrarei a minha sal\TerminatioEV{va}{ção}. \Responsorium{Sob suas penas}[\LinkPT]}
}