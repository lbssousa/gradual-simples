% !TeX root = ../fasciculo-3/a4.tex
% chktex-file 1

\subsection{Entrada}\label{subsection:missae-vitivae/missa-votiva-de-sanctissima-eucharistia/introitus}
\MakeChantAntiphonPsalm{sacerdos-in-aeternum.1f}[1]{psalmi-ad-introitum}

\AllowPageFlush

\subsection[Salmo Responsorial]{Salmo Responsorial \textmd{E 5}}\label{subsection:missae-vitivae/missa-votiva-de-sanctissima-eucharistia/psalmus-responsorius}
\MakeChantPsalmTwoVerses{psalmi-responsorii}{exaltabo-te.E5}

\AllowPageFlush

\subsection{Aleluia}\label{subsection:missae-vitivae/missa-votiva-de-sanctissima-eucharistia/alleluia}
\MakeChantAlleluiaPsalm{3g}{attendite-popule-meus.3g}

\AllowPageFlush

\subsection[Salmo Aleluiático]{Salmo Aleluiático \textmd{C 4}}\label{subsection:missae-vitivae/missa-votiva-de-sanctissima-eucharistia/psalmus-alleluiaticus}
\begin{rubrica}
  O primeiro {\normalfont\Rbar} pode ser cantado apenas pelo grupo de cantores ou por todos. O segundo {\normalfont\Rbar} é cantado por todos.
\end{rubrica}
\MakeChantPsalmOneVerse{psalmi-alleluiatici}{attendite-popule-meus.C4}

\AllowPageFlush

\subsection{Ofertório}\label{subsection:missae-vitivae/missa-votiva-de-sanctissima-eucharistia/offertorium}
\MakeChantAntiphonPsalm{angelorum-esca.2D}[1]{psalmi-ad-offertorium}

\AllowPageFlush

\subsection{Comunhão}\label{subsection:missae-vitivae/missa-votiva-de-sanctissima-eucharistia/communio}
\MakeChantAntiphonPsalm{ego-sum-panis-vivus.1f}[1]{psalmi-ad-communionem}