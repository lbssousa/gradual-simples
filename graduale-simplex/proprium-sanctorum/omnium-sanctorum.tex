% chktex-file 1
\def\Prefix{subsection:proprium-sanctorum/omnium-sanctorum}

\SectionAntiphonPsalm{Entrada}[\Prefix/introitus]{diligite-dominum.2D}{Salmo 30(31)}{psalmi-ad-introitum}

\AllowPageFlush

\SectionPsalmMulti{Responsorial}[\Prefix/psalmus-responsorius]{Salmo 150}{psalmi-responsorii}{laudate-dominum-in-sanctis-eius.C3g}{
    {psalmus-v1}{psalmus-v1-pt},
    {psalmus-v2}{psalmus-v2-pt}
}

\AllowPageFlush

\SectionAlleluiaPsalm[\Prefix/alleluia]{8c.1}{Salmo 32(33)}{exsultate-iusti-in-domino.8c}

\AllowPageFlush

\SectionPsalm[\begin{rubrica} O primeiro {\normalfont\Rbar} pode ser cantado apenas pelo grupo de cantores ou por todos. O segundo {\normalfont\Rbar} é cantado por todos. \end{rubrica}]{Aleluiático}[\Prefix/psalmus-alleluiaticus]{Salmo 32(33)}{psalmi-alleluiatici}{exsultate-iusti-in-domino.C4}

\AllowPageFlush

\SectionAntiphonPsalm{Ofertório}[\Prefix/offertorium]{nomen-sempiternum.7a}{Salmo 5}{psalmi-ad-offertorium}

\AllowPageFlush

\SectionAntiphonPsalm{Comunhão}[\Prefix/communio]{beati-pacifici.1f}{Salmo 125(126)}{psalmi-ad-communionem}

\AllowPageBreak