\SetLinkTextL{\Responsorium{Allelúia}}
\SetLinkTextR{\Responsorium{Aleluia}}

\SetFirstVersePair{
  {\VSup{1,15}Ecce tu pulchra es, amíca mea, ecce tu \MediatioE{pul}[chra es:] óculi tui colum\TerminatioEII{bá}rum.}%
    {Como és bela, minha amada, como és \MediatioE{be}[la,] com teus olhos de \TerminatioEII{pom}ba!}
}

\SetVersePairs{
  {\VSup{2,11}Iam enim hiems \MediatioE{tráns}[iit,] imber ábiit et re\TerminatioEII{cés}sit.}%
    {O inverno pas\MediatioE{sou}[,] as chuvas cessaram e já se \TerminatioEII{fo}ram.},
  {\VSup{12}Flores apparuérunt in terra \MediatioE{no}[stra,] tempus putatiónis ad\TerminatioEII{vé}nit.}%
    {Apareceram as flores no \MediatioE{cam}[po,] chegou o tempo da \TerminatioEII{po}\-da.},
  {\VSup{13bc}Vox túrturis audíta est in terra \MediatioE{no}[stra,] víneæ floréntes dedérunt odórem \TerminatioEII{su}um.}%
    {A rola já faz ouvir seu canto em nossa \MediatioE{ter}[ra.] Soltam perfume as vinhas em \TerminatioEII{flor}.},
  {\VSup{14bc}Sonet vox tua in áuribus \MediatioE{me}[is:] vox enim tua dulcis, et fácies tua de\TerminatioEII{có}ra.}%
    {A tua voz ressoe aos meus ou\MediatioE{vi}[dos,] pois a tua voz é suave e o teu rosto é \TerminatioEII{lin}do!},
  {\VSup{4,11}Favus distíllans lábia tua \MediatioE{spon}[sa,] mel et lac sub língua \TerminatioEII{tu}a.}%
    {Teus lábios, minha esposa, são favo que destila o \MediatioE{mel}[;] sob a tua língua há mel e \TerminatioEII{lei}te.},
  {Odor vestimentórum tu\MediatioE{ó}[rum] super ómnia a\TerminatioEII{ró}\-ma\-ta.}%
    {O perfume de tuas \MediatioE{ves}[tes] é como o perfume do \TerminatioEII{Lí}ba\-no.},
  {\VSup{12}Hortus conclúsus, soror mea, \MediatioE{spon}[sa,] hortus conclúsus, fons si\TerminatioEII{gná}tus.}%
    {És um jardim fechado, minha irmã e es\MediatioE{po}[sa,] jardim fechado e fonte la\TerminatioEII{cra}da.},
  {\VSup{5,2b}Aperi mihi, soror mea, amíca \MediatioE{me}[a,] colúmba mea, immaculáta \MediatioE{me}a.}%
    {Abre-me, ó minha irmã e a\MediatioE{ma}[da,] minha pomba, minha imacu\TerminatioEII{la}da.},
  {\VSup{6,4}Pulchra es, amíca \Flexa*{me}[a,] suávis et decóra sicut Ie\MediatioE{\-rú}[\-sa\-lem,] terríbilis ut castrórum ácies ordi\TerminatioEII{ná}ta.}%
    {Tu és bela, minha a\Flexa*{ma}[da,] formosa como Jerusa\MediatioE{\-lém}[,] terrível como um exército em ordem de ba\TerminatioEII{ta}\-lha.}
}