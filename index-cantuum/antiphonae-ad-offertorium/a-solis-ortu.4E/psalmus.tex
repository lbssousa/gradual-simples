% !TeX root = ../../../a4.tex
\DeclareDocumentCommand{\Annotation}{}{
    \MakeAnnotation{\CantusID{020003}[Sl 113(112),3]}{Sl 113(112),1ab--2.4--6}
}

\def\LinkLA{tempus-per-annum/missa-8/offertorium/antiphona}
\def\LinkPT{\LinkLA-pt}

\def\VersePairs{
    % 2
    {\Inchoatio{Sit}[ ]{no}men Dómini \MediatioIV{be}{ne}{díc}[tum] ex hoc nunc et \TerminatioIV{us}{que}[ ]{in}[ ]{sǽcu}lum. \Antiphona{A solis ortu}[\LinkLA]}%
        {\Inchoatio{Se}{ja} bendito o nome \MediatioIV{do}[ ]{Se}{nhor}[,] desde agora \TerminatioIV{e}[ ]{pa}{ra}[ ]{sem}pre. \Antiphona{De ond\Elisio{e} o sol nasce}[\LinkPT]},
    % 4
    {\Inchoatio{Ex}{cél}sus super omnes \MediatioVII{gen}{tes }{Dó}[minus,] super cælos \TerminatioVII{gló}{\-ri\-a }{e}ius. \Antiphona{A solis ortu}[\LinkLA]}%
        {\Inchoatio{Ex}{cel}so é o Senhor, sobre todas \MediatioIV{as}[ ]{na}{ções}[,] a sua glória está a\TerminatioIV{ci}{ma}[ ]{dos}[ ]{céus}. \Antiphona{De ond\Elisio{e} o sol nasce}[\LinkPT]},
    % 5--6
    {\Inchoatio{Quis}[ ]{si}cut Dóminus Deus noster, qui in \MediatioVII{al}{tis }{há}[\-bi\-tat,] et humília réspicit in cælo \TerminatioVII{et}{ in }{ter}ra? \Antiphona{A solis ortu}[\LinkLA]}%
        {\Inchoatio{Quem}[ ]{é} como o Senhor, o nosso Deus, que mora \MediatioIV{nas}[ ]{al}{tu}[ras] e se inclina, para olhar sobre o céu e \TerminatioIV{so}{bre}[ ]{a}[ ]{ter}ra? \Antiphona{De ond\Elisio{e} o sol nasce}[\LinkPT]}
}