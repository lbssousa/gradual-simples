% !TeX root = ../../main.tex
% chktex-file 1
\def\Prefix{proprium-sanctorum/sanctorum-petri-et-pauli-apostolorum/}

\subsection{Entrada}\label{subsection:proprium-sanctorum/sanctorum-petri-et-pauli-apostolorum/introitus}
\MakeChantAntiphonPsalm{\Prefix}{introitus/}

\subsection[Salmo Responsorial]{Salmo Responsorial \textmd{C 3 g}}\label{subsection:proprium-sanctorum/sanctorum-petri-et-pauli-apostolorum/psalmus-responsorius}
\MakeChantPsalmThreeVerses{\Prefix}{psalmus-responsorius/}

\AllowPageFlush

\subsection{Aleluia}\label{subsection:proprium-sanctorum/sanctorum-petri-et-pauli-apostolorum/alleluia}
\MakeChantAntiphonPsalm{\Prefix}{alleluia/}

\subsection[Salmo Aleluiático]{Salmo Aleluiático \textmd{C 4}}\label{subsection:proprium-sanctorum/sanctorum-petri-et-pauli-apostolorum/psalmus-alleluiaticus}
\begin{rubrica}
  O primeiro {\normalfont\Rbar} pode ser cantado apenas pelo grupo de cantores ou por todos. O segundo {\normalfont\Rbar} é cantado por todos.
\end{rubrica}
\MakeChantPsalmOneVerse{\Prefix}{psalmus-alleluiaticus/}

\AllowPageFlush

\subsection{Ofertório}\label{subsection:proprium-sanctorum/sanctorum-petri-et-pauli-apostolorum/offertorium}
\MakeChantAntiphonPsalm{\Prefix}{offertorium/}

\AllowPageBreak

\subsection{Comunhão}\label{subsection:proprium-sanctorum/sanctorum-petri-et-pauli-apostolorum/communio}
\MakeChantAntiphonPsalm{\Prefix}{communio/}