\SetVerseAnnotation{Sl 9,2--3.8b--9a.10--11.14--15.19.10,17(9,38)}

\SetLinkTextL{\Responsorium{Allelúia}}
\SetLinkTextR{\Responsorium{Aleluia}}

\def\VersePairs{
  % 3
  {Lætábor et exsultábo \MediatioE{in}[ te,] psallam nómini tuo, Al\TerminatioEII{tís}\-si\-me.}%
    {Eu me alegro e exultarei em \MediatioE{vós}[,] cantarei salmos ao vosso nome, ó Al\TerminatioEII{tís}simo.},
  % 8b--9a
  {Parávit in iudícium thronum \MediatioE{su}[um,] et ipse iudicábit orbem terræ in iu\TerminatioEII{stí}tia.}%
    {Preparado o seu trono para o julga\MediatioE{men}[to,] ele mes\-mo julgará o mundo com jus\TerminatioEII{ti}ça.},
  % 10
  {Et erit Dóminus refúgium op\MediatioE{prés}[so,] refúgium in opportunitátibus, in tribulati\TerminatioEII{ó}ne.}%
    {O Senhor será refúgio para o opri\MediatioE{mi}[do,] refúgio nas necessidades, na tribula\TerminatioEII{ção}.},
  % 11
  {Et spérent in te qui novérunt nomen \MediatioE{tu}[um,] quóniam non dereliquísti quæréntes te, \TerminatioEII{Dó}mine.
     }%
    {Em vós esperem os que conhecem o vosso \MediatioE{no}[me,] pois não abandonais os que vos procuram, Se\TerminatioEII{nhor}.},
  % 14
  {Miserére mei, \Flexa*{Dó}[mine;] vide affiictiónem meam de
      inimícis \MediatioE{me}[is,] qui exáltas me de portis \TerminatioEII{mor}tis.}%
    {Tende piedade de mim, Se\Flexa*{nhor}[,] vede minha aflição por causa dos meus ini\MediatioE{mi}[gos.] Sois vós quem me arranca das portas da \TerminatioEII{mor}te.},
  % 15
  {Ut annúntiem omnes laudatiónes tuas in portis fíliæ
      \MediatioE{Si}[on,] exsúltem in salutári \TerminatioEII{tu}o.}%
    {Para que eu anuncie todos os vossos louvores nas portas da filha de Si\MediatioE{ão}[,] e me alegre com a vossa salva\TerminatioEII{ção}.},
  % 19
  {Quóniam non in finem oblívio erit \MediatioE{páu}[peris;] exspectátio páuperum non períbit in æ\TerminatioEII{tér}num.
     }%
    {Do pobre, porém, não haverá esquecimento até o \MediatioE{fim}[;] a esperança dos pobres não será frustrada para \TerminatioEII{sem}pre.},
  % 38 (10,17)
  {Desidérium páuperum exaudísti, \MediatioE{Dó}[mine:] confirmábis cor eórum, inténdes aurem \TerminatioEII{tu}am.}%
    {Adentestes, Senhor, o desejo dos \MediatioE{po}[bres;] confirmareis seu coração, e lhes dareis ou\TerminatioEII{vi}do.}
}