% !TeX root = ../../../a4.tex
\DeclareDocumentCommand{\Annotation}{}{
  \SetVerseAnnotation{\CantusID{002846}[Sl 30(29),11]}{Sl 30(29),2--3.10cd--11}
}

\def\LinkLA{tempus-quadragesimae/feria-iv-cinerum/offertorium/antiphona}
\def\LinkPT{\LinkLA-pt}

\def\VersePairs{
  % 3
  {\Inchoatio{Dó}{mi}ne Deus meus, cla\MediatioI{má}{vi }{ad}[ te,] \TerminatioI{et}[ ]{sa}{nás}ti me. \Antiphona{Factus est}[\LinkLA]}%
    {\Inchoatio{Se}{nhor}, meu Deus, a \MediatioI{vós}{ cla}{mei} e \TerminatioI{me}[ ]{cu}{ras}tes. \Antiphona{Meu Deus}[\LinkPT]},
  % 10cd
  {\Inchoatio{Num}{quid} confitébitur \MediatioI{ti}{bi }{pul}[vis,] aut annuntiábit veri\TerminatioI{tá}{tem}[ ]{tu}am? \Antiphona{Factus est}[\LinkLA]}%
    {\Inchoatio{A}{ca}so o pó \MediatioI{vai}{ lou}{var}[-vos] e proclamar a vossa fi\TerminatioI{de}{li}{da}de? \Antiphona{Meu Deus}[\LinkPT]},
  % 11
  % Nota: o texto original da Bíblia Oficial da CNBB está no modo verbal 
  %       indicativo, mas aqui nós mudamos a conjugação verbal para o modo
  %       imperativo, em conformidade com o texto correspondente
  %       no original em latim.
  {\Inchoatio{Au}{di}, Dómine, et mise\MediatioI{ré}{re }{me}[i,] Dómine, esto mi\TerminatioI{hi}[ ]{ad}{iú}tor. \Antiphona{Factus est}[\LinkLA]}%
    {\Inchoatio{Ou}{vi}, Senhor, e tende compai\MediatioI{xão}{ de }{mim}[;] Senhor, tor\-nai-vos o meu \TerminatioI{pro}{te}{tor}. \Antiphona{Meu Deus}[\LinkPT]}
}