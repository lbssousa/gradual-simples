% chktex-file 1
\subsection{Antífona de Entrada}\label{subsection:tempus-nativitatis/in-epiphania-domini/introitus}
\MakeChantAntiphon{venite-adoremus-eum.8G}
\nobreaksubsection{Salmo 94(95)}
\MakeChantPsalm{psalmi-ad-introitum}{venite-adoremus-eum.8G}

\AllowPageFlush

\subsection[Salmo Responsorial]{Salmo Responsorial}\label{subsection:tempus-nativitatis/in-epiphania-domini/psalmus-responsorius}
\subsubsection{Salmo 71(72)}
\MakeChantPsalmMulti{psalmi-responsorii}{adorabunt-eum.E5}{
  {psalmus-v1}{psalmus-v1-pt},
  {psalmus-v2}{psalmus-v2-pt}
}

\AllowPageFlush

\subsection{Aleluia}\label{subsection:tempus-nativitatis/in-epiphania-domini/alleluia}
\MakeChantAlleluia{3g}
\nobreaksubsection{Salmo 28(29)}
\MakeChantPsalm{psalmi-ad-alleluia}{afferte-domino.3g}

\AllowPageFlush

\subsection[Salmo Aleluiático]{Salmo Aleluiático}\label{subsection:tempus-nativitatis/in-epiphania-domini/psalmus-alleluiaticus}
\subsubsection{Salmo 28(29)}
\begin{rubrica}
  O primeiro {\normalfont\Rbar} pode ser cantado apenas pelo grupo de cantores ou por todos. O segundo {\normalfont\Rbar} é cantado por todos.
\end{rubrica}
\MakeChantPsalm{psalmi-alleluiatici}{afferte-domino.C4}

\AllowPageFlush

\subsection{Antífona de Ofertório}\label{subsection:tempus-nativitatis/in-epiphania-domini/offertorium}
\MakeChantAntiphon{reges-tharsis.1a2}
\nobreaksubsection{Salmo 71(72)}
\MakeChantPsalm{psalmi-ad-offertorium}{reges-tharsis.1a2}

\AllowPageBreak

\subsection{Antífona de Comunhão}\label{subsection:tempus-nativitatis/in-epiphania-domini/communio}
\MakeChantAntiphon{vidimus-stellam-eius.4E}
\nobreaksubsection{Salmo 95(96)}
\MakeChantPsalm{psalmi-ad-communionem}{vidimus-stellam-eius.4E}

\AllowPageFlush