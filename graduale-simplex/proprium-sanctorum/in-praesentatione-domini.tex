% !TeX root = ../../a4.tex
% chktex-file 1

\subsection{Antífona para a benção das velas}\label{subsection:proprium-sanctorum/in-praesentatione-domini/ad-benedictionem-candelarum}

\iftoggle{onecolumn}{%
  \GregorioImportL{index-cantuum/antiphonae}{ecce-dominus-noster.3}{antiphona}{antiphona-pt}

  \nobreaksubsection{Convite para a procissão}

  \begin{rubrica}
    \begin{center}
      O díacono ou o celebrante diz:
    \end{center}
  \end{rubrica}

  \vspace{-20pt}

  \GregorioImportL{index-cantuum/varia}{procedamus-in-pace}{versum}{versum-pt}

  \begin{rubrica}
    \begin{center}
      Todos respondem:
    \end{center}
  \end{rubrica}

  \vspace{-20pt}

  \GregorioImportL{index-cantuum/varia}{procedamus-in-pace}{responsorium}{responsorium-pt}

  \clearpage

  \GregorioImportR{index-cantuum/antiphonae}{ecce-dominus-noster.3}{antiphona}{antiphona-pt}

  \nobreaksubsection{Convite para a procissão}

  \begin{rubrica}
    \begin{center}
      O díacono ou o celebrante diz:
    \end{center}
  \end{rubrica}

  \vspace{-20pt}

  \GregorioImportR{index-cantuum/varia}{procedamus-in-pace}{versum}{versum-pt}

  \begin{rubrica}
    \begin{center}
      Todos respondem:
    \end{center}
  \end{rubrica}

  \vspace{-20pt}

  \GregorioImportR{index-cantuum/varia}{procedamus-in-pace}{responsorium}{responsorium-pt}
}{%
  \begin{paracol}{2}
    \GregorioImportLR{index-cantuum/antiphonae}{ecce-dominus-noster.3}{antiphona}{antiphona-pt}
  \end{paracol}

  \nobreaksubsection{Convite para a procissão}

  \begin{rubrica}
    \begin{center}
      O díacono ou o celebrante diz:
    \end{center}
  \end{rubrica}

  \vspace{-20pt}

  \begin{paracol}{2}
    \GregorioImportLR{index-cantuum/varia}{procedamus-in-pace}{versum}{versum-pt}
  \end{paracol}

  \begin{rubrica}
    \begin{center}
      Todos respondem:
    \end{center}
  \end{rubrica}

  \vspace{-20pt}

  \begin{paracol}{2}
    \GregorioImportLR{index-cantuum/varia}{procedamus-in-pace}{responsorium}{responsorium-pt}
  \end{paracol}
}

\AllowPageFlush

\subsection{Procissão I}\label{subsection:proprium-sanctorum/in-praesentatione-domini/ad-processionem-1}
\MakeChantAntiphonPsalm{lumen-ad-revelationem.8G}{psalmi-ad-processionem}

\AllowPageFlush

\subsection{Procissão II}\label{subsection:proprium-sanctorum/in-praesentatione-domini/ad-processionem-2}
\MakeChantAntiphonPsalm{obtulerunt.8G}{psalmi-ad-processionem}

\subsection{Procissão III}
\begin{rubrica}
  Ver Solenidade de São José, Esposo da Virgem Maria, página~\pageref{subsection:proprium-sanctorum/sancti-ioseph-sponsi-bmv/psalmus-responsorius}.
\end{rubrica}

\AllowPageFlush

\subsection{Entrada}\label{subsection:proprium-sanctorum/in-praesentatione-domini/introitus}
\MakeChantAntiphonPsalm{suscepimus-deus.8G}{psalmi-ad-introitum}[suscepimus-deus.8G.1/]

\nobreaksubsection{Salmo Responsorial}

\begin{rubrica}
  Ver Solenidade da Imaculada Conceição de Nossa Senhora, página~\pageref{subsection:proprium-sanctorum/in-conceptione-immaculata-bmv/psalmus-responsorius}.
\end{rubrica}

\subsection{Aleluia}\label{subsection:proprium-sanctorum/in-praesentatione-domini/alleluia}
\MakeChantAlleluiaPsalm{4E.1}{fundamenta-eius.4E}

\AllowPageFlush

\subsection[Salmo Aleluiático]{Salmo Aleluiático \textmd{E 2 d}}\label{subsection:proprium-sanctorum/in-praesentatione-domini/psalmus-alleluiaticus}
\MakeChantPsalmOneVerse{psalmi-alleluiatici}{fundamenta-eius.E2d}

\AllowPageFlush

\subsection{Ofertório}\label{subsection:proprium-sanctorum/in-praesentatione-domini/offertorium}
\MakeChantAntiphonPsalm{diffusa-est-gratia.1g4}{psalmi-ad-offertorium}

\AllowPageFlush

\subsection{Comunhão}\label{subsection:proprium-sanctorum/in-praesentatione-domini/communio}
\MakeChantAntiphonPsalmExtended{responsum-accepit-simeon.7a}{psalmi-ad-communionem}{%
  \begin{rubrica}
    Pode-se complementar com um cântico de comunhão à escolha do fascículo VI.
  \end{rubrica}%
}