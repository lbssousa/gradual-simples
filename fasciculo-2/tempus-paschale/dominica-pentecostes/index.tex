% !TeX root = ../../a4.tex
% chktex-file 1
\def\Prefix{tempus-paschale/dominica-pentecostes}

\subsection{Entrada}\label{subsection:tempus-paschale/dominica-pentecostes/introitus}
\MakeChantAntiphonPsalm{\Prefix}{introitus}

\subsection{Entrada (opcional)}\label{subsection:tempus-paschale/dominica-pentecostes/introitus-ad-libitum}
\MakeChantAntiphonPsalm{\Prefix}{introitus-ad-libitum}

\subsection{Salmo Aleluiático I}\label{subsection:tempus-paschale/dominica-pentecostes/psalmus-alleluiaticus-1}
\begin{annotation}
    \textcolor{gregoriocolor}{\Vbar.} Sl 68(67),2--3.5.10.27.29--30.33--34a.35a
\end{annotation}
\MakeChantLongPsalm*{\Prefix}{psalmus-alleluiaticus-1}{
    {psalmus-v01}{psalmus-v01-pt},
    {psalmus-v02}{psalmus-v02-pt},
    {psalmus-v03}{psalmus-v03-pt},
    {psalmus-v04}{psalmus-v04-pt},
    {psalmus-v05}{psalmus-v05-pt},
    {psalmus-v06}{psalmus-v06-pt},
    {psalmus-v07}{psalmus-v07-pt},
    {psalmus-v08}{psalmus-v08-pt},
    {psalmus-v09}{psalmus-v09-pt},
    {psalmus-v10}{psalmus-v10-pt}
}

\subsection{Sequência}\label{subsection:tempus-paschale/dominica-pentecostes/sequentia}
\begin{annotation}
    Missal Romano, 2ª Edição Típica (CNBB)
\end{annotation}
\MakeChantLongPsalm{\Prefix}{sequentia}{
    {sequentia}{sequentia-pt}
}

\subsection{Salmo Aleluiático II}\label{subsection:tempus-paschale/dominica-pentecostes/psalmus-alleluiaticus-2}
\MakeChantPsalmOneVerse{\Prefix}{psalmus-alleluiaticus-2}

\subsection{Ofertório}\label{subsection:tempus-paschale/dominica-pentecostes/offertorium}
\MakeChantAntiphonPsalm{\Prefix}{offertorium}

\subsection{Comunhão}\label{subsection:tempus-paschale/dominica-pentecostes/communio}
\MakeChantAntiphonPsalm{\Prefix}{communio}

\subsection{Comunhão (opcional)}\label{subsection:tempus-paschale/dominica-pentecostes/communio-ad-libitum}
\MakeChantAntiphonPsalm{\Prefix}{communio-ad-libitum}