% !TeX root = ../../../main.tex
\DeclareDocumentCommand{\Annotation}{}{
  \MakeAnnotation{\CantusID{002235}[Sl 31(30),24]}{Sl 31(30)2bc--4.20--22.25}
}

\def\LinkLA{proprium-sanctorum/omnium-sanctorum/introitus/antiphona}
\def\LinkPT{\LinkLA-pt}

\def\VersePairs{
  % 3ab
  {\Inchoatio{In}{clí}na ad me aurem \MediatioII{tu}[am,] accélera, ut é\TerminatioII{ru}{as} me. \Antiphona{Dilígete Dóminum}[\LinkLA]}%
    {\Inchoatio{In}{cli}nai para mim o vosso ou\MediatioII{vi}[do,] apressai-vos para \TerminatioII{li}{vrar}-me. \Antiphona{Amai o Senhor}[\LinkPT]},
  % 3cd
  {\Inchoatio{E}{sto} mihi in rupem præsídii et in domum mu\MediatioII{ní}[tam,] ut salvum \TerminatioII{me}[ ]{fá}cias. \Antiphona{Dilígete Dóminum}[\LinkLA]}%
    {\Inchoatio{Se}{de} para mim o rochedo onde me a\MediatioII{bri}[gue,] a casa fortificada, para que me \TerminatioII{sal}{veis}. \Antiphona{Amai o Senhor}[\LinkPT]},
  % 4
  {\Inchoatio{Quó}{ni}am fortitúdo mea et refúgium meum \MediatioII{es}[ tu,] et propter nomen tuum dedúces me \TerminatioII{et}[ ]{pa}sces me. \Antiphona{Dilígete Dóminum}[\LinkLA]}%
    {\Inchoatio{Pois}[ ]{a} minha fortaleza e meu refúgio sois \MediatioII{vós}[,] por causa do vosso nome me guiareis e me apascen\TerminatioII{ta}{reis}. \Antiphona{Amai o Senhor}[\LinkPT]},
  % 20ab
  {\Inchoatio{Quam}[ ]{ma}gna multitúdo dulcédinis tuæ, \MediatioII{Dó}[mine,] quam abscondísti timén\TerminatioII{ti}{bus} te. \Antiphona{Dilígete Dóminum}[\LinkLA]}%
    {\Inchoatio{Co}{mo} é grande vossa bondade, Se\MediatioII{nhor}[,] que reservastes aos que \TerminatioII{vos}[ ]{te}mem. \Antiphona{Amai o Senhor}[\LinkPT]},
  % 20cd
  {\Inchoatio{Per}{fe}císti eis qui sperant \MediatioII{in}[ te,] in conspéctu filió\TerminatioII{rum}[ ]{hó}{mi}num. \Antiphona{Dilígete Dóminum}[\LinkLA]}%
    {\Inchoatio{Vós}[ ]{a} consumastes para os que esperam em \MediatioII{vós}[,] à vista dos filhos \TerminatioII{dos}[ ]{ho}mens. \Antiphona{Amai o Senhor}[\LinkPT]},
  % 21ab
  {\Inchoatio{Ab}{scón}des eos in abscóndito faciéi \MediatioII{tu}[æ] a conturbatió\TerminatioII{ne}[ ]{hó}minum. \Antiphona{Dilígete Dóminum}[\LinkLA]}%
    {\Inchoatio{Vós}[ ]{os} escondereis no segredo da vossa \MediatioII{fa}[ce,] longe das intrigas \TerminatioII{hu}{ma}nas. \Antiphona{Amai o Senhor}[\LinkPT]},
  % 21cd
  {\Inchoatio{Pró}{te}ges eos in taber\MediatioII{ná}[culo] a contradictióne linguárum. \Antiphona{Dilígete Dóminum}[\LinkLA]}%
    {\Inchoatio{Na}[ ]{vos}sa tenda os protege\MediatioII{reis}[,] longe das línguas acu\TerminatioII{\-sa}{\-do}\-ras. \Antiphona{Amai o Senhor}[\LinkPT]},
  % 22 
  {\Inchoatio{Be}{ne}díctus \MediatioII{Dó}[minus,] quóniam mirificávit misericórdiam suam mihi in civitáte \TerminatioII{mu}{ní}ta. \Antiphona{Dilígete Dóminum}[\LinkLA]}%
    {\Inchoatio{Ben}{di}to seja o Se\MediatioII{nhor}[,] porque maravilhosamente mos\-trou-me a sua misericórdia, numa cidade forti\TerminatioII{fi}{ca}da. \Antiphona{Amai o Senhor}[\LinkPT]},
  % 25
  {\Inchoatio{Vi}{rí}liter ágite, et confortétur cor \MediatioII{ve}[strum,] omnes qui sperátis \TerminatioII{in}[ ]{Dó}mino. \Antiphona{Dilígete Dóminum}[\LinkLA]}
    {\Inchoatio{Se}{de} corajosos e fortaleça-se o vosso cora\MediatioII{ção}[,] todos vós que esperais no \TerminatioII{Se}{nhor}. \Antiphona{Amai o Senhor}[\LinkPT]}
}