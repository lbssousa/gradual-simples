% !TeX root = ../../main.tex
% chktex-file 1
\def\Prefix{tempus-paschale/in-ascencione-domini/}

\subsection{Entrada}\label{subsection:tempus-paschale/in-ascencione-domini/introitus}
\MakeChantAntiphonPsalm{\Prefix}{introitus/}

\subsection{Salmo Aleluiático I}\label{subsection:tempus-paschale/in-ascencione-domini/psalmus-alleluiaticus-1}
\begin{annotation}
  \textcolor{gregoriocolor}{\Vbar.} Sl 47(46)
\end{annotation}
\MakeChantLongPsalm*{\Prefix}{psalmus-alleluiaticus-1/}{
  {psalmus-v01}{psalmus-v01-pt},
  {psalmus-v02}{psalmus-v02-pt},
  {psalmus-v03}{psalmus-v03-pt},
  {psalmus-v04}{psalmus-v04-pt},
  {psalmus-v05}{psalmus-v05-pt},
  {psalmus-v06}{psalmus-v06-pt},
  {psalmus-v07}{psalmus-v07-pt},
  {psalmus-v08}{psalmus-v08-pt},
  {psalmus-v09}{psalmus-v09-pt},
  {psalmus-v10}{psalmus-v10-pt}
}

\subsection{Salmo Aleluiático II}\label{subsection:tempus-paschale/in-ascencione-domini/psalmus-alleluiaticus-2}
\MakeChantPsalmOneVerse{\Prefix}{psalmus-alleluiaticus-2/}

\subsection{Ofertório}\label{subsection:tempus-paschale/in-ascencione-domini/offertorium}
\MakeChantAntiphonPsalm{\Prefix}{offertorium/}

\subsection{Comunhão}\label{subsection:tempus-paschale/in-ascencione-domini/communio-1}
\MakeChantAntiphonPsalm{\Prefix}{communio/}