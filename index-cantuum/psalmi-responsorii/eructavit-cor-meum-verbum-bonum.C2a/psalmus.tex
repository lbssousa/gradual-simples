\SetVerseAnnotation{Sl 45(44),1.3b.8--15}

\SetLinkTextL{\Responsorium{Ópera mea}}
\SetLinkTextR{\Responsorium{Ó Rei}}

\SetFirstVersePair{
  {}%
  {Transborda um poema do meu cora\MediatioC{ção}[:] vou can\TerminatioCII{tar}-vos.}
}

\SetVersePairs{
  % 3bc
  {Diffúsa est grátia in lábiis \MediatioC{tu}[is,] proptérea benedíxit te Deus in æ\TerminatioCII{ter}num.}%
    {Vossos lábios espalham a graça, o en\MediatioC{can}[to,] porque Deus, para sempre, vos deu sua \TerminatioCII{bên}ção.},
  % 8
  {Dilexísti iustítiam et odísti iniqui\Flexa*{tá}[tem,] proptérea unxit te Deus, Deus \MediatioC{tu}[us,] oleo lætítiæ præ consórtibus \TerminatioCII{tu}is.}%
    {Vós amais a justiça e odiais a mal\Flexa*{da}[de.] É por isso que Deus vos ungiu com seu \MediatioC{ó}[leo,] deu-vos mais alegria que aos vossos a\TerminatioCII{mi}gos.},
  % 9
  {Myrrha et áloe et cásia ómnia vestiménta \MediatioC{tu}[a,] e dómibus ebúrneis chordæ delecta\TerminatioCII{vérunt} te.}%
    {Vossas vestes exalam preciosos per\MediatioC{fu}[mes.] De ebúrneos palácios os sons vos de\TerminatioCII{lei}tam.},
  % 10
  {Fíliæ regum inter honorátas \MediatioC{tu}[as;] ástitit regína a dextris tuis ornáta auro ex \TerminatioCII{O}phir.}%
    {As filhas de reis vêm ao vosso en\Flexa*{con}[tro,] e à vossa direita se encontra a ra\MediatioC{i}[nha] com veste esplendente de ouro de O\TerminatioCII{fir}.},
  % 11
  {Audi, fília, et \Flexa*{vi}[de,] et inclína aurem \MediatioC{tu}[am,] et oblivíscere pópulum tuum et domum patris \TerminatioCII{tu}i.}%
    {Escutai, minha filha, olhai, ouvi \MediatioC{is}[to:] Esquecei vosso povo e a casa pa\TerminatioCII{ter}na!},
  % 12
  {Et concupíscet rex spéciem \MediatioC{tu}[am,] quóniam ipse est dóminus tuus, et adóra \TerminatioCII{e}um.}%
    {Que o Rei se encante com vossa be\MediatioC{le}[za!] Prestai-lhe homenagem: é vosso Se\TerminatioCII{nhor}!},
  % 13
  {Fíliæ Tyri cum mu\MediatioC{né}[ribus;] vultum tuum deprecabúntur dívites \TerminatioCII{ple}bis.}%
    {O povo de Tiro vos traz seus pre\MediatioC{sen}[tes,] os grandes do povo vos pedem fa\TerminatioCII{vo}res.},
  % 14
  {Gloriósa nimis fília regis in\MediatioC{trín}[secus,] textúris áureis circuma\TerminatioCII{mí}cta.}%
    {Majestosa, a princesa real vem che\MediatioC{gan}[do,] vestida de ricos brocados de \TerminatioCII{ou}ro.},
  % 15
  {In véstibus variegátis adducétur \MediatioC{re}[gi;] vírgines post eam, próximæ eius, afferúntur \TerminatioCII{ti}bi.}%
    {Em vestes vistosas ao Rei se di\MediatioC{ri}[ge,] e as virgens amigas lhe formam cor\TerminatioCII{te}jo.}
}