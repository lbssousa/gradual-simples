\SetVerseAnnotation{Sl 5,2.4--5a.9}

\SetLinkTextL{\Antiphona{Inténde}}
\SetLinkTextR{\Antiphona{Atendei}}

\SetVersePairs{
  {\VSup{4ab}\Inchoatio{Quó}{ni}am ad te orábo, \MediatioVIII{Dó}[mine,] mane exáudies \TerminatioVIII{vo}{cem}[ ]{me}am.}%
    {\VSup{4ab}\Inchoatio{É}[ ]{a} vós que eu dirijo a minha \MediatioVIII{pre}[ce,] e de manhã \TerminatioVIII{já}[ ]{me}[ ]{es}cutais!},
  {\VSup{4a}\Inchoatio{Ma}{ne} astábo tibi et exspec\MediatioVIII{tá}[bo,] \VSup{5a}quóniam non Deus volens iniqui\TerminatioVIII{tá}{tem}[ ]{tu} es.}%
    {\VSup{4cd}\Inchoatio{Des}{de} cedo eu me preparo \Flexa{pa}[ra vós,] e permaneço à vossa es\MediatioVIII{pe}[ra.] \VSup{5a}Não sois um Deus a quem agrade a i\TerminatioVIII{ni}{qui}{da}de.},
  {\VSup{9}\Inchoatio{Dó}{mi}ne, deduc me in iustítia tua propter inimícos \MediatioVIII{me}[os,] diríge in conspéctu meo \TerminatioVIII{vi}{am}[ ]{tu}am.}%
    {\VSup{9}\Inchoatio{Que}[ ]{me} possa conduzir vossa jus\MediatioVIII{ti}[ça,] por causa do \TerminatioVIII{i}{ni}{mi}go! \Sequitur{À} minha frente aplainai vosso ca\MediatioVIII{mi}[\-nho,] e gui\TerminatioVIII{ai}[ ]{meu}[ ]{ca}minhar!}
}