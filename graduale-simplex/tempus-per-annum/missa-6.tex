% !TeX root = ../fasciculo-3/a4.tex
% chktex-file 1

\subsection{Entrada}\label{subsection:tempus-per-annum/missa-6/introitus}
\MakeChantAntiphonPsalm{da-pacem-domine.2D}{introitus}

\nobreaksubsection{Salmo Responsorial}
\begin{rubrica}
    Ver Missa II, página~\pageref{subsection:tempus-per-annum/missa-2/psalmus-responsorius}, ou Missa III, página~\pageref{subsection:tempus-per-annum/missa-3/psalmus-responsorius}.
\end{rubrica}

\nobreaksubsection{Aleluia}
\begin{rubrica}
    Ver Missa II, página~\pageref{subsection:tempus-per-annum/missa-2/alleluia}, ou Missa III, página~\pageref{subsection:tempus-per-annum/missa-3/alleluia}.
\end{rubrica}

\nobreaksubsection{Salmo Aleluiático}
\begin{rubrica}
    Ver Missa II, página~\pageref{subsection:tempus-per-annum/missa-2/psalmus-alleluiaticus}, ou Missa III, página~\pageref{subsection:tempus-per-annum/missa-3/psalmus-alleluiaticus}.
\end{rubrica}

\AllowPageFlush

\subsection{Ofertório}\label{subsection:tempus-per-annum/missa-6/offertorium}
\MakeChantAntiphonPsalm{benefac-domine.8c}{offertorium}

\AllowPageFlush

\subsection{Comunhão}\label{subsection:tempus-per-annum/missa-6/communio}
\MakeChantAntiphonPsalm{adiuva-me.8G}{communio}