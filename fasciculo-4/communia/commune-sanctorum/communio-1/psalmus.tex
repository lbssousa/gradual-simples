% !TeX root = ../../../main.tex
\DeclareDocumentCommand{\Annotation}{}{
  \MakeAnnotation{\CantusID{000000}[Mt 24,45]}{Sl 34(33),2--11}
}

\def\LinkLA{communia/commune-sanctorum/communio-1/antiphona}
\def\LinkPT{\LinkLA-pt}

\def\VersePairs{
  % 3
  {\Inchoatio{Dó}{mi}no gloriábitur \MediatioIII{á}{nima }{me}[a,] áudiant mansuéti et \TerminatioIIIa{læ}{tén}tur. \Antiphona{Fidélis servus et prudens}[\LinkLA]}%
    {\Inchoatio{No}[ ]{Se}nhor se gloria \MediatioIII{mi}{nha }{al}[ma,] Ouçam os humildes e se \TerminatioIIIa{a}{le}grem. \Antiphona{Servo fiel e prudente}[\LinkPT]},
  % 4
  {\Inchoatio{Ma}{gni}ficáte \MediatioIII{Dó}{minum }{me}[cum,] et exaltémus nomen eius in \TerminatioIIIa{id}{íp}sum. \Antiphona{Fidélis servus et prudens}[\LinkLA]}%
    {\Inchoatio{En}{gran}decei comigo \MediatioIII{o}{ Se}{nhor}[,] e exaltemos todos juntos o \TerminatioIIIa{seu}[ ]{no}me. \Antiphona{Servo fiel e prudente}[\LinkPT]},
  % 5
  {\Inchoatio{Ex}{qui}sívi Dóminum, et \MediatioIII{ex}{au}{dívit}[ me,] et ex ómnibus terróribus meis erí\TerminatioIIIa{pu}{it} me. \Antiphona{Fidélis servus et prudens}[\LinkLA]}%
    {\Inchoatio{Bus}{quei} o Senhor, e ele \MediatioIII{res}{pon}{deu}[-me,] e de todos os meus temores me \TerminatioIIIa{li}{vrou}. \Antiphona{Servo fiel e prudente}[\LinkPT]},
  % 6
  {\Inchoatio{Re}{spí}cite ad eum, et il\MediatioIII{lu}{mi}{námi}[ni,] et fácies ve\-stræ non con\TerminatioIIIa{fun}{dén}tur. \Antiphona{Fidélis servus et prudens}[\LinkLA]}%
    {\Inchoatio{O}{lhai} para ele e vos tornareis i\MediatioIII{lu}{mi}{na}[dos,] e vossas faces não se cobrirão de \TerminatioIIIa{ver}{go}nha. \Antiphona{Servo fiel e prudente}[\LinkPT]},
  % 14
  {\Inchoatio{I}{ste} pauper cla\Flexa{má}[vit,] et Dóminus exau\MediatioIII{dí}{vit }{e}[um,] et de ómnibus tribulatiónibus eius salvá\TerminatioIIIa{vit}[ ]{e}um. \Antiphona{Fidélis servus et prudens}[\LinkLA]}%
    {\Inchoatio{Es}{te} pobre cla\Flexa{mou} e o Se\MediatioIII{nhor}{ o ou}{viu}[,] e de todas as tribulações o li\TerminatioIIIa{ber}{tou}. \Antiphona{Servo fiel e prudente}[\LinkPT]},
  % 16
  {\Inchoatio{Val}{lá}bit ángelus Dómini in circúitu ti\MediatioIII{mén}{tes }{e}[um,] et erípi\TerminatioIIIa{et}[ ]{e}os. \Antiphona{Fidélis servus et prudens}[\LinkLA]}%
    {\Inchoatio{O}[ ]{an}jo do Senhor acampa ao redor dos \MediatioIII{que}{ o }{te}[\-mem,] e ele os \TerminatioIIIa{li}{ber}ta. \Antiphona{Servo fiel e prudente}[\LinkPT]},
  % 17
  {\Inchoatio{Gu}{stá}te et vidéte quóniam su\MediatioIII{á}{vis est }{Dómi}[nus;] beátus vir qui sperat \TerminatioIIIa{in}[ ]{e}o. \Antiphona{Fidélis servus et prudens}[\LinkLA]}%
    {\Inchoatio{Pro}{vai} e vede como é \MediatioIII{bom}{ o Se}{nhor}[;] é feliz quem nele \TerminatioIIIa{con}{fi}a. \Antiphona{Servo fiel e prudente}[\LinkPT]},
  % 18
  {\Inchoatio{Ti}{mé}te Dóminum, \MediatioIII{sanc}{ti }{e}[ius,] quóniam non est inópia timénti\TerminatioIIIa{bus}[ ]{e}um. \Antiphona{Fidélis servus et prudens}[\LinkLA]}%
    {\Inchoatio{Te}{mei} o Senhor, \MediatioIII{vós}{, seus }{san}[tos,] pois nada falta aos que \TerminatioIIIa{o}[ ]{te}mem. \Antiphona{Servo fiel e prudente}[\LinkPT]},
  % 19
  {\Inchoatio{Dí}{vi}tes eguérunt et e\MediatioIII{su}{ri}{é}[runt,] inquiréntes autem Dóminum non defícient om\TerminatioIIIa{ni}[ ]{bo}no. \Antiphona{Fidélis servus et prudens}[\LinkLA]}%
    {\Inchoatio{Ri}{cos} empobrecem e \MediatioIII{pas}{sam }{fo}[me,] mas os que buscam o Senhor não sentirão falta \TerminatioIIIa{de}[ ]{na}da. \Antiphona{Servo fiel e prudente}[\LinkPT]}
}